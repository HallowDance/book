\chapter{More on T-duality and the TsT transformation}
\chapterauthor{Ivo Iliev\\Sofia University}
\adjustmtc
\minitoc
In this chapter  we will briefly delve into bosonic and fermionic $T$ duality
and the associated $TsT$ transformation in the context of a string $\sigma$
model.
\section{The notion of Bosonic and Fermionic $T$-duality}
Lets consider a generic (classical\footnote{A dilaton $\phi$ enters the string
action at a higher order in the coupling $\alpha'$. At the classical level
  the dilaton has to be introduced in the corresponding supergravity (e.g. the
  RR-forms appear always as $e^\phi F_{\mu_1,\dots,\mu_p}$). As we will not do
  explicit field redifinitions, we neglect it and its behaviour under $T$
  duality from the start. Working at the classical level we also disregard any
prefactors of the action and are only interested in its schematical form.})
string $\sigma$ model of the form
\begin{equation}
  S \propto \int\dd^2\sigma\partial_+ Z^M\mathcal{E}_{MN}(Z)\partial_- Z^{N}\equiv
  \int \dd^2\sigma\mathcal{L}, \quad M,N = 1,\dots,D,
\end{equation}
where we work in  conformal gauge for the sake of convenience, and understand
$Z^M$ as
\begin{equation}
  Z^M = (X^\mu(\sigma),\theta^\Delta(\sigma))
\end{equation}
with some bosonic fields $X^\mu$ and some fermionic Grassmann-valued fields
$\theta^\Delta$. We refer to the parity of the  coordinate $Z^M$ as $s(M)$.
$\mathcal{E}_{MN}(Z)$ is the background field describing the coupling between
the fields\footnote{$\mathcal{E}_{MN}$ could be decomposed into its graded
  symmetric (metric like) and graded skewsymmetric (B-field like) part:
  $\mathcal{E}_{MN} = \mathcal{G}_{MN} + \mathcal{B}_{MN}$. But only the order
  $\theta^0$ terms in $\mathcal{G}_{\mu\nu}$ and respectively $B_{\mu\nu}$ have
  direct physical interpretation as the components of the metric and the
  B field. We shall stick to the quite abstract "background" $\mathcal{E}_{MN}$
as it is practical and sufficient for our further considerations.}
with parity $s(\mathcal{E}_{MN}) = s(M) + s(N)$, so that $s(\mathcal{L}=0)$.
\par Now we assume the model has a manifest isometry and choose the associated
coordinate to be $Z^1$, meaning the symmetry is realised as a shift of $Z^1$.
We write $Z^M = (Z^1, Z^{\underline{M}})$ with $\underline{M} = 2,\dots,D$, so
that $\mathcal{E}_{MN} \equiv \mathcal{E}_{MN}(Z^{\underline{M}})$. $Z^1$ can
be either bosonic or fermionic\footnote{In the fermionic case the generator $Q$
  dual to the isometry coordinate has to anticommute with itself in order to
  correspond to a shift isometry. In other words, fermionic $T$-duality
requires a nullpotent supercharge $Q, Q^2=0$.}. This allows us to rewrite the
Lagrangian by introducing gauge fields $A_{\pm}$:
\begin{equation}
  \partial_{\pm}Z^1 \propto A_{\pm},\quad
  \mathcal{L}\rightarrow\mathcal{L}-\bar{Z}^1(\partial_+A_ - \partial_-A_+),
\end{equation}
where the Lagrange multiplier $\bar{Z}^1$ ensures $A_{\pm} = \partial_{\pm}Z^1$
by its equations of motion. Integrating out $A_{\pm}$ instead of $\bar{Z}^1$
yields the action of the dual model
\begin{equation}
  \bar{S}\propto\int\dd^2\sigma\partial_+\bar{X}^M\bar{\mathcal{E}}_{MN}\partial_-\bar{X}^N,
\end{equation}
