\chapter{Strings and Conformal Field Theory}
\chapterauthor{Ivo Iliev\\Sofia University}
\adjustmtc
\minitoc
\section{String theory as a CFT}
\par String theories in the conformal gauge are two-dimensional
conformal field theories. Thus, instead of the operatorial analysis, one can give an equivalent description
by using the language of conformal field theory in which one works with the
OPE rather then commutators or anticommutators and that contributes to
simplify many calculations. 
\subsection{Variables and coordinates in the CFT formulation}
In the case of a closed string it is convenient to
introduce the variables $z$ and $\bar{z}$ that are related to the world sheet variables
$\tau$ and $\sigma$ through a conformal transformation:
\begin{equation}
 z=e^{2i(\tau-\sigma)}\quad;\quad \bar{z} = e^{2i(\tau + \sigma)}
\end{equation}
\par In the case of an euclidean world sheet ($\tau \rightarrow  -i\tau$), $z$
and $\bar{z}$ are complex conjugates of each other. In terms of them we can write the bosonic coordinate $X^\mu$ as follows:
\begin{equation}
X^\mu =\left(z,\bar{z}\right) = \frac{1}{2}\left[X^\mu\left(z\right)+\tilde{X}^\mu\left(\bar{z}\right)\right]
\end{equation}
where
\begin{equation}
X^\mu\left(z\right) = \hat{q}^\mu - i\sqrt{2\alpha'}\log{z}\alpha_0^\mu
+ i\sqrt{2\alpha'}\sum_{n\neq 0}{\frac{\alpha_n^\mu}{n}z^{-n}} 
\label{1} 	
\end{equation}
and 
\begin{equation}
\tilde{X}^\mu\left(\bar{z}\right) = \hat{q}^\mu - i\sqrt{2\alpha'}\log{\bar{z}}\tilde{\alpha}_0^\mu + i\sqrt{2\alpha'}\sum_{n\neq 0}\frac{\tilde{\alpha}_n^\mu}{n}\bar{z}^{-n}  	
\end{equation}
with $\alpha_0^\mu = \tilde{\alpha}_0^\mu = \sqrt{\frac{\alpha'}{2}}\hat{p}^\mu$ 
\par In the case of \textbf{an open string theory} one can
introduce the variables:
\begin{equation}
 z=e^{i(\tau-\sigma)};\quad \bar{z} = e^{i(\tau+\sigma)}
\end{equation}
and the string coordinate can be written as
\begin{equation}
	X^\mu\left(z,\bar{z}\right) = \frac{1}{2}\left[X^\mu\left(z\right)+X^\mu\left(\bar{z}\right)\right]
\end{equation}
where $X^\mu$ is given in \eqref{1} and $\alpha_0^\mu = \sqrt{2\alpha'}\hat{p}^\mu$. In superstring theory
we must also introduce a conformal field with conformal dimension equal
to 1/2 corresponding to the fermionic coordinate. 
\par\textbf{In the closed string case}
we have two in-dependent fields for the holomorphic and anti-holomorphic
sectors which are obtained through the Wick
rotation  (\(\tau \rightarrow -i\tau)\) and the conformal transformation  (\(\tau,\sigma)\rightarrow\left(z,\bar{z}\right)\)
\begin{equation}
	\Psi^\mu\left(z\right) \sim \sum_t{\psi_tz^{-t-1/2}};\quad \tilde{\Psi}^\mu\left(\bar{z}\right) \sim \sum_t{\tilde{\psi}_t\bar{z}^{-t-1/2}}
\label{2}
\end{equation}
In the open string case, applying the same operations we get again eqs\eqref{2}, but this time with the same oscillators.
\par
In what follows we will explicitly consider only the \textbf{holomorphic sector for the closed string}. In the case of an open string it is sufficient to consider the string
coordinate at the string endpoint $\sigma = 0$. In both cases it is convenient to introduce a \textit{bosonic dimensionless variable}:
\begin{equation}
	x^\mu\left(z\right)\equiv \frac{X^\mu\left(z\right)}{\left(\sqrt{2\alpha'}\right)} = \tilde{q}^\mu -i\alpha_0^\mu\log z + i\sum_{n \neq 0}{\frac{\alpha_n}{n}z^{-n}}
\end{equation}
where $\tilde{q} = \hat{q}/\sqrt{2\alpha'}$ and a \textit{fermionic} one:
\begin{equation}
	\psi^\mu\left(z\right) = -i\sum_t{\psi_tz^{-t-1/2}}
\end{equation}
\subsection{Operator Product Expansion}
The theory can be quantized by imposing the following OPEs
\begin{equation}
	x^\mu\left(z\right)x^\nu\left(\omega\right) = -\eta^{\mu\nu}\log\left(z-\omega\right)+\cdots;\quad \psi^\mu\left(z\right)\psi^\nu\left(\omega\right) = -\frac{\eta^{\mu\nu}}{z-\omega}+\cdots ,
\end{equation}
where the dots denote finite terms for $z\rightarrow \omega$. In terms of the previous conformal fields we can define the generators
of superconformal transformations:
\begin{equation}
	G\left(z\right) = -\frac{1}{2}\psi.\partial x\quad; T\left(z\right) = T^x\left(z\right) + T^\psi\left(z\right) = -\frac{1}{2}\left(\partial x\right)^2 - \frac{1}{2}\partial\psi.\psi
\end{equation}
These conformal fields satisfy the following OPEs:
\begin{equation}
	T\left(z\right)T\left(\omega\right) = \frac{\frac{d}{d\omega}T\left(\omega\right)}{z-\omega} + 2\frac{T\left(\omega\right)}{\left(z-\omega\right)^2} + \frac{c/2}{\left(z-\omega\right)^4}+\cdots
\end{equation}
\begin{equation}
	T\left(z\right)G\left(\omega\right) = \frac{\partial/\partial\omega G\left(\omega\right)}{z-\omega} + \frac{3}{2}\frac{G\left(\omega\right)}{\left(z-\omega\right)^2} +\cdots
\end{equation}
\begin{equation}
	G\left(z\right)G\left(\omega\right) = \frac{2T\left(z\right)}{z-\omega} + \frac{d}{\left(z-\omega\right)^3}+\cdots
\end{equation}
We have to translate the $L_0$ operator in the R sector by a constant:
\begin{equation}
	L_0 \rightarrow L_0^{conf} \equiv L_0 + \frac{d}{16} = \sum_{n=1}^{\infty}{\left(\alpha_{-n}.\alpha_{n}+n\psi_{-n}.\psi_n\right)} +\alpha'p^2 + \frac{d}{16}
	\label{3}
\end{equation}
Therefore in the R sector we have two $L_0$ operators that are related by
eq.\eqref{3}. $L_0$ determines the spectrum of superstring while $L_0^{conf}$ encodes the correct conformal properties of the R sector.
\subsection{Primary fields \& Highest weight states}
In conformal field theory one introduces the concept of conformal or
\textit{primary field} 
\begin{definition}[Primary field]
  A conformal field $\Phi(z)$ is called a \textit{primary} field with dimension
  $h$ if it satisfies the following OPE with the energy-momentum tensor:
\begin{equation}
	T\left(z\right)\Phi\left(\omega\right) = \frac{\partial_\omega\Phi\left(\omega\right)}{z-\omega} + h\frac{\Phi\left(\omega\right)}{\left(z-\omega\right)^2}
\end{equation}
\end{definition}
From it one can compute the corresponding \textit{highest weight state} $\ket{\Phi}$ by means of the following limiting procedure
\begin{equation}
\ket{\Phi} = \lim_{z\rightarrow 0}\Phi\left(z\right)\ket{0}\quad,\quad\bra{\Phi} = \lim_{z\rightarrow 0}\bra{0}\Phi^{\dagger}\left(z\right)\sim\ \lim_{z\rightarrow \infty}\bra{0}\left(z^2\right)^h\Phi\left(z\right)
\end{equation}
The hermitian conjugate field $\Phi^{\dagger}$ in the previous expression has been defined as the field transformed under the conformal transformation $z\rightarrow 1/z$. It is easy to show that:
\begin{equation}
	L_0\ket{\Phi} = h\ket{\Phi}\quad;\quad L_n\ket{\Phi} = 0
\end{equation}
\subsection{BRST charge \& Vertex operators}
\par With the introduction of ghosts, the string action in the conformal gauge becomes invariant under the BRST transformations and the physical states are characterized by the fact that they are annihilated by the BRST charge that in the bosonic case is given by
\begin{equation}
	Q\equiv \oint{\frac{dz}{2\pi i }c\left(z\right)J_{BRST}} \equiv \oint{\frac{dz}{2\pi i }c\left(z\right)\left[T^x\left(z\right)  \frac{1}{2}T^{bc}\left(z\right)\right]}
\end{equation}
where $T^x\left(z\right)= -1/2\left(\partial x\right)^2$ and $T^{bc}\left(z\right) = \sum_n{L_n z^{-n-2}}$. The physical states are annihilated by the BRST charge
\begin{equation}
	Q\ket{\phi_{phys}} = 0
\end{equation}
By using the OPE it can be shown that in the bosonic string the most general BRST invariant \textit{vertex operator} has the following form
\begin{equation}
	\mathcal{W}= c\left(z\right)\mathcal{V}_\alpha^x\left(z\right)
\end{equation}
where $\mathcal{V}_\alpha^x\left(z\right)$ is a conformal field with dimension
equal to 1 that depends only on the string coordinate $x^\mu$.
\section{T Duality}
The compactification of a dimension in string theory is characterized by the appearance of new interesting phenomena with respect to those already present in field theory. In fact, in the case of a closed string, together with the Kaluza-Klein (K-K) excitations, a new kind of states called winding states appear in the spectrum. It turns out that the bosonic closed string
theory is invariant under the exchange of the winding modes with the K-K modes according to a transformation that is called T-duality. In the supersymmetric case, instead, this transformation is in general not a symmetry anymore but brings from a certain string theory to another string theory. For instance T-duality along a certain direction acts interchanging the IIA with the IIB theory. In the case of an open string, instead, this analysis
naturally leads to the existence of other objects called Dp-branes.
\subsection{Equations of motion \& momentum}
The most general solution of the eqs. of motion for the bosonic closed string can be written as:
\begin{multline}
	X^\mu\left(\tau,\sigma\right) = q^\mu + \sqrt{2\alpha'}\left(\alpha_0^\mu + \tilde{\alpha}_0^\mu\right)\tau - \sqrt{2\alpha'}\left(\alpha_0^\mu - \tilde{\alpha}_0^\mu\right)\sigma +\\ \\
	+i\sqrt{\frac{\alpha'}{2}}\sum_{n\neq 0}{\left(\frac{\alpha_n^\mu}{n}e^{-2in\left(\tau-\sigma\right)} + \frac{\tilde{\alpha}_n^\mu}{n}e^{-2in\left(\tau+\sigma\right)}\right)}
\end{multline}
where the momentum of the string is given by 
\begin{equation}
	p^\mu = \frac{1}{\sqrt{2\alpha'}}\left(\alpha_0^\mu+\tilde{a}_o^\mu\right)
\end{equation}
In the uncompactified case the two zero modes must be identified because the string coordinate must be invariant under $\sigma \rightarrow \sigma+\pi$ and the expression for the momentum becomes:
\begin{equation}
	p^\mu = \sqrt{\frac{2}{\alpha'}}\alpha_0^\mu
	\label{T1}
\end{equation}
\subsection{Compactification of a space dimension}
Let us compactify one of the space dimensions along a circle with radius R. For the coordinate X, corresponding to that dimension we have: 
\begin{equation}
	X\sim X + 2\pi R
\end{equation}
The conjugate momentum corresponding to the compactified direction must be quantized as:
\begin{equation}
	p = \frac{n}{R}\quad;\quad n \in Z
	\label{T2}
\end{equation}
Also, we have
\begin{equation}
	\pi\sqrt{2\alpha'}\left(a_0-\tilde{a_0}\right) = 2\pi\omega R\quad;\quad \omega\in Z
	\label{T3}
\end{equation}
 where $\omega$ corresponds to the number of times the close string winds around the compact direction. 
Equations \eqref{T1},\eqref{T2} and \eqref{T3} imply the zero modes for the compact direction must have the following expression
\begin{equation}
	\alpha_0 = \sqrt{\frac{\alpha'}{2}}\left(\frac{n}{R}+\frac{\omega R}{\alpha'}\right)\quad and\quad \tilde{\alpha}_0 = \sqrt{\frac{\alpha'}{2}}\left(\frac{n}{R}-\frac{\omega R}{\alpha'}\right) 
\end{equation}
For the Virassoro operator $L_0$, including the contribution from the uncompactified directions, we get:
\begin{equation}
	L_0 = \frac{\alpha'}{4}\hat{p}^2 + \frac{1}{2}\alpha_0^2 +\sum_{n=1}^\infty{\alpha_{-n}.\alpha_n} = \frac{\alpha'}{4}\hat{p}^2+\frac{\alpha'}{4}\left(\frac{n}{R}+\frac{\omega R}{\alpha'}\right)^2 + \sum_{n=1}^\infty{\alpha_{-n}.\alpha_n}
\end{equation}
and
\begin{equation}
 \tilde{L_0} = \frac{\alpha'}{4}\hat{p}^2 + \frac{\alpha'}{4}\left(\frac{n}{R}- \frac{\omega R}{\alpha'}\right)^2 + \sum_{n=1}^{\infty}{\tilde{\alpha}_{-n}.\tilde{\alpha_n}}
\end{equation}
\subsection{Mass operator \& winding modes}
The mass operator becomes
\begin{equation}
M^2 = \frac{2}{\alpha'}\left[\sum_{n=1}^\infty{\alpha_{-n}.\alpha_n+ \tilde{\alpha}_{-n}.\tilde{\alpha_n} -2}\right]+\left(\frac{n}{R}\right)^2 + \left(\frac{\omega R}{\alpha'}\right)^2
\label{T4}
\end{equation}
The last term corresponds to a new kind of excitations that are called \textit{winding modes} because they can be thought of as generated by the winding of the closed string around the compact direction.
\par From \eqref{T4} we see that the spectrum of the theory is invariant under the exchange of KK modes with winding modes together with an inversion of the radius of compactification:
\begin{equation}
	\omega \leftrightarrow n\quad ; \quad R\leftrightarrow \hat{R}\equiv\frac{\alpha'}{R}
\end{equation}
This is called a \textit{T-Duality} transformation and $\hat{R}$ is the compactification radius of the T-dual theory. \textbf{Correlators and the partition function are invariant under T-duality}. This means that T-duality is a symmetry of the bosonic closed string theory. As a consequence of this invariance, whenever we have to consider compactified theories,
we can limit ourselves to the case $R \geq \sqrt{\alpha'}$. That is the reason why $\sqrt{\alpha'}$ is often called the \textit{minimal length} of the string theory.
\subsection{Action of T-duality on the string coordinate}
Writing:
\begin{equation}
	X = \frac{1}{2}\left(X_-+X_+\right)
\end{equation}
where
\begin{equation}
 X_- = q +2\sqrt{2\alpha'}\left(\tau-\sigma\right)\alpha_0 + i\sqrt{2\alpha'}\sum_{n\neq 0}{\frac{\alpha_n}{n}e^{-2in\left(\tau-\sigma\right)}}	
\end{equation}
and 
\begin{equation}
 X_- = q +2\sqrt{2\alpha'}\left(\tau+\sigma\right)\tilde{\alpha}_0 + i  \sqrt{2\alpha'}\sum_{n\neq 0}{\frac{\tilde{\alpha}_n}{n}e^{-2in\left(\tau+\sigma\right)}}	
\end{equation}
it is evident that the T-dual coordinate must satisfy the conditions:
\begin{equation}
	\partial_\tau X \rightarrow \partial\tau\hat{X} = -\partial_\sigma X\quad ;\quad \partial_\sigma X \rightarrow \partial_\sigma\hat{X} = -\partial_\tau X 
\end{equation}
They are satisfied if the T-dual coordinate is equal to
\begin{equation}
	\hat{X} = \frac{1}{2}\left(X_- - X_+\right)
\end{equation}
Therefore the T-duality transformation acts on the right sector as a parity
operator changing sign of the right moving coordinate and leaving
unchanged the left moving one.
\subsection{T-duality in open string theory}
In an open string theory the string coordinate does not satisfy any periodicity
requirement on $\sigma$. This implies that in its compactified version
there are only K-K modes, while the winding modes are absent. This could
suggest that T-duality is not a symmetry of the open string theory. Such a
conclusion, however, leads to some problem when we remember that theories
with open strings also contain closed strings. Let us consider a theory
with open and closed strings with $d-p-1$ directions compactified on circles
with radii $R^l$ and take the limit
\begin{equation}
	R^l\rightarrow 0\quad \forall\quad\text{compact direction}
\end{equation}
In this limit the open string theory loses effectively $d-p-1$ directions. Therefore in this limit the open string will
appear to only be living in a $p+1$-dimensional subspace of the entire d-dimensional target space. We can conclude that But in we would end up with a theory in which open strings live in a p+1-dimensional subspace of the entire space-time, while closed strings live in the entire d-dimensional target space. This mismatch can be solved by requiring that, in the T-dual
picture, open string still can oscillate in d dimensions, while their endpoints are fixed on a $p+1$-dimensional hyperplane that we call Dp-brane. Open strings with their endpoints fixed on these hyperplanes satisfy Dirichlet boundary conditions in the $d-p-1$ transverse directions. We can extend the definition of a T-dual coordinate to the open string case. We obtain:
\begin{equation}
	\hat{X}^l = \frac{1}{2}\left[X_-^l - X_+^l\right]
\end{equation}
where now the left and right movers contain the same set of oscillators:
\begin{equation}
	X_-^l = q^l + c^l + \sqrt{2\alpha'}\left(\tau-\sigma\right)\alpha_-^l + i\sqrt{2\alpha'}\sum_{n\neq 0}{\frac{\alpha_n^l}{n}e^{-in\left(\tau-\sigma\right)}}
\end{equation}
and 
\begin{equation}
	X_+^l = q^l + c^l + \sqrt{2\alpha'}\left(\tau+\sigma\right)\alpha_-^l + i\sqrt{2\alpha'}\sum_{n\neq 0}{\frac{\alpha_n^l}{n}e^{-in\left(\tau+\sigma\right)}}
\end{equation}
One can immediately see that T-duality has transformed a string coordinate satisfying Neumann boundary conditions into a T-dual one satisfying Dirichlet boundary conditions. 
\subsection{Dp-branes}
The fact that open strings satisfy Dirichlet boundary conditions implies
the existence in the theory of objects, called the $Dp$-branes, that are characterized
by the fact that open strings have their endpoints attached to
them. It can be easily shown that:
\begin{equation}
	\hat{X}^l\left(\pi\right)\sim \hat{X}^l\left(0\right)
\end{equation}
That means that in the T-dual theory the two endpoints of the string are attached to the same D-brane. Up to now we have treated a $Dp$-brane as a pure geometrical hyperplane
to which open strings are attached and we have completely disregarded the
excitations of the attached open strings. But we will see that, as soon as
we let them come into play, they provide dynamical degrees of freedom to
the $Dp$-brane.
Among all possible excitations of an open string the massless ones have
the peculiarity of not changing the energy of the $Dp$-brane to which the
open string is attached. Therefore from the brane point of view they can
be interpreted as collective coordinates of the brane. For the treatment of compactification in the presence of Chan-Paton factors, see \cite{DBRANES1}.
\par The fundamental observation made by Polchinski has been to identify
the Dp-branes required by T-duality with the p-branes obtained as classical
solutions of the low-energy string effective action. Therefore, on the one
hand the p-branes are new non-perturbative states of string theory and on
the other hand have the important property that open string have their
endpoints attached on them. The latter property will allow one to compute
their interactions and more in general to study their properties by computing
open string one-loop diagrams. On the other hand we should not
be worried that the Dirichlet boundary conditions break Poincaré invariance
because this happens in presence of any kind of solitonic state. In the
next chapter we will introduce the boundary state and we will show that
it provides a stringy description of these new states.

\section{Bosonic Boundary State}
For now, we treat $Dp$-branes as static and rigid objects. The open string with the endpoint at $\sigma = 0$ attached to a $Dp$-brane
satisfies the usual Neumann boundary conditions along the directions longitudinal
to the world volume of the brane
\begin{equation}
	\partial_\sigma X^\alpha\vert_{\sigma=0}=0\quad\quad \alpha = 0,1,\cdots,p
	\label{eq:conditions}
\end{equation}
and Dirichlet boundary conditions along the directions transverse to the brane
\begin{equation}
	X^i\vert_{\sigma=0} = y^i\quad\quad i = p+1,\cdots,d-1
\end{equation}
where $y^i$ are the coordinates of the brane and $d$ is the dimension of the Minkowski spacetime, that in the case of the bosonic string is equal to $d=26$.
\par The interaction between two Dp-branes is given
by the vacuum fluctuation of an open string that is stretching between
them. This means that their interaction is simply given by the one-loop open string ”free-energy” which is usually represented by the annulus or
equivalently by the cylinder diagram. From either of those two diagrams it
is easy to see that by exchanging the variables $\sigma$ and $\tau$ the one-loop open
string amplitude can also be viewed as a tree diagram of a closed string
created from the vacuum, propagating for a while and then annihilating
again into the vacuum. These two equivalent descriptions of the same diagram
are called respectively the ‘open-channel’ and the ‘closed-channel’.
We want to stress that the physical content of the two descriptions is a
priori completely different. In the first case we describe the interaction between
two Dp-branes as a one-loop amplitude of open strings, which is the
amplitude of a quantum theory of open strings, while in the second case
we describe the same interaction as a tree-level amplitude of closed strings,
which is instead a classical amplitude in a theory of closed strings. The fact
that these two descriptions are equivalent is a consequence of the conformal
symmetry of string theory that allows one to connect the two a priori
different descriptions.
\subsection{Transformation from the open channel to the closed channel}
\par To show that, let us consider a one-loop diagram with an open string
circulating in it and stretching between two parallel Dp-branes with coordinates
respectively given by $(y^{p+1},\cdots{}, y^{d-1})$ and $(w^{p+1},\cdots{},w^{d-1})$. The open string satisfies the boundary conditions in \eqref{eq:conditions} both at $\sigma = 0$ and $\sigma = \pi$ along the world-volume directions of the brane, while along the transverse directions satisfies the following equations
\begin{equation}
X^i\vert_{\sigma=0} = y^i\quad X^i\vert_{\sigma=\pi} = \omega^i\quad i = p+1,\cdots{},d-1\quad ,
 \end{equation}
where we take $\sigma$ and  $\tau$ in the two intervals $\sigma\in\left[0,\pi\right]$ and $\tau\in\left[0,T\right]$.
\par The goal is to find a conformal transformation acting on the open string boundary conditions in order to transform them into the boundary conditions for a closed string propagating between two $Dp$-branes. In terms of the complex coordinate $\zeta\equiv\sigma+i\tau$ a conformal transformation simply transforms $\zeta\rightarrow f\left(\zeta\right)$, where $f\left(\zeta\right)$ is an arbitrary holomorphic function of $\zeta$. Lets consider the following transformation:
\begin{equation}
\zeta \rightarrow -i\zeta
\end{equation}
which in terms of $\left(\sigma,\tau\right)$ has the form:
\begin{equation}
\left(\sigma,\tau\right)\rightarrow\left(\tau,\sigma\right)
\end{equation}
In order to have the closed string variables $\sigma$ and $\tau$ vary in the intervals $\sigma\in\left[0,\pi\right]$ and $\tau\in \small[ 0,\hat{T}\small]$, corresponding to a closed string propagating between the two D branes, one must also perform the following conformal rescaling:
\begin{equation}
	\sigma \rightarrow\frac{\pi}{T}\sigma\quad\tau\rightarrow\frac{\pi}{T}\tau
\end{equation}
where $\hat{T} = \pi^2/T$.
\subsection{Boundary states}
Thus we have constructed a conformal transformation that brings us from the open string channel to the closed string channel. In the closed string channel we need to construct the two \textbf{boundary states}, $\ket{B_x}$, that describe the $Dp$-branes at $\tau = 0$ and $\tau = \hat{T}$ respectively. The equations that characterize these states are obtained by applying the conformal transformation previously constructed to the boundary conditions for the open string. At $\tau = 0$ we get the following conditions:
\begin{equation}
\partial_\tau X^\alpha\vert_{\tau=0}\ket{B_X} = 0\quad\alpha = 0,\cdots{},p
\end{equation}
\begin{equation}
X^i\vert_{\tau=0}\ket{B_X} = y^i\quad i=p+1,\cdots{},d-1
\end{equation}
Analogous equations can be obtained for the $D_p$-brane at $\tau = \hat{T}$.
\par The previous equations can be expressed using the closed string oscillators, obtaining:
\begin{equation}
\left(\alpha_n^\alpha + \tilde{\alpha}^\alpha_{-n}\right)\ket{B_X} = 0\quad;\quad \left(\alpha_n^i - \tilde{\alpha}^i_{-n}\right)\ket{B_X} = 0\quad \forall n \neq 0
\label{eq:overlap1}
\end{equation}
\begin{equation}
	\hat{p}^\alpha\ket{B_X} = 0 \quad \left(\hat{q}^i-y^i\right)\ket{B_X} = 0.
	\label{eq:overlap2}
\end{equation}
Introducing the matrix
\begin{equation}
	S^{\mu\nu} = \left(\eta^{\alpha\beta},-\delta^{ij}\right)
\label{smatrix}
\end{equation}
the equations for the non-zero modes can be rewritten as
\begin{equation}
	\left(\alpha^\mu_n + S^\mu_\nu\tilde{\alpha}^\nu_{-n}\right)\ket{B_X} = 0\quad \forall n \neq 0
\end{equation}
The state satisfying the previous condition can be determined to be:
\begin{equation}
	\ket{B_X} = N_p\delta^{d-p-1}\left(\hat{q}^i-y^i\right)\left(\prod_{n=1}^\infty{e^{-\frac{1}{n}\alpha_{-n}S.\tilde{\alpha}_{-n}}}\right)\ket{0}_{\alpha}\ket{0}_{\tilde{\alpha}}\ket{p=0},
\label{eq:staticboundarystate}
\end{equation}
where $N_p$ is an yet-to-be fixed normalization constant. We can find this constant by computing the interaction between the two parallel $Dp$-branes both in the open and the close string channel and comparing the results. We find that
\begin{equation}
	N_p = \frac{T_p}{2},\quad T_p = \frac{\sqrt{\pi}}{2^{\frac{d-10}{4}}}\left(2\pi\sqrt{\alpha'}\right)^{\frac{d}{2}-2-p}
\label{eq:normalization}
\end{equation}
\subsection{Ghost contributions to the boundary state}
A BRST invariant boundary state is in fact the \textbf{product} of the boundary state $\ket{B_X}$ for the bosonic coordinate, that was constructed in the previous section, and of $\ket{B_{gh}}$, the ghost contribution:
\begin{equation}
	\ket{B} = \ket{B_X}\ket{B_{gh}}
\end{equation}
BRST invariance requires that the total boundary state satisfies the equation
\begin{equation}
	\left(Q + \tilde{Q}\right)\ket{B} = 0,
	\label{eq:brstinvar}
\end{equation}
where the BRST charge is equal to
\begin{equation}
	Q = \sum_{n}{c_n}L_n^{X} + \sum_{n=-1}^{\infty}{c_{-n}L^{gh}_n} + \sum_{n=2}^{\infty}{L^{gh}_{-n}c_n}
\end{equation}
The overlap conditions \eqref{eq:overlap1} and \eqref{eq:overlap2} imply that the boundary state for the bosonic coordinate satisfies the following eqs:
\begin{equation}
	\left(L^X_m - \tilde{L}^X_{-m}\right)\ket{B_X} = 0
	\label{eq:bosonicoverlap}
\end{equation}
Using the expressions for $Q$ and $\tilde{Q}$ and using \eqref{eq:bosonicoverlap} we can see that \eqref{eq:brstinvar} implies the following overlap conditions for the ghost boundary state
\begin{equation}
	\left(c_n+\tilde{c}_{-n}\right)\ket{B_{gh}} = 0\quad ; \quad \left(b_n - \tilde{b}_{-n}\right)\ket{B_{gh}} = 0.
	\label{eq:ghostoverlap}
\end{equation}
Also
\begin{equation}
	\left(L^{gh}_m - \tilde{L}^{gh}_{-m}\right)\ket{B_{gh}} = 0
\end{equation}
Equations \eqref{eq:ghostoverlap} are satisfied by the state
\begin{equation}
	\ket{B_{gh}} = e^{\sum_{n=1}^{\infty}{\left(c_{-n}\tilde{b}_{-n}-b_{-n}\tilde{c}_{-n}\right)}}\left(\frac{c_0 + \tilde{c}_0}{2}\right)\ket{q=1}\tilde{\ket{q=1}}
\end{equation}
where $\ket{q=1}$ is the state, annihilated by the following oscillators
\begin{equation}
	c_n\ket{q=1} = 0\quad\forall n\geq 1\quad;\quad b_m\ket{q=1} = 0\quad\forall m\geq 0
\end{equation}

\section{Dynamic D Branes}

Until now, we've considered D$p$-branes as static and rigid objects to which open strings are attached. We have not discussed the fact that they can be boosted, rotated and the excitations of the attached open strings provide dynamical degrees of freedom to them.\cite{DBRANES2} In particular the massless excitations that have the property of not changing the energy of the brane, can be interpreted as collective coordinates of the D$p$-branes. In this sections, boosted and rotated boundary states are constructed. Also we will construct a boundary state that contains a constant abelian gauge field, living in the world volume of the brane. We will also show that some of those states are related to T-duality.

\subsection{Boosted and Rotated Boundary State}
In the previous chapters we have considered the boundary state, corresponding to a static D$p$-brane. We want to extend this notion to a boosted and a rotated boundary state. It is easy to see that a boost along a longitudinal direction of a brane does not modify the boundary state from the previous chapter as expected from Poincaré invariance of the classical solution along the longitudinal directions of the brane. Therefore, it is sufficient to concentrate on a boost along one the transverse directions. Let us call that direction $k$. The way to construct the boundary state is to once again start from the boundary conditions for an open string attached to such a D$p$-brane and then translate them into the language of the closed string channel via a conformal transformation and a conformal rescaling.
\subsection{Boosted Boundary State}
\par The boundary conditions for an open string attached to a D$p$-brane boosted with velocity $v$ in the direction of $k$ are\cite{Bachas95}

\begin{gather}
\partial_\sigma X^\alpha\vert_{\sigma=0} = 0\quad\quad\alpha = 1,\cdots{},p \\
\partial_\sigma\left(X^0 + vX^k\right)\vert_{\sigma=0} = 0 \\
X^i\vert_{\sigma=0} = y^i,\quad\quad i=p+1,\cdots{},D-1, \quad and\quad i\neq k \\
\left(\frac{X^k+vX^0}{\sqrt{1-v^2}}\right)\vert_{\sigma=0} = \frac{y^k}{\sqrt{1-v^2}}
\end{gather} 

where $\vec{y}$ is a vector belonging to the space transverse to the D$p$-brane and therefore has zero component along the time and the other world volume directions of the D$p$-brane. Translating to the \textbf{closed channel}, these conditions define the equations, characterizing the boosted boundary state $\ket{B,v,y}$ in the bosonic string:

\begin{gather}
\partial_\tau X^\alpha\vert_{\tau=0}\ket{B,v,y} = 0\quad\quad\alpha = 1,\cdots{},p\\
\partial_\tau\left(X^0 + vX^k\right)\vert_{\tau=0}\ket{B,v,y} = 0 \\
\left(X^i-y^i\right)\vert_{\sigma=0}\ket{B,v,y} = 0,\quad\quad i=p+1,\cdots{},D-1, \quad and\quad i\neq k \\
\left(\frac{\left(X^k-y^k\right)+vX^0}{\sqrt{1-v^2}}\right)\vert_{\sigma=0}\ket{B,v,y} = 0
\end{gather}
the last equation can be exchanged for a less restrictive one
\begin{equation}
\partial_\sigma\left(X^k+vX^0\right)\vert_{\tau=0}\ket{B,v} = 0,
\end{equation}
which corresponds to the case of \textit{a brane, delocalized in the} $k$ \textit{direction}.\\
\par The only overlap conditions that differ from those of the static case are those in the directions of the boost, namely the time and the $k$ directions and they are equal to

\begin{gather}
\left(\hat{p}^0 + v\hat{p}^k\right)\ket{B,v,y} = 0 \label{boundaryMomentum}\\
\left[\left(\alpha_n^0+\tilde{\alpha}_{-n}^0\right)+ v\left(\alpha_n^k+\tilde{\alpha}_{-n}^k\right) \right]\ket{B,v,y} = 0 \quad \forall n\neq 0 \label{boundaryNonzero1}\\
\frac{\hat{q}^k + vq^0}{\sqrt{1-v^2}}\ket{B,v,y} = \frac{y^k}{\sqrt{1-v^2}}\ket{B,v,y} \label{boundaryDelta}\\
\left[\left(\alpha_n^k\tilde{\alpha}_{-n}^k\right)+ v\left(\alpha_n^0+\tilde{\alpha}_{-n}^0\right) \right]\ket{B,v,y} = 0 \quad \forall n\neq 0
\label{boundaryNonzero2}
\end{gather}

From these conditions we can now determine the explicit expression for the boosted boundary state $\ket{B,v,y}$. For the zero mode part, \eqref{boundaryDelta} tells us that the boundary state must contain a delta function of the type

\begin{equation}
\delta\left(\frac{q^k+vq^0-y^k}{\sqrt{1-v^2}}\right) = \sqrt{1-v^2}\delta\left(q^k+vq^0-y^k\right)
\label{explicitDelta}
\end{equation}

Because the operator that acts on the boundary state in \eqref{boundaryMomentum} commutes with the $\delta$-function in \eqref{explicitDelta}, in order to satisfy \eqref{boundaryMomentum} it is sufficient to write the zero mode part as follows:

\begin{equation}
\sqrt{1-v^2}\delta\left(q^k+vq^0-y^k\right)\ket{p=0}
\label{explicitBoostedBoundatyZero}
\end{equation}

It is straightforward to check that it satisfies both zero mode eqs. \eqref{boundaryMomentum} and \eqref{boundaryDelta}. Now we must consider the non-zero modes part of the overlap conditions. It can be seen that in order to satisfy both \eqref{boundaryNonzero1} and \eqref{boundaryNonzero2}, the non-zero mode part of the boundary state must have the following structure

\begin{equation}
\prod_{n=1}^{\infty}\left(e^{-\frac{1}{n}\alpha_{-n}\cdot M(v)\cdot\tilde{\alpha}_{-n}}\right)\ket{0}_\alpha\ket{0}_{\tilde{\alpha}}
\label{explicitBoostedBoundatyNonZero}
\end{equation}
where the matrix $M$ is obtained from the matrix $S$ in \eqref{smatrix} by substituting its elements $\left(S_{00},S_{0k},S_{k0},S_{kk}\right)$ with the correspondent ones

\begin{equation}
M_{00}=M_{kk} = -\frac{1+v^2}{1-v^2}\quad ; \quad M_{0k}=M_{k0}= -\frac{2v}{1-v^2}
\end{equation}
Putting eqs. \eqref{explicitBoostedBoundatyZero} and \eqref{explicitBoostedBoundatyZero} together we get the final expression for the boosted boundary state:

\begin{multline}
\boxed{\ket{B,v,y} = \frac{T_p}{2}\prod_{i=p+1,i\neq j}^{d-1}\left[\delta\left(\hat{q}^i-y^i\right)\right]\sqrt{1-v^2}\delta\left(q^k+vq^0-y^k\right)
e^{-\sum_{n=1}^\infty{\frac{1}{n}\alpha_{-n}\cdot M\cdot\tilde{\alpha}_{-n}}}\ket{0}_\alpha\ket{0}_{\tilde{\alpha}}
}
\label{eq:boostedboundary}
\end{multline}
Although we have fixed the normalization factor to be $T_p/2$ as in the static case, the overlap conditions do not allow us to fix it uniquely. In general, the boundary state \eqref{eq:boostedboundary} could include an arbitrary function $N(v)$ of the physical velocity of the brane, that can only be determined by requiring agreement between the calculation of the interaction between two D-branes in the closed and open string channels. In this case, however, we have an independent way of uniquely fixing its normalization by applying to the static boundary state $\ket{B_X}$ in \eqref{eq:staticboundarystate} an operator that performs a boost along the direction $k$ transverses to the world volume of the D-brane

\begin{equation}
\ket{B,y,\omega} = e^{i\omega^kJ^0k}\ket{B,^{(\omega)}y},
\label{eq:staticboosted}
\end{equation}
where $\omega$ is related to the physical velocity via the relation

\begin{equation}
v = \tanh\omega
\end{equation}
and $^{(\omega)}y = y^k\cosh\omega$ is the boosted position of the D-brane and the generator of the Lorentz transformation is equal to

\begin{equation}
J^{\mu\nu} = q^\mu p^\nu - q^\nu p^\nu - i\sum_{n=1}^\infty\left(a_n^{\dagger \mu}a_n^\nu - a_n^{\dagger \nu}a_n^\mu + \tilde{a}_n^{\dagger \mu}\tilde{a}_n^\nu - \tilde{a}_n^{\dagger \nu}\tilde{a}_n^\mu \right)
\end{equation}
with $a_n\sqrt{n} = \alpha_n$ and $a_n^\dagger\sqrt{n} =\alpha_{-n}$ with $n>0$. After some algebra it can be seen that the boosted boundary state can in equation \eqref{eq:staticboosted} can be written in the following form

\begin{multline}
\ket{B,y,\omega(v)} = \frac{T_p}{2} \prod_{i=p+1,i\neq k}^{d-1}[\delta(\hat{q}^i - y^i)]\frac{1}{\cosh\omega}\delta(q^k + \tanh\omega q^0 - y^k)\times\\
\times e^{-\sum_{n=1}^\infty \frac{1}{n}\alpha_{-n}\cdot M(\omega)\cdot\tilde{\alpha}_{-n}}\ket{0}_\alpha \ket{0}_{\tilde{\alpha}}\ket{p=0},
\end{multline}
which exactly coincides with the previous expression, meaning that the overall normalization that we chose for the boosted state is correct.

\subsubsection{Rotated Boundary State}
The previous construction can easily be generalized for the case of a rotated D-brane. Obviously, the configuration of a D$p$-brane embedded in a $d$-dimensional space-time is invariant under rotations in the longitudinal space as well as in the transverse space. This implies that the boundary state is invariant under rotations in the planes $(\alpha,\beta)$ or $(i,j), \forall \alpha,\beta\in{0,\cdots,p}$ and $\forall i,j \in {p+1,\cdots,d-1}$. This means that in order to get a new boundary state, we must consider a D$p$-brane which is rotated with an angle $\omega$ in one of the planes specified by the directions $(\alpha,k)$.
\par The open string attached to this brane at $\sigma = 0$ satisfies the boundary conditions 
\begin{equation}
\partial_\sigma X^\beta\vert_{\sigma=0} = 0 \quad \forall \beta\in \{0,\cdots,p\},\beta\neq\alpha
\end{equation}
\begin{equation}
\partial_\sigma(X^\alpha \cos\omega + X^k\sin\omega)\vert_{\sigma=0} = 0
\end{equation}
\begin{equation}
X^i\vert_{\sigma=0} = y^i i = p+1,\cdots,D-1, i\neq k
\end{equation}
\begin{equation}
(X^k\cos\omega - X^\alpha\sin\omega - y^k\cos\omega)\vert_{\sigma=0} = 0
\end{equation}
The the rotated boundary state in the directions different from $(\alpha,k)$ must satisfy the same overlap conditions as the unrotated case. On the other hand, in the directions $(\alpha,k)$ we must impose the following conditions

\begin{equation}
\partial_\tau\left(X^\alpha\cos\omega + X^k\sin\omega\right)\vert_{\tau=0}\ket{B,\omega,y} - 0
\end{equation}
and
\begin{equation}
\left(X^k\cos\omega - X^\alpha\sin\omega - y^k\cos\omega\right)\vert_{\tau=0}\ket{B,\omega,y} = 0
\end{equation}
In terms of oscillators, the overlap conditions become:
\begin{gather}
\left(\hat{p}^\alpha\cos\omega + \hat{p}^k\sin\omega\right)\ket{B,\omega,y} =  0 \\
\left[\left(\alpha_n^\alpha + \tilde{\alpha}^\alpha_{-n}\right)+ \tan\omega\left(\alpha_n^k + \tilde{\alpha}^k_{-n}\right)\right]\ket{B,\omega,y} =  0 \quad\forall n\neq 0\\
\left(\hat{q}^k\cos\omega - \hat{q}^\alpha\sin\omega\right)\ket{B,\omega,y} = y^k\cos\omega\ket{B,\omega,y}\\
\left[\left(\alpha_n^k - \tilde{\alpha}^k{-n}\right)- \tan\omega\left(\alpha_n^\omega - \tilde{\alpha}^\omega_{-n}\right)\right]\ket{B,\omega,y} =  0\quad\forall n\neq 0
\end{gather}
From here, it can be shown that the rotated boundary state has the form
\begin{multline}
\ket{B,y,\omega(v)} = \frac{T_p}{2} \prod_{i=p+1,i\neq k}^{d-1}[\delta(\hat{q}^i - y^i)]\delta(q^k + \sin\omega q^\alpha - y^k\cos\omega)\times\\
\times e^{-\sum_{n=1}^\infty \frac{1}{n}\alpha_{-n}\cdot M(\omega)\cdot\tilde{\alpha}_{-n}}\ket{0}_\alpha \ket{0}_{\tilde{\alpha}}\ket{p=0},
\end{multline}
where in this case the matrix $M$ can be obtained from the matrix $S$, substituting its elements in the following way:

\begin{equation}
M_{\alpha\alpha} = - M_{kk} = \cos 2\omega\quad M_{0k} = M_{k0} = \sin 2\omega
\end{equation}
Once again, the boundary state for a rotated D-brane can be obtained by acting on the static boundary state in \eqref{eq:staticboundarystate} with the rotation operator
\begin{equation}
\ket{B,y,\omega} = e^{i\omega^kJ^\alpha k}\ket{B,^{(\omega)}y},
\end{equation}
where $J^{\mu\nu}$ is the previously defined quantity and $^{(\omega)}y^k = y^k\cos\omega$ is the rotated position of the D-brane. The previous considerations can be easily extended to the fermionic coordinate obtaining the boosted boundary state in the case of superstring.

\subsubsection{Compactified boundary state}

In this section we explore the case of the boundary state describing a D$p$-brane when all directions are compactified on circles. From it, by decompactifying some direction we can obtain the boundary state in which only some directions are compactified. For simplicity, we take all radii to be equal to $R$, but from the formulas that we will get, it will be trivial to extend the results to arbitrary radii. Firstly, it is convenient to introduce notation, such that we can distinguish between position and momentum operators for the momentum and winding degrees of freedom. We will require them to satisfy the following commutation relations:

\begin{equation}
\left[q^\mu_\omega,p^\nu_\omega\right] = i\eta^{\mu\nu}\quad ;\quad \left[q^\mu_n,p^\nu_n\right] = i\eta^{\mu\nu}
\end{equation}

Here the subscripts $n$ and $\omega$ omega correspond to the momentum and winding degrees of freedom respectively and all other commutators vanish. By denoting with $\ket{n^\mu,\omega^\nu}$ an eigenstate of the two "momentum" operators

\begin{equation}
p^\rho_n\ket{n,\omega} = \frac{n^\rho}{R}\ket{n,\omega}\quad ; \quad p^\rho_\omega\ket{n,\omega} = \frac{\omega^\rho R}{\alpha'}\ket{n,\omega}
\end{equation}

it is easy to convince oneself that the previous state can also be written as follows

\begin{equation}
\ket{n,\omega} = e^{iq_n\cdot n/R}e^{iq_w\cdot wR/\alpha'}\ket{0,0}
\label{eq:compactstate}
\end{equation}
 
where $\ket{0,0}$ is the state with zero momentum and winding number. The state in eq.\eqref{eq:compactstate} can be normalized as:

\begin{equation}
\bra{n,\omega}\ket{n',\omega'} = \Phi \delta_{nn'}\delta_{\omega\omega'}
\end{equation}

where $\Phi$ is the "self dual" volume that has the following properties:
\begin{equation}
\Phi = 2\pi R \enskip if\enskip R\rightarrow\infty \quad ; \quad \Phi=\frac{2\pi\alpha'}{R} \enskip if\enskip R\rightarrow 0
\label{eq:selfdualvolume}
\end{equation}

Let us use this formalism to write the boundary state in the compactifed case. In this case the part corresponding to the non-zero modes is unchanged, while the one corresponding to the zero modes of the bosonic coordinate becomes \cite{Frau97}

\begin{equation}
\ket{\Omega} = \mathcal{N}_p\prod_{\alpha=0}^p{\left[\sum_{\omega^\alpha}e^{i(q_\omega^\alpha - y^\alpha)\omega_\alpha R/\alpha'}\right]}\prod_{i=P+1}^{d-1}{\left[\sum_{n^i}e^{i(q_n^i - y^i)n_i/R}\right]}\ket{n=0,\omega=0}
\end{equation}
where the parameters $y^\alpha$ $y^i$ correspond respectively to Wilson lines\footnote{In gauge theory, a Wilson line, or Wilson loop is a gauge-invariant observable obtained from the holonomy of the gauge connection around a given loop. We will not be investigating this object any further as it is outside of the scope of the thesis. } turned on along the world volume of the brane to the position of the brane in the transverse directions.
\par
The previous boundary state satisfies the overlap conditions:
\begin{equation}
\left(e^{i(R/\alpha')q_\omega^\alpha} - e^{i(R/\alpha')y^\alpha} \right)\ket{\Omega} = p_n^\alpha\ket{\Omega} = 0,\enskip \alpha = 0,\cdots,p
\end{equation}
and
\begin{equation}
\left(e^{iq_n^i/R} - e^{iy^i/R} \right)\ket{\Omega} = p_\omega^i\ket{\Omega} = 0,\enskip i = p+1,\cdots,9-p
\end{equation}

The overall normalization can be determined by comparing the calculation of the brane interaction done in the closed and open string channels. In this way one gets the following relation \cite{Frau97}:

\begin{equation}
\mathcal{N}_p^2\frac{\alpha'}{4}\Phi^d = \frac{VC_1}{2\pi}
\label{eq:compactifiednormalization}
\end{equation}
where 
\begin{equation}
V = (2\pi R)^{p+1}\left(\frac{2\pi\alpha'}{R}\right)^{d-p-1}\quad;\quad C_1 = (2\pi)^{-d}(2\alpha')^{-d/2}
\end{equation}
From eq.\eqref{eq:compactifiednormalization} we get
\begin{equation}
\mathcal{N}_p = \sqrt{\frac{2VC_1}{\pi\alpha'(2\pi R)^d}}(2\pi R)^{d-p-1}\left[\left(\frac{2\pi R}{\Phi}\right)^{d/2}(2\pi R)^{p+1-d}\right]
\end{equation}

After some calculation it can be seen that the part outside of the square brackets is just the normalization of the boundary state in the uncompactifed case, therefore we get:
\begin{equation}
\mathcal{N}_p = \frac{T_p}{2}\left[\left(\frac{2\pi R}{\Phi}\right)^{d/2}(2\pi R)^{p+1-d}\right]
\end{equation}
where $T_p$ was defined back in \eqref{eq:normalization}. Now we can show that the factor $\mathcal{N}_p$ reduces to $T_p/2$ in the decompactified limit. In the decompactified limit, meaning $R\rightarrow\infty$ it can easily be checked that the following relations hold:
\begin{equation}
e^{i(q_\omega^\alpha-y^\alpha)\omega_\alpha(R/\alpha')}\ket{0,0}\rightarrow\ket{0,0}
\end{equation}
and
\begin{gather}
e^{i(q_n^i-y^i)n_i/R}\ket{0,0}\rightarrow R\int{dke^{i(q_n-y)k}}\ket{0,0} =\\
=(2\pi R)\int{\frac{dk}{2\pi}e^{i(q_n-y)k}}\ket{0,0} = 2\pi R\delta(q-y)\ket{0,0}.
\end{gather}
In the first relation we have taken into account that in the decompactification limit $R\rightarrow 0$ only the term with $\omega = 0$ survives and in the second relation we have substituted the summation with an integral by introducing $k=n/R$. Using the two previous relations and the first equation in eq. \ref{eq:selfdualvolume} we can see that the normalization factor indeed reduces to $T_p/2$ in the decompactified limit, which is the correct result. An interesting case arises when we decompactify only time and the transverse directions. The zero mode then becomes

\begin{equation}
\frac{T_p}{2}\left(\frac{2\pi R}{\Phi}\right)^{p/2}\prod_{\alpha=1}{p}\left[\sum_{\omega_\alpha}e^{i\theta^\alpha\omega^\alpha}\ket{n^\alpha=0,\omega^\alpha}\right]\ket{k^0 = 0}\prod_{i=p+1}^{d-1}{\left[\delta(q^y-y^i)\ket{k^i=0}\right]}
\end{equation}
where $\theta = -yR/\alpha'$.
\subsection{Interaction Between Branes}
In this section, the interaction between a D$p$ brane located at $y_1$ and a D$p'$ brane located at $y_2$, is studied. We consider the case with $NN\equiv min\left\{p,p';\right\}+1$ directions common to the brane world volumes,\\
$DD\equiv min\left\{d-p-1,d-p'-1;\right\}$ directions transverse to both, and $\nu = \left(d-NN-DD\right)$ directions of mixed type. We will not consider instantonic D-branes, hence also $NN\geq 1$. The two D-branes simply interact via tree-level exchange of closed strings whose propagator is
\begin{equation}
	D = \frac{\alpha'}{4\pi}\int{\frac{d^2z}{|z|^2}z^{L_0}\bar{z}^{\widetilde{L_0}}}
\end{equation}
so that indicating with $\ket{B_1}$ and $\ket{B_2}$ the boundary states describing the two D-branes the static amplitude is given by


\begin{equation}
A = \mel{B_1}{D}{B_2} = \frac{T_p T_{p'}}{4}\frac{\alpha'}{4\pi}\int_{\abs{z}\leq 1}{\frac{d^z}{\abs{z}^2}\mathcal{A}\mathcal{A^{(0)}}},
\end{equation}

where we have indicated with $\mathcal{A}$ and $\mathcal{A^{(0)}}$ respectively
the non zero mode and the zero mode contribution in which the previous
amplitude can be factorized. we do not have any intercept as we had for the
bosonic string because we assume that both $L_0$ and $\tilde{L}_0$ contain the
ghost degrees of freedom.
