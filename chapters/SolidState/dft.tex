\chapter{Introduction to Density Functional Theory}
\chapterauthor{Ivo Iliev\\Sofia University}
\adjustmtc
\minitoc
Density functional theory (DFT) is a computational quantum mechanical modelling
method used in physics, chemistry and materials science to investigate the
electronic structure (principally the ground state) of many-body systems, in
particular atoms, molecules, and the condensed phases. Let us consider
a $N$-electron system (atom, molecule or solid) in the Born-Oppenheimer and
non-relativistic approximations:

\begin{gather}
  \psi_{total} = \psi_{electronic} \otimes \psi_{nuclear}\nonumber\\
  v<<c
\end{gather}

\section{The many-body problem}
We can write the electronic Hamiltonian of such a system in the position
representation and atomic units as:

\begin{equation}
  \mathcal{H}(\vec{r}_1,\vec{r}_2,\cdots,\vec{r}_n)
  = -\frac{1}{2}\sum_{i=1}^{N}{\nabla_{r_i}^2
    + \frac{1}{2}\sum_{i=1}^N\sum_{\substack{j=0\\j\neq
      i}}^N\frac{1}{\abs{\vec{r}_i-\vec{r}_j}}
    + \sum_{i=1}^N{v_{\mathrm{ne}}(\vec{r}_i}),
\end{equation}
where $v_{\mathrm{ne}}=-\sum_{\alpha}Z_\alpha/\abs{\vec{r}_i-\vec{R}_\alpha}$ is
the nuclei-electron interaction ($\vec{R}_\alpha$ and $Z_\alpha$ are the
position and charges of the nuclei, respectively). From now on we will omit the
vector symbols for ease of notation. The stationary states are determined by
the time-independent Schrodinger equation

\begin{equation}
  \mathcal{H}(r_1,r_2,\cdots,r_N)\Psi(x_1,x_2,\cdots,x_N)
  = E\Psi(x_1,x_2,\cdots,x_N),
\end{equation}
where $\Psi(x_1,x_2,\cdots,x_N)$ is a wave function written with space spin
coordinates $x_i=(r_i,\sigma_i)$(with $r_i\in\mathbb{R}^3$ and $\sigma_i
=\uparrow\mathrm{ or }\downarrow$) which is antisymmetric with respect to the
change of two coordinates, and $E$ is the associated energy.

Using Dirac notation, the Schrodinger equation can be rewritten in
a representation-independent formalism

\begin{equation}
  \hat{\mathcal{H}}\bra{\Psi}=E\bra{\Psi},
\end{equation}
where the Hamiltonian is formally written as:
\begin{equation}
  \hat{\mathcal{H}}=\hat{T}+\hat{W}_{ee}+\hat{V}_{ne},
\end{equation}
where $\hat{T}$ is the kinetic-energy operator, $\hat{W}_{ee}$ is the
electron-electron interaction operator, and $\hat{V}_{ne}$ is the
nuclei-electron interaction operator. These operators can be expressed in
convinient ways in second quantization because of the fact that they will be
independent of the number of electrons (i.e. we work in Fock space). The
kinetic-energy operator becomes:
\begin{equation}
  \hat{T}
  = - \frac{1}{2}\sum_{\sigma=\uparrow,\downarrow}\int{\hat{\psi}^\dagger_\sigma(r)\nabla^2\hat{\psi}_\sigma(r)dr}
\end{equation}
$\hat{W}_{ee}$, the electron-electron operator, becomes:
\begin{equation}
  \hat{W}_{ee}
  = \frac{1}{2}\sum_{\sigma_1=\uparrow,\downarrow}\sum_{\sigma_2=\uparrow,\downarrow}\int\int\hat{\psi}^\dagger_{\sigma_2}(r_2)\hat{\psi}^\dagger_{\sigma_1}(r_1)w_{ee}(r_1,r_2)\hat{\psi}^\dagger_{\sigma_1}(r_1)\hat{\psi}^\dagger_{\sigma_2}(r_2)dr_1dr_2,
\end{equation}
with $w_{ee}(r_1,r_2) = \frac{1}{\abs{r_1-r_2}}$, and $\hat{V}_{ne}$ is the
nuclei-electron operator:
\begin{equation}
  \hat{V}_{ne}
  = \sum_{\sigma=\uparrow,\downarrow}\int\hat{\psi}^\dagger_\sigma(r)v_{ne}(r)\hat{\psi}_{\sigma}(r)dr.
\end{equation}
In these expressions, $\hat{\psi}^\dagger_\sigma(r)$ and
$\hat{\psi}^\dagger_\sigma(r)$ are the creation and annihilation field
operators, repsectively, which obey fermionic anticommutation rules:
\begin{gather}
  \{\hat{\psi}^\dagger_\sigma(r),\hat{\psi}^\dagger_{\sigma '}(r')\}=0\\
\{\hat{\psi}_\sigma(r),\hat{\psi}_{\sigma '}(r')\} = 0\\
\left\{\hat{\psi}^\dagger_\sigma(r),\hat{\psi}^\dagger_{\sigma'}(r')\right\}
= \delta(r-r')\delta_{\sigma\sigma'}
\end{gather}
It is also convinient to define the density operator
\begin{equation}
  \hat{n}(r)
  = \sum_{\sigma=\uparrow,\downarrow}\hat{\psi}^\dagger_\sigma(r)\hat{\psi}_\sigma(r),
\end{equation}
the one-particle density-matrix operator
\begin{equation}
  \hat{n}_1(r,r')
  = \sum_{\sigma=\uparrow,\downarrow}{\hat{\psi}^\dagger_\sigma(r')\hat{\psi}_\sigma(r)}
\end{equation}
and the pair-density operator
\begin{align}
  \hat{n}_2(r_1,r_2) &=
\sum_{\sigma_1=\uparrow,\downarrow}\sum_{\sigma_2=\uparrow,\downarrow}\hat{\psi}^\dagger_{\sigma_2}(r_2)\hat{\psi}^\dagger_{\sigma_1}(r_1)\hat{\psi}_{\sigma_1}(r_1)\hat{\psi}_{\sigma_2}(r_2)\nonumber\\
&= \hat{n}(r_2)\hat{n}(r_1) - \hat{r_1}\delta(r_1-r_2)\nonumber\\
&= \hat{n}(r_1)\hat{n}(r_2) - \hat{n}(r_1)\delta(r_1 - r_2)
\end{align}
Now we can write a compact form for the kinetic and interaction operators in
thethe Hamiltonian:
\begin{gather}
  \hat{T}
  = -\frac{1}{2}\int\left[\nabla^2_r\hat{n}_1(r,r')\right]_{r'=r}\mathrm{d}r,\\
  \hat{W}_{ee}
  = \frac{1}{2}\int\int\omega_{ee}(r_1,r_2)\hat{n}_2(r_1,r_2)\mathrm{d}r_1\mathrm{d}r_2,\\
  \hat{V}_{ne} = \int v_{ne}(r)\hat{n}(r)\mathrm{d}r.
\end{gather}
We can also use the second-quantization formalism in an orthonormal
spin-orbital basis $\{\psi_p(x)\}$, where $x=(r,\sigma)$. For this, we expand
the field operators as
\begin{equation}
  \hat{\psi}^\dagger_\sigma(r) = \sum_p{\psi_p^*(x)\hat{a}^\dagger_{p}}
\end{equation}
and
\begin{equation}
\hat{\psi}_\sigma(r) = \sum_p \psi_p(x)\hat{a}_p
\end{equation}
where $\hat{a}_p^\dagger$ and $\hat{a}_p$ are the creation and annihilation
operators in this basis, which still obey anticommutation rules:
$\left\{\hat{a}_p^\dagger, \hat{a}_q^\dagger\right\}
= \left\{\hat{a}_p,\hat{a}_q\right\} = 0$ and $\left\{\hat{a}_p^\dagger,
\hat{a}_q\right\} = \delta_{pq}$. The expressions of the operators are then
\begin{gather}
  \hat{T} = \sum_{pq}t_{pq}\hat{a}_p^\dagger\hat{a}_q\\
  \hat{W}_{ee}
  = \frac{1}{2}\sum_{pqrs}\bra{\psi_p\psi_q}\ket{\psi_r\psi_s}\hat{a}_p^\dagger\hat{a}_q^\dagger\hat{a}_s\hat{a}_r'\\
    \hat{V}_{ne} = \sum_{pq}v_{ne,pq}\hat{a}^\dagger_p\hat{a}_q
  \end{gather}
  where $t_{pq}$ and $v_{\mathrm{ne},pq}$ are the one-electron kinetic and
  nuclei-electron integrals, respectively, and
  $\bra{\psi_p\psi_q}\ket{\psi_r\psi_s}$ are the two-electron integrals.
\par The quantity of primary interest is the ground-state energy $E_0$. The
variational theorem establishes that $E_0$ can be obtained by the following
minimization
\begin{equation}
  E_0 = \mathrm{min}_{\Psi}\mel{\Psi}{\hat{H}}{\Psi},
\end{equation}
where the search is over all $N$-electron antisymmetric wave functions $\Psi$,
normalized to unity $\bra{\Psi}\ket{\Psi}= 1$. DFT is based on a reformulation of
the variational theorem in terms of the one-electron density defined
as\footnote{The integration over the spin coordinate $\sigma$ in the following
equation just means a sum over the two values $\sigma = \uparrow$ and $\sigma
= \downarrow$}
\begin{equation}
  n(r) = N\int\cdots\int
\abs{\Psi(x,x_2,\cdots,x_N}^2\mathrm{d}\sigma\mathrm{d}x_2\cdots\mathrm{d}x_N),
\end{equation}
which is normalized to the electron number, $\int n(r)\mathrm{d}r = N.$
\section{The universal density functional}
\subsection{The Hohenberg-Kohn theorem}
Consider an electronic system with an arbitrary external local potential $v(r)$
in place of $v_{\mathrm{ne}}(r)$. A corresponding ground-state wave function
$\Psi$ (there can be several of them if the ground staate is degenerate) can be
obtained by solving the Schrodinger equation, from which the associated
ground-state density $n(r)$ can be deduced. Therefore, one has a mapping from
the potential $v(r)$ to the considered ground-state density $n(r)$
\begin{equation}
  v(r)\rightarrow n(r).
\end{equation}
In 1964, Hohenberg and Kohn showed that this mapping can be inverted, i.e. the
ground-state density $n(r)$ determines the potential $v(r)$ up to an arbitrary
additive constant
\begin{equation}
  \boxed{
  n(r)\xrightarrow{Hohenberg-Kohn} v(r) + \mathrm{const}.}
\end{equation}
\begin{proof}
  The two-step proof by contradiction proceeds as follows. We consider two
  locallocal potentials $v_1(r)$ and $v_2(r)$ differing by more than an
  additive constant, $v_1(r) \neq v_2(r) +\mathrm{const}$, and we denote by
  $E_1$ and $E_2$ the ground-state energies of the Hamiltonians $\hat{H}_1
  = \hat{T} + \hat{W}_{ee} + \hat{V}_1$ and $\hat{H}_2 = \hat{T} + \hat{W}_{ee}+
  \hat{V}_2$, respectively.
  Now, assume that $\hat{H}_1$ and $\hat{H}_2$ have the same ground-state wave
  functions $\Psi$, i.e. $\hat{H}_1\ket{\Psi} = E_1\ket{\Psi}$ and
  $\hat{H}_2\ket{\Psi} = E_2\ket{\Psi}$. Then, substacting these two equations
  gives
  \begin{equation}
    (\hat{V}_1 - \hat{V}_2)\ket{\Psi} = (E_1 - E_2)\ket{\Psi},
  \end{equation}
  or, in position representation
  \begin{equation}
    \sum_{i=1}^{N}\left[v_1(r_i) - v_2(r_i)\right]\Psi(x_1,x_2,\cdots,x_N)
  = \left(E_1 - E_2\right)\Psi(x_1,x_2,\cdots,x_N)
  \end{equation}
\end{proof}
which implies $v_1(r) - v_2(r) = \mathrm{const}$, in contradiction with the
initial hypothesis. Note that, to eliminate $\Psi$ it was assumed that 
$\Psi(x_1,x_2,\cdots,x_N)\neq 0$ for all spatial coordinates
$(r_1,r_2,\cdots,r_N)$ and at least one fixed set of spin coordinates
$(\sigma_1,\sigma_2,\cdots,\sigma_N)$. This is in fact true "almost everywhere"
for "reasonably well behaved" potentials. In this case, we thus conclude that
two local potentials differing by more than an additive constant cannot share
the same ground-state wave function.
\par Let then $\Psi_1$ and $\Psi_2$ be (necessarily different) ground-state
wave functions of $\hat{H}_1$ and $\hat{H}_2$, respectively, and assume that
$\Psi_1$ and $\Psi_2$ have the same ground-state density $n(r)$. The
variational theorem leads to the following inequality. 
\begin{equation}
  E_1= \mel{\Psi_1}{\hat{H}_1}{\Psi_1}<\mel{\Psi_2}{\hat{H}_1}{\Psi_2}
  = \mel{\Psi_2}{\hat{H}_2 + \hat{V}_1 - \hat{V}_2}{\Psi_2} = E_2
  + \int\left[v_1(r) - v_2(r)\right]n(r)\mathrm{d}r,
\end{equation}
where the strict inequality comes from the fact that $\Psi_2$ cannot be
a ground-state
wave function of $\hat{H}_1$, as shown in the first step of the proof.
Symmetrically, by exchanging the role of system 1 and 2, we have the stict
inequality
\begin{equation}
  E_2 < E_1 + \int\left[v_2(r) - v_1(r)\right]n(r)\mathrm{d}r.
\end{equation}
Adding the two previous equations, we get
\begin{equation}
E_1 + E_2 < E_1 + E_2,
\end{equation}
which finally leads to the conclusion that there cannot exist two local
potentials differing by more than an additive constant which have the same
ground-state density. Note that this proof does not assume non-degenerate
ground states (contrary to the original Hohenberg-Kohn proof) \blacksquare

So, the ground-state density $n(r)$ determinese the potential $v(r)$, which in
turn determines the Hamiltonian, and thus everything about the many-body
problem. In other words, the potential $v$ is a unique (up to an additive
constant) functional of the ground-state density $n$, and all other properties
as well. The ground-state wave function $\Psi$ for the potential $v(r)$ is
itself a functional of $n$, denoted by $\Psi[n]$, which was exploited by
Hohenberg-Kohn to define the universal (i.e., independent from the external
potential) density functional
\begin{equation}
  F[n] = \bra{\Psi[n]}\hat{T} + \hat{W}_{ee}\ket{\Psi[n]},
\end{equation}
which can be used to define the total electronic energy functional
\begin{equation}
  E[n] = F[n] + \int v_{ne}(r)n(r)\mathrm{d}r,
\end{equation}
for the specific potential $v_{ne}(r)$ of the  system considered. Note that,
forfor degenerate ground states, $\Psi[n]$ is not unique but stands for any
degenerate ground-state wave function. However, all $\Psi[n]$ give te same
$F[n]$, which is thus a unique functional of $n$.
\par Hohenberg and Kohn further showed that the density functional $E[n]$
satisfies a variational property: the ground-state energy $E_0$ of the system
considered is obtained by minimizing this functional with respecst to
$N$-electr-electron densities $n$ that are ground-state densities associated
with some local potential (referred to as $v$-representable densities).
\begin{equation}
  E_0 = \underset{n}{\mathrm{min}}\left\{F[n] + \int
  v_{ne}(r)n(r)\mathrm{d}r\right\},
\end{equation}
the minimum being reached for a ground-state density $n_0(r)$ corresponding to
the potential $v_{ne}(r)$.
The existance of a mapping from a ground-state density to a local potential,
the existance of the universal density functional, and the variational property
with respect to the density constitutes the \textit{Hohenberg-Kohn theorem}.
