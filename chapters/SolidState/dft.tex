\chapterauthor{Ivo Iliev}{Sofia University}

\chapter{Introduction to Density Functional Theory}
 
Density functional theory (DFT) is a computational quantum mechanical modelling
method used in physics, chemistry and materials science to investigate the
electronic structure (principally the ground state) of many-body systems, in
particular atoms, molecules, and the condensed phases. Let us consider
a $N$-electron system (atom, molecule or solid) in the Born-Oppenheimer and
non-relativistic approximations:

\begin{gather}
  \psi_{total} = \psi_{electronic} \otimes \psi_{nuclear}\nonumber\\
  v<<c
\end{gather}

We can write the electronic Hamiltonian of such a system in the position
representation and atomic units as:

\begin{equation}
  \mathcal{H}(\vec{r}_1,\vec{r}_2,\cdots,\vec{r}_n)
  = -\frac{1}{2}\sum_{i=1}^{N}{\nabla_{r_i}^2
    + \frac{1}{2}\sum_{i=1}^N\sum_{\substack{j=0\\j\neq
      i}}^N{\frac{1}{\abs{\vec{r}_i-\vec{r}_j}}
        + \sum_{i=1}^N{v_{\mathrm{ne}}(\vec{r}_i}),

\end{equation}
where $v_{\mathrm{ne}}=-\sum_{\apha}Z_\alpha/\abs{\vec{r}_i-\vec{R}_\alpha}$ is
the nuclei-electron interaction ($\vec{R}_\alpha$ and $Z_\alpha$ are the
position and charges of the nuclei, respectively). From now on we will omit the
vector symbols for ease of notation. The stationary states are determined by
the time-independent Schrodinger equation

\begin{equation}

  \mathcal{H}(r_1,r_2,\cdots,r_N)\Psi(x_1,x_2,\cdots,x_N)
  = E\Psi(x_1,x_2,\cdots,x_N),

\end{equation}
where $\Psi(x_1,x_2,\cdots,x_N)$ is a wave function written with space spin
coordinates $x_i=(r_i,\sigma_i)$(with $r_i\in\mathbb{R}^3$ and $\sigma_i
=\uparrow\mathrm{ or }\downarrow$) which is antisymmetric with respect to the
change of two coordinates, and $E$ is the associated energy.

Using Dirac notation, the Schrodinger equation can be rewritten in
a representation-independent formalism

\begin{equation}
  \hat{\mathcal{H}}\bra{\Psi}=E\bra{\Psi},
\end{equation}
where the Hamiltonian is formally written as

\begin{equation}
  \hat{\mathcal{H}}=\hat{T}+\hat{W}_{ee}+\hat{V}_{ne},

\end{equation}
where $\hat{T}$ is the kinetic-energy operator, $\hat{W}_{ee}$ is the
electron-electron interaction operator, and $\hat{V}_{ne}$ is the
nuclei-electron interaction operator. These operators can be expressed in
convinient ways in second quantization because of the fact that they will be
independent of the number of electrons (i.e. we work in Fock space). The
kinetic-energy operator becomes:
\begin{equation}
  \hat{T}
  = - \frac{1}{2}\sum_{\sigma=\uparrow,\downarrow}\int{\hat{\psi}^\dagger_\sigma(r)\nabla^2\hat{\psi}_\sigma(r)dr

\end{equation}

In `
