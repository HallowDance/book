\section{$T\bar{T}$ deformation}
In some sense all QFT's that we use in physics are \textit{effective} field
theories - they are valid only over some range of energy (respectively, length)
scales. It has been shown that if the theory is renormalizable, this range of
scales can be very large, thus making the theory more applicable and
predictive. Still, even for (perturbatively) renormalizable theories, new
physics enters the stage at some point, usually at some UV point (or region).
Non-renormalizable theory, meaning a theory where the  action involves 
operators with dimension greater than the spacetime dimension ($\Delta > d$)
still can be applicable up to some energy scale (the so called "UV cut-off")
$\Lambda$.
Investigation of non-renormalizable theories has led to the idea of the "UV
completion" - a way to make sense of some theories at higher energies. For
example a theory might flow from a non-trivial RG fixed point, an effect
called "asymptotic safety". (revize) Another possibility is that the UV limit
is not a conventional UV fixed point corresponding to a local QFT but is
something else entirely (eg \textit{String Theory}). The $T\bar{T}$ deformation
of 2d QFT is an example of a non-renormalizable deformation of a local QFT for
which many physical quantities make sense and are finite and calculable in
terms of the data of the undeformed theory. This deformation is very special in
itself, it has been termed 'asymptotic fragility' - which in turn can be used
as a constraint on physical theories.


\subsection{Overview}
Consider a sequence of 2d Eucleadian field theories
$\mathcal{T}\;(t\in\mathbb{R})$ in a domain endowed with a flat Eucleadian
metric $\eta_{ij}$, each with a local stress-energy tensor
\begin{equation}
  T_{ij}^{(t)}(x)\sim \delta S^{(t)}/ \delta g^{ij}(x)
\end{equation}
Imagine now that the first object in the sequence, $\mathcal{T}^{(0)}$ is
a conventional local QFT (massive) or a CFT (massless). We can define the
deformation for our theories
\begin{equation}
  S^{(t+\delta t)} = S^{(t)} - \delta t\int{\det T^{(t)} d^2 x}
\end{equation}
equivalently we can arrive at this by inserting $\int\det Td^2x$ into the
correlation functions. We can calculate the determinant:
\begin{equation}
  \det T = \frac{1}{2}\epsilon^{ik}\epsilon^{jl} T_{ij}T_{kl} \propto
  T_{zz}T_{\bar{zz}} - T^2_{z\bar{z}}
  \label{eq:ttdeterminant}
\end{equation}
where \eqref{eq:ttdeterminant} is written in complex
(holomorphic-antiholomorhic) coordinates. In the language of conformal field
theories this can be written as $T\bar{T}$. Obviously the dimension of this
deformation is 4, hence one would expect that $\langle \det T\rangle \sim
\Lambda^4$. The first very interesting result using these types of
transformations came from Zamolodchikov (2004). By requiring that the
stress-energy tensor is conserved, $\partial^i T_{ij} =0$, we can show that
\begin{equation}
  \frac{\partial}{\partial y_m}\epsilon^{ik}\epsilon^{jl}T_{ij}(x)T_{kl}(x+y)
  = \frac{\partial}{\partial x_i}\epsilon^{mk}\epsilon^{jl}T_{ij}T_{kl}(x+y)
\end{equation}
Whilst at first glance this does not look very interesting, one can make the
following statement:
\begin{tcolorbox}
  For any transitionally invariant state the vacuum expectation value
  \begin{equation}
    \langle \epsilon^{ik}\epsilon^{jl}T_{ij}(x)T_{kl}(x)\rangle = \langle
    \epsilon^{ik}\epsilon^{jl}T_{ij}(x)T_{kl}(x+y)\rangle
    \label{eq:ttinvariant}
  \end{equation}
\end{tcolorbox}
The vacuum expectation value in $\eqref{eq:ttinvariant}$ calculable and finite
in terms of the matrix elements of $T_{ij}$. In some interpretation this means
that the $T\bar{T}$ transformation is solvable and integrable, something that
is in itself a very rare occurrence in quantum field theories. We will now
follow Zamolodchikov's paper, omitting some extra detail and focusing on the
$T\bar{T}$ deformations in terms  of conformal field theories.

\subsection{One-point expectation functions}
The one-point expectation values $\langle \mathcal{O}_i\rangle$ of local fields
control the linear reaction of the system to external forces which couple to
the fields $\mathcal{O}_i$(z). Also, in view of the operation-product
expansions (OPE):
\begin{equation}
 \cfto_i(z)\cfto_j(z') = \sum)_k C_{ij}^k(z-z')\cfto_k(z')
 \label{eq:operationproductexpansion}
 \end{equation}
 the two-point correlation functions $\langle \cfto_i(z)\cfto_j(z')\rangle$
 (and, by extension and repeated application of
 \eqref{eq:operationproductexpansion}, all higher-point correlation functions)
 are expressed through the OPE structure functions $C_{ij}^k(z-z')$ and the
 one-point expectation values $\langle\cfto_k\rangle$. While this is an
 interesting result, it is seldom used for calculation because even though the
 structure functions (which describe local dynamics of the field theory)
 usually admit perturbative expansions, the one-point expectation values
 (incorporation information about the vacuum state of the theory) are
 typically nonperturbative quantities and no general to their systematic
 evaluation is known.\footnote{In 2d, fairly accurate numerical estimates can
   be  obtained for a lot of cases via a version of the Truncated Conformal
 Space Approach}. Even before Zamolodchikov it was shown that for some specific
 models (namely the sine-Gordon model and some minimal CFT's) the expectation
 value of the composite field $T\bar{T}$ is related to the trace component of
 the  stress-energy tensor:
 \begin{equation}
   \langle T\bar{T}\rangle = -\langle \Theta \rangle^2,
   \label{eq:specifictt}
 \end{equation}
 where $\Theta = \frac{\pi}{2}T^\mu_\mu$.
 Zamolodchikov showed that this equations is valid for a broad spectrum of 2d
 quantum field theories, including some theories that are not required to be
 integrable. Instead of considering an infinite Eucleadian plane, let us
 consider a field theory on an infinite cylinder, with one of the Eucleadian
 axis compactified on a circle (the so-called Matsubara representation of field
 theories at finite temperature). It can be shown that the last equation can be
 generalized to:
 \begin{equation}
   \langle T\bar{T}\rangle = \langle T\rangle \langle \bar{T}\rangle
   - \langle{\Theta}\rangle\langle\Theta\rangle.
   \label{eq:generaltt}
 \end{equation}
 When the circumference of the cylinder goes to infinity (equivalently, the
 temperature goes to zero) the global rotational symmetry is  restored, making
 the expectation value of the chiral components $T$ and $\bar{T}$ vanish - this
 limit \eqref{eq:generaltt} reduces to \eqref{eq:specifictt}. It can also be
 argued that the vacuum expectation values $\langle\cdots\rangle
 = \mel{0}{\cdots}{0}$ are replaced by more general diagonal elements
 $\mel{n}{\cdots}{n}$, where $\ket{n}$ is any non-degenerate eigenstate of the
 energy and momentum operators (in the case of the cylinder, to make this
 statement precise one has to take the Hamiltonian picture in which the
 coordinate along the cylinder is taken as the Eucleadian time). Note that we
 consider quantum field theory in flat 2d space in terms of Eucleadian version
 of the theory. The points $z$ of the 2d space can be labeled by the Cartesian
 coordinates $(x,y)$, but we usually use complex coordinates
 $z=(\mathrm{z},\mathrm{\bar{z}}) = (\mathrm{x}+i\mathrm{y},
 \mathrm{x}-i\mathrm{y})$. The usual normalization of the energy-momentum
 tensor $T_{\mu\nu}$ is assumed - for instance, in the picture where
 $\mathrm{y}$ is taken as the Eucleadian time, $-T_{yy}$ coincides with the
 energy density. The chiral components $T,\bar{T},\Theta$ are normalized
 according to the CFT convention, namely
 $T=-(2\pi)T_{\mathrm{zz}},\;\bar{T}=-(2\pi)T_{\bar{\mathrm{z}}\bar{\mathrm{z}}},\;\Theta=(2\pi)T_{z\bar{z}}$.
 \subsection{Assumptions and main idea of the argument}
 Some of the assumptions in this section concern the local dynamics of the
 field theory, and others  will be about the global settings. To make the
 proper distinction, we shall label the former with (L) and the latter with
 (G).
 \begin{enumerate}
  \item (L). Local translational and rotational symmetry. This implies
    existence of local field $T_{\mu\nu}$ (the energy-momentum tensor) which is
    symmetric, $T^{\mu\nu}(z) = T^{\nu\mu}(z)$, and satisfies the continuity 
    equation $\partial_\mu T^{\mu\nu}(z)=0$. In terms of the conventional
    chiral components $T = -2\pi T_{zz}, T=-2\pi T_{\bar{z}\bar{z}}$ and 
    $\Theta = 2\pi T_{z\bar{z}}=2\pi T_{\bar{z}z}$ the continuity equation is
    written as
    \begin{gather}
      \partial_{\bar{z}}T(z) = \partial_z \Theta(z),\\
      \partial_z\bar{T}(z) = \partial_\bar{z} \Theta(z).
      \label{eq:continuitytheta}
    \end{gather}
    This assumption is already taken into account in writing the OPE
    \eqref{eq:operationproductexpansion}, where the structure functions
    $C^k_{ij}$ are assumed to depend on the separations $z-z'$ only.
  \item (G). Global translational symmetry. It is assumed that for any
    local field $\cfto_i(z)$ the expectation value $\langle \cfto_i(z)\rangle$
    is a constant independent of $z$. It follows from
    \eqref{eq:operationproductexpansion} that the two-point correction
    functions depend only on the separations,
    \begin{equation}
      \langle \cfto_i(z)\cfto_j(z')\rangle = G_{ij}(z-z').
      \label{eq:cftseparations}
    \end{equation}
  \item (G). Infinite separations. We assume that at least one direction (i.e.
    Eucleadian vector $e = (\mathrm{e},\bar{\mathrm{e}})$) exists, such that
    for any $\cfto_i$ and $\cfto_j$
    \begin{equation}
      \lim_{t\rightarrow\infty}\langle \cfto_i(z+et)\cfto_j(z') \rangle
        = \langle \cfto_i\rangle\langle \cfto_j\rangle.
        \label{eq:infiniteseparations}
    \end{equation}
    The "global" assumptions 2 and 3 imply that the underlying geometry of 2D
    space is either an infinite plane, or an infinitely long cylinder.
  \item (L) CFT limit at short distances. We will assume that the
    short-distance behaviour of the field theory is governed by a conformal
    field theory, and that certain no-resonance condition is satisfied. We will
    detail the content of this assumptions later. This assumption is needed in
    order to make the definition of the composite field $T\bar{T}$ essentially
    unambiguous\footnote{There is intrinsic ambiguity in adding certain total
      derivatives, which does not affect the expectation value $\langle
      T\bar{T}\rangle$.}
  \end{enumerate}

  The main idea of the arguments stems from a simple identity involving
  two-point correlation functions of the energy-momentum tensor, consequence of
  the assumptions 1-3 alone. Consider the following combination of two-point
  correlation functions
  \begin{equation}
    \mathcal{C} = \langle T(z)\bar{T}(z') \rangle - \langle \Theta(z)
    \Theta(z')\rangle.
    \label{eq:constantc}
  \end{equation}
  Now, we take the derivative $\partial_{\bar{z}}$ of the above and transform it
  as follows. In the first term, we use \eqref{eq:continuitytheta} to replace
  the derivatives to replace the derivative $\partial_{\bar{z}}T(z)$ by
  $\partial_z\Theta(z)$, and then apply \eqref{eq:cftseparations} to
  move the derivative to the second entry $\bar{T}(z')$. When the
  derivative $\partial_\bar{z}\Theta(z)$ in the second term is also
  moved to $\Theta(z')$, one finds
  \begin{equation}
  \langle \partial_{\bar{z}}T(z)\bar{T}(z')
  - \partial_{\bar{z}}\Theta(z)\Theta(z')\rangle = \langle
    -\Theta(z)\partial_{z'}\bar{T}(z')
    + \Theta(z)\partial_{\bar{z}'}\Theta(z')\rangle = 0,
  \end{equation}
  where the equation \eqref{eq:continuitytheta} was used again in the last
  step. By similar transformations it can be shown that the derivative
  $\partial_z$ also vanishes, and hence the quantity $\mathcal{C}$ is
  a constant, independent of the coordinates.
  \par Note that in this derivation, only assumptions 1 and 2 are used. Adding
  assumption 3 allows one to relate this constant to the one-point expectation
  values of the fields involved. Taking the limit
  \eqref{eq:infiniteseparations} of the right-hand side of
  \eqref{eq:constantc}, one finds:
  \begin{equation}
    \mathcal{C} = \langle T\rangle \langle \bar{T}\rangle - \langle
    \Theta\rangle \langle \Theta\rangle.
  \end{equation}
  On the other hand, some reflection about equation \eqref{eq:constantc} seems
  to hint to the fact that the constant $\mathcal{C}$ coincides with the
  expectation value of appropriately defined composite operator $T\bar{T}$.
  Indeed, one expects that the composite field $T\bar{T}$ can be obtained 
  in some way form the product $T(z)\bar{T}(z')$ by bringing the points 
  $z$ and $z'$ together. The main obstacle is in the presence of singular
  terms in the operator product expansion of $T(z)\bar{T}(z')$, which makes
  straightforward limits impossible. We will show in the next section that the
  second term in the combination $T(z)\bar{T}(z') - \Theta(z)\Theta(z')$
  exactly subtracts these singular terms, so that the limit $z\rightarrow z'$
  in \eqref{eq:constantc} can be taken, leading to \eqref{eq:generaltt}.
  \subsection{Operator Product Expansion}
  It is not difficult to repeat manipulations of the previous section, this time
  working not with the two-point functions \eqref{eq:constantc}, but with
  the combination of the operator products $T(z)\bar{T}(z')
  - \Theta(z)\Theta(z')$ itself. Using \eqref{eq:continuitytheta} one finds
  \begin{multline}
    \partial_{\bar{z}}\left(T(z)\bar{T}(z')-\Theta(z)\Theta(z')\right)
    =
    (\partial_z + \partial_{z'})\Theta(z)\bar{T}(z')
    - (\partial_{\bar{z}}+\partial_{\bar{z}'})\Theta(z)\Theta(z'),
    \label{eq:ope3.1}
  \end{multline}
  and
  \begin{equation}
    \partial_z(T(z)\bar{T}(z') - \Theta(z)\Theta(z')) = 
    (\partial_z + \partial_{z'})T(z)\bar{T}(z') - (\partial_{\bar{z}}
    + \partial_{\bar{z}'})T(z)\Theta(z').
    \label{eq:ope3.2}
  \end{equation}
  The meaning of these expressions becomes clearer after inserting the operator
  product expansions
  \begin{gather}
    \Theta(z)\bar{T}(z') = \sum_i B_i(z-z')\cfto_i(z'),\\
    \T(z)\Theta(z') = \sum_i A_i(z-z')\cfto_i(z'),
  \end{gather}
  and
  \begin{gather}
    T(z)\bar{T}(z') = \sum_i D_i(z-z')\cfto_i(z'),\\
    \Theta(z)\Theta(z') = \sum_i C_i(z-z')\cfto_i(z'),
    \label{eq:opetheta15}
  \end{gather}
  where the sums involve complete sets of local fields
  $\left\{\cfto_i\right\}$. The equations \eqref{eq:ope3.1}, \eqref{eq:ope3.2} then
  read 
  \begin{gather}
    \sum_i \partial_{\bar{z}} F_i (z-z')\cfto_i(z') =\nonumber\\
    \sum_i\left(B_i(z-z')\partial_{z'}\cfto_i(z')
    - C_i(z-z')\partial_{\bar{z}'}\cfto_i(z')\right),
    \label{eq:ope3.1final}
  \end{gather}
  \begin{gather}
    \sum_i\partial_z F_i(z-z')\cfto_i(z') = \nonumber\\
    \sum_i\left(D_i(z-z')\partial_{z'}\cfto_i(z')
    - A_i(z-z')\partial_{\bar{z}'}\cfto_i(z')\right),
    \label{eq:ope3.2final}
  \end{gather}
  where 
  \begin{equation}
    F_i(z-z') = D_i(z-z') - C_i(z-z')
  \end{equation}
  Note that the right-hand side of equations \eqref{eq:ope3.1final} and
  \eqref{eq:ope3.2final} involve only coordinate derivatives of local
fields. It follows that any operator $\cfto_i$ appearing in the expansion
\begin{equation}
  T(z)\bar{T}(z') - \Theta(z)\Theta(z') = \sum_i F_i(z-z')\cfto_i(z'),
  \label{eq:ttexpansion}
\end{equation}
unless itself is a coordinate derivative of another local operator, comes with
a constant (i.e. coordinate-independent) coefficient $F_i$. In other words, the
operator product expansion can be written as:
\begin{equation}
  T(z)\bar{T}(z') - \Theta(z)\Theta(z') = \cfto_{T\bar{T}}(z')
  + \mathrm{derivative\;terms}
\end{equation}
where $\cfto_{T\bar{T}}(z)$ is some local operator. At this point it is
possible to \textit{define} the composite field $T\bar{T}$ through equation
above:
\begin{equation}
  T\bar{T}(z):= \cfto_{T\bar{T}}(z);
\end{equation}
then the desired relation \eqref{eq:generaltt} follows immediately. Note
that although in this way one defines $T\bar{T}$ only modulo derivative
terms, in view of the assumption 2, those terms bring no contribution to the
left hand side of \eqref{eq:generaltt}. However, this definition may look a bit
too formal to bring much insight into the meaning of \eqref{eq:generaltt}. To
understand the nature of the limit $z\rightarrow z'$ in
\eqref{eq:ttexpansion}, and thus to make contact with a more constructive
definition of the composite field $T\bar{T}$, one needs to know more about the 
short-distance behaviour of the field theory. For this case, assumption 4 holds
the relevant information for our construction.
\subsection{Dimensional Analysis}
As was mentioned, we assume the short-distance limit of the field theory is
controlled by certain conformal field theory, which we will refer to as the
CFT. More precisely, we assume that the field theory at hand is the CFT
perturbed by its relevant operators. To avoid unnecessarily complex
expressions, let me first assume that the perturbation is by a single operator
$\Phi_\Delta$ of the dimensions $(\Delta, \Delta)$ with $\Delta < 1$; then the
theory is  described by the action
\begin{equation}
  \mathcal{A} = A_{CFT} + \mu\int\Phi_\Delta(z)d^2z,
  \label{eq:qftaction}
\end{equation}
where $\mu$ is a coupling constant which has the dimension
[length]$^{2\Delta-2}$. This formulation of the theory makes it possible to
carry out dimensional analysis of the structure functions.
\par Let $\{\cfto_i\}$ be a complete set of local fields of the CFT, including
primary fields as well as their descendants, and let $(\Delta_i,
\bar{\Delta_i})$ be the left and right scale dimensions of the fields
$\cfto_i$. This set includes the field $T\bar{T}$ (of the dimensions)
(2,2)), which in CFT is just the descendant $T\bar{T} = L_{-2}\bar{L}_{-2}I$ of
the identity operator. Equivalently, this field can be defined as 
$T\bar{T}(z') = \lim_{z\rightarrow z'} T(z)\bar{T}(z')$, where the
limit is straightforward since in CFT the above operator product has no
singularity at $z=z'$.
\par As was explained in REF [12], the fields $\cfto_I$ of the perturbed theory
are in one-to-one correspondence with the fields of the CFT (hence we use the
same notations). The field $\cfto$ has the spin $s_i = \Delta_i
- \bar{\Delta}_i$ and the mass dimension $d_i = \Delta_i + \bar{\Delta}_i$, and
$\cfto_i$ coincides with the corresponding CFT field in the limit
$\mu\rightarrow 0$. Unless certain resonance conditions are met, these
properties characterize the field $\cfto_i$ uniquely. One says that the field
$\cfto_i$ has $n$-th order resonance with the field $O_j$ if these fields 
have the same spins, $s_j = s_i$, and their dimensions satisfy the equation 
$d_i = d_j + 2n(1-\Delta)$ (the resonance condition) with some positive integer
$n$. When this resonance condition is fulfilled the above characterization of
the field $\cfto_i$ allows for the ambiguity $\cfto_i \rightarrow \cfto_i
+ \mathrm{const}*\mu^n\cfto_j$
\par The field $T\bar{T}$ always has the intrinsic ambiguity of the form
$T\bar{T}\rightarrow T\bar{T}
+ \mathrm{cosnt}*\partial_z\partial_{\bar{z}}\Theta,$ where $\Theta$ is
the trace component of the energy-momentum tensor tensor of the perturbed
theory. Using the equation for the action, it is clear that $\Theta
= (1-\Delta)\pi\mu\Phi_\Delta$, the ambiguity is due to the first-order
resonance of $T\bar{T}$ with the derivative
$\partial_z\partial_{\bar{z}}\Phi_\Delta$. However, this ambiguity has
no effect on the expectation value of $T\bar{T}$. For the present analysis
the danger is in possible resonances with non-derivative fields. Since at 
this time I do not know how to handle the resonance cases, we accept the
following no-resonance assumption:
\begin{itemize}
  \item (4') Dimensions $\Delta_i$ of the fields $\cfto_i$ of the CFT satisfy
    the condition
    \begin{equation}
      \Delta_i - 2 + n(1-\Delta) \neq 0, \quad \mathrm{for}\; n=1,2,3,\dots,
      \label{eq:assumptiondimension}
    \end{equation}
    with the only exception of $\Delta_i = \Delta + 1$ (which  is the dimension
    of $\partial_z\partial_{\bar{z}}\Phi_\Delta$
    )
\end{itemize}
According to [3], the OPE structure functions in
\eqref{eq:operationproductexpansion} admit power-series expansions in $\mu$,
with the coefficients computable (in principle) through the conformal
perturbation theory. Thus, the structure functions $D_i(z-z')$ in
\eqref{eq:opetheta15} can be written as
\begin{equation}
  D_i(z-z') = \sum_{n=0}^{\infty} (z-z')^{\Delta_i -2
  +n(1-\Delta)}(\bar{z}-\bar{z}')^{\bar{\Delta}_i -2
  + n(1-\Delta)}D_i^{(n)}\mu^n.
  \label{eq:expansionstructure}
\end{equation}
The zero-order coefficients $D_i^{(0)}$ are taken from the unperturbed CFT,
hence $D_i^{(0)} =0$ unless $\cfto_i$ is the field $T\bar{T}$ or one of its
derivatives, a nd $D_{T\bar{T}}^{(0)} = 1$. Then it follows from
\eqref{eq:assumptiondimension} that the only terms in the expansions
\eqref{eq:expansionstructure} which carry vanishing powers  of both $z-z'$ and
$\bar{z} - \bar{z}'$ are the zero-order term of $D_{T\bar{T}}$, and the
first-order term associated with $\cfto_i
= \partial_{z'}\partial_{\bar{z}'}\Phi_\Delta$. 
\par Similar expansion can be written down for the structure functions
$C_i(z-z')$ in the OPE \eqref{eq:opetheta15},
\begin{equation}
  C_i(z-z') = \sum_{n-2}^{\infty}(z-z')^{\Delta_i
    - 2 +n(1-\Delta)}(\bar{z}-\bar{z}')^{\bar{\Delta}_i -2
  +n(1-\Delta)}C_i^{(n)}\mu^n.
\end{equation}
Note that the sum here starts from $n=2$, consequence of the fact that
$\Theta\sim\mu\Phi_\Delta$. In this case the no-resonance condition implies
that there are no terms with vanishing powers of both $z-z'$ and
$\bar{z}-\bar{z}'$ at all.
\par Consider now the differences $F_i(z-z') = D_i(z-z') - C_i(z-z')$. It
follows from some of the previous equations that, unless $\cfto_i$ is
a derivative of another local field, all terms with nonzero powers of $z-z'$ or
$\bar{z}-\bar{z}'$ must cancel out in this difference \footnote{This implies
  for  instance $D_{T\bar{T}}^{(1)} = 0$, a statement easily verified in
conformal perturbation theory}. Therefore
\begin{equation}
  F_i(z-z') = 0\quad \mathrm{unless}\quad \cfto_i
  = T\bar{T}\quad\mathrm{or}\quad\cfto_i=\mathrm{derivative},
\end{equation}
and
\begin{equation}
  F_{T\bar{T}}(z-z') = 1.
\end{equation}
One concludes that the definition of $T\bar{T}$ through the conformal
perturbation theory agrees with the formal definition \eqref{eq:definitiontt}.
\par It is not difficult to generalize this analysis to the case when the CFT
is perturbed by a mixture. $\sum_a \mu_a \int \Phi_{\Delta_a}(z)d^2z$ of
relevant operators $\Phi_{\Delta_a}$. The dimensional analysis can be carried
out in a similar straightforward way provided the no-resonance condition is 
modified as follows:
\begin{itemize}
 \item 4''. The dimensions $\Delta_i$ of the fields $\cfto_i$ of the CFT
   satisfy the conditions
   \begin{equation}
     \Delta_i - 2 + \sum_a n_a(1-\Delta_a)\neq 0
   \end{equation}
   for any non-negative integers $n_a$ such that $\sum_a n_a > 0$, with only
   the exeption of $\Delta_i = \Delta_a + 1.$
\end{itemize}
\subsection{Further remarks}
We can consider the 2D space to be a cylinder, with one of the  Cartesian
coordinates compactified on a circle of circumference $R$, $(x,y) \sim
(x+R,y)$, and let $\mathbb{H}$ and $\mathbb{P}$ be the Hamiltonian and the
momentum operators in the picture where the  coordinate $y$ along the cylinder
is  taken as the Eucleadian time. The arguments of the previous sections
validate the relation \eqref{eq:generaltt} with $\langle\dots\rangle$ standing
for the matrix element $\mel{0}{\dots}{0}$, where $\ket{0}$ is the ground state
of the Hamiltonian $\mathbb{H}$ (and it is assumed that the states are
orthogonal, meaning $\braket{0}{0}= 1$). It turns out that the same relation
remains valid if the vacuum expectation values there are replaced by generic
diagonal matrix elements $\mel{n}{\dots}{n}$, where $\ket{n}$ is an arbitrary
non-degenerate eigenstate of the energy and momentum operators,
\begin{equation}
  \mathbb{H}\ket{n} = E_n\ket{n}, \quad \mathbb{P}\ket{n} = P_n\ket{n},
\end{equation}
and again the normalization $\braket{n}{n}=1$ is assumed. Indeed, of the
previously listed assumptions, the local ones (1 and 4) are independent on the 
choice of matrix element, while assumption 2 (global translational invariance)
certainly remains valid when any diagonal matrix element between
energy-momentum eigenstates is taken. Hence, one can repeat the calculation at
the end of the second section (which only uses assumptions 1 and 2) and again
show that the combination
\begin{equation}
  \mathcal{C}(n) = \mel{n}{T(z)\bar{T}(z')}{n}
  - \mel{n}{\Theta(z)\Theta(z')}{n}
  \label{eq:constantcombination}
\end{equation}
is constant, independent of the points $z$ and $z'$. In general, the asymptotic
factorization \eqref{eq:infiniteseparations} (the one in assumption 3), no
longer holds, since the two-point function $\mel{n}{\cfto_i (x,y) \cfto_j
  (x',y')}{n}$ can pick up contributions from the  intermediate states
  $\ket{n'}$ with $E_{n'} < E_n$ which give rise to terms growing exponentially
  with $\abs{y-y'}$. However, one can write down the spectral decompositions of
  the two-point functions in the right-hand side of
  \eqref{eq:constantcombination}, i.e.
\begin{equation}
  \mel{n}{T(z)\bar{T}(z')}{n}
  = \sum_{n'}\mel{n}{T(z)}{n'}\mel{n'}{\bar{T}(z)}{n}e^{(E_n-E_{n'})\abs{y-y'}+i(P_n
    + P_{n'})(x-x')}
\end{equation}
and similar decomposition of $\mel{n}{\Theta(z)\Theta(z')}{n}$, where $(x,y)$
and $(x',y')$ are Cartesian coordinates of the points $z$ and $z'$,
respectively. Clearly, for the combination \eqref{eq:constantcombination} to be
independent of the coordinates, all terms in these decompositions with $n\neq
n'$ must cancel out between the two correlators in the right-hand side of
\eqref{eq:constantcombination}. If $\ket{n}$ is non-degenerate, if follows that
\begin{equation}
  \mathcal{C} = \mel{n}{T(z)}{n}\mel{n}{\bar{T}(z')}{n}
  - \mel{n}{\Theta(z)}{n}\mel{n}{\Theta(z')}{n},
\end{equation}
and by taking the limit $z\rightarrow z'$ one arrives at the desired relation
\begin{equation}
  \mel{n}{T\bar{T}}{n} = \mel{n}{T}{n}\mel{n}{\bar{T}}{n} - \mel{n}{\Theta}{n}\mel{n}{\Theta}{n}
  \label{eq:decompositionformula}
\end{equation}
It's revealing to rewrite this relation in somewhat different form. In terms
of Cartesian components of the energy-momentum tensor $T_{\mu\nu}$ it
reads\footnote{the factor $\pi$ is due to the factor $2\pi$ n the definition of
the  chiral components of the composite operator - $T$ and $\bar{T}$}
\begin{equation}
  \mel{n}{T\bar{T}}{n} = -\pi^2\left(\mel{n}{T_{yy}}{n}\mel{n}{T_{xx}}{n}
  - \mel{n}{T_{xy}}{n}\mel{n}{T_{xy}}{n}\right),
\end{equation}
On the other hand, because we're talking about the energy-momentum tensor, we
have
\begin{equation}
  \mel{n}{T_{yy}}{n} = -\frac{1}{R}E_n(R), \quad \mel{n}{T_{xx}}{n}
    = - \frac{d}{dR}E_n(R),
\end{equation}
and
\begin{equation}
  \mel{n}{T_{xy}}{n} = \frac{i}{R}P_n(R).
\end{equation}
where it was explicitly indicated that the energy-momentum eigenvalues depend
on $R$. Of course, in the case of $P_n$, the $R$-dependence is fixed by the
momentum quantization condition: $P_n(R) = 2\pi p_n /R$, where $p_n$ are
$R$-independent integers. Thus the expectation value
\eqref{eq:decompositionformula} can be expressed in terms of the eigenvalues
$E_n(R), P_n(R)$ as follows
\begin{equation}
  \mel{n}{T\bar{T}}{n} = -\frac{\pi}{R}\left(E_n(R)\frac{d}{dR}E_n(R)
  + \frac{1}{R}P^2_n(R)\right).
\end{equation}
\par Suppose that the field theory \eqref{eq:qftaction} is massive, with $M_0$
being the mass of it's lightest particle. Then for $R>>M_0^{-1}$ the
ground-state energy $E_0(R)$ approaches its asymptotic linear form with
exponential accuracy, i.e.
\begin{equation}
  E_0(R) = F_0 R + O(e^{-M_0R}),
\end{equation}
where $F_0$ is the vacuum energy density in infinite space. In the same limit,
the first excited state $\ket{1}$ corresponds to the one-particle state with
zero momentum, hence
\begin{equation}
  E_1(R) = F_0 R + M_0 + O(e^{-M_0R})
\end{equation}
Then it follows from \eqref{eq:decompositionformula} that (up to terms $\sim
e^-M_0R$)
\begin{equation}
  \frac{1}{\pi^2}\mel{0}{T\bar{T}}{0} = - F^2_0,\quad
    \frac{1}{\pi^2}\mel{1}{T\bar{T}}{1} = -F^2_0 - \frac{1}{R}F_0M_0.
\end{equation}
These expressions can be useful in analysis of subleading singularities in
statistical systems near criticality, in the situations where the irrelevant
operator $T\bar{T}$ plays a significant role. This is the case, for instance, for
the Ising phase transition near the Ising tri-critical point, because the RG
flow from the tri-critical fixed point (the $c=7/10$ minimal CFT) down to the
Ising fixed point (the $c=1/2$ minimal CFT) arrives at the  latter along
a direction the field $T\bar{T}$ as its most significant (i.e. least
irrelevant) component. [13]. Another example is the Ising field theory with
pure imaginary magnetic field, taken near the Yang-Lee singularity. In such
cases the final relations lead to predictions abouth the amplitudes of
subleading singular terms in the expansions of the free energy and correlation
length near the critical point.
\section{$T\bar{T}$ as a TsT transformation}
\subsection{$T\bar{T}$ deformations and uniform light-cone gauge}
\subsubsection{Uniform light-cone gauge}
We consider a non-linear sigma model with metric $G_{\mu\nu}(X)$, where $X$
collectively denotes all the fields, and B-field $B_{\mu\nu}(X)$. The metric
part of the action is coupled to a two-dimensional metric
$\gamma^{\alpha\beta}$, which we take to have unit determinant. By
construction, the theory is invariant under re-parametrizations at the classic
level. For the moment, we will be interested in the classical theory, and we
will not assume that the metric and B-field describe a string background. We
will however assume that the metric has at least two shift isometries: one for
a time-like coordinate which we denote by $t, t\rightarrow t+\delta t$, and
wich yields the  target-space energy $E$, and one for $\phi\rightarrow \phi
+ \delta\phi$, which yields some (angular) momentum $J$. In terms of the
action, we have
\begin{equation}
  S = -\frac{1}{2}\int_{-\infty}^{\infty}\dd\tau\int_{0}^{R}\dd\sigma\left(\gamma^{\alpha\beta}\partial_\alpha
    X^\mu\partial_\beta X^\nu G_{\mu\nu}(X) + \epsilon^{\alpha\beta}\partial_\alpha
  X^\mu\partial_\beta X^\nu B_{\mu\nu}(X)\right).
\end{equation}
The minus sign takes into account that the world sheet metric has signature
$(-,+)$. It is convinient to introduce the momenta $p_\mu$, which are
canonically conjugated to $X^\mu$:
\begin{equation}
  p_\mu = \frac{\delta S}{\delta \partial_\tau X^\mu}
  = -\gamma^{0\beta}\partial_\beta X^\nu G_{\mu\nu}(X) - \dot{X}^\n
  B_{\mu\nu}(X),
\end{equation}
where we have introduced the notation $\dot{X}^\nu \equiv \partial_\sigma
X^\mu$. By Noether's theorem we immediately get two conserved charges:
\begin{equation}
  E = - \int_{0}^{R}\dd\sigma p_t, \quad \mathrm{and}\quad
  J = \int_{0}^{R}\dd\sigma p_\phi .
\end{equation}
An advantage of the first-order formalism is thath the action takes the form
\begin{equation}
  S = \int_{-\infty}^{\infty}\dd\tau\int_{0}^{R}\left(p_\mu\dot{X}^\mu
    + \frac{\gamma^{01}}{\gamma^{00}}\mathcal{C}_1
  + \frac{1}{2\gamma^{00}}\mathcal{C}_2\right),
\end{equation}
where the worldsheet metric takes the  form of a Lagrange multiplier and yields
the two Virasoro constraints:
\begin{align}
  0 &= \mathcal{C}_1 = p_\mu \dot{X}^\mu\nonumber\\
  0 &= \mathcal{C}_2 = p_\mu p_\nu G^{\mu\nu} + \dot{X}^\mu\dot{X}^\nu
  G_{\mu\nu} + 2 G^{\mu\nu} B_{\nu\rho} p_\mu \dot{X}^\rho
  + G^{\mu\nu}B_{\mu\rho}B_{\nu\lambda}\dot{X}^\rho\dot{X}^\lambda,
\end{align}
where we  suppressed the dependence of the (inverse) metric and B-field on
$X^{\mu}$.




