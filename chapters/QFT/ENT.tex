%\chapterauthor{Second Author}{Second Author Affiliation}
\chapter{Entaglement Entropy}
\chapterauthor{Ivo Iliev\\Sofia University}
\adjustmtc
\minitoc
\section{Introduction to Entaglement entropy}
Entaglement entropy is an important way to quantify quantum correlations in
a system. For gauge theories, the definition of entaglement entropy 
turns out to be subtle. Physicall, this is because there are non-local
excitations in the system - for example, loops of electric or magnetic
flux created by Wilson or 't Hooft loop operators, which can cut across
the boundary of the region of interest. More precisely, it is because
the Hilbert space of gauge invariant states, $\hilspace_{g\mathrm{inv}}$, does
not admit a tensor product decomposition between the region of interest
and its complement.
\par A definition of the entaglement entropy has been given for a general gauge
theory based on an extended Hilbert space (EHS) construction. For a non-Abelian
theory without matter, on a spatial latice, this takes the rather compact form
\begin{equation}
  S_{\mathrm{EE}} = -\sum_i{p_i\log p_i}+\sum_i{\log(d^i_a)
    -\sum_i{p_i\mathrm{Tr}_{\hilspace_i}\bar{\rho}_i\log(\bar{\rho}_i)}.
    }
    \label{eq:entanglemententropy}
\end{equation}
The index $i$ in the summation denotes various sectors which are determined
by the value of the normal component of the electric field at all the
boundary points. The probability to be in the sector $i$ is given by $p_i$
,and $d_a^i$, with index $a$ denoting the particular boundary point, is the
dimension of the representation specifying the outgoing electric flux at that
boundary point in the $i^{\mathrm{th}}$ sector. The last term is a weighted sum
with $\bar{\rho}_i$ being the reduced density matrix in the $i^{\mathrm{th}}$
sector and the trace being taken over the Hilbert space $\hilspace_i$ in this
sector. It has been argued that this EHS definition agrees with the
replica trick method of calculating the entaglement entropy. In the discussion
below, we will sometimes refer to the first two terms in
\eqref{eq:entanglemententropy} as the classical terms and the last term as the
quantum term.
\par It should be obvious that the middle term in
\eqref{eq:entanglemententropy}, the one depending on $d_a^i$ is absent in the
Abelian theory. One can argue that the lack of a tensor product decomposition
of the Hilbert space is related to the presence of a non-trivial centre. While
the Hilbert of gauge-invariant states does not admit a  tensor product
decomposition, it could be written as a sum of tensor product terms,
\begin{equation}
  \hilspace_{g\mathrm{inv}} = \bigoplus_i\hilspace_i\otimes\hilspace_i',
\end{equation}
where each factor $\hilspace_i\otimes\hilspace_i'$ is the Hilbert space in 
a sector corresponding to a particular value for the centre. Different choices
for the algebra of gauge-invariant operators lead to different choices of
centres and to a different value of the entaglement entropy in general. In
particular, keeping all gauge-invariant operators in the region of interest in
the algebra gives rise to the electric centre which is specified by the normal
component of the electric field along the boundary. It is this definition
which agrees with the EHS definition for the Abelian case. Other choices of
centres can also be made. In particular, removing all the tangental components
of the electric field at the boundary from the algebra gives rise to a magnetic
centre.
\par Here, we are interested in studying the behaviour of the first two terms 
of \eqref{eq:entanglemententropy}, which correspond to the non-extractable
contributions to the entanglement entropy, in the continuum limit. These terms
depend on the distribution $p_i$ which determines the probability for being in
the various superselection sectors. For the electric centre case\footnote{In
  the non-Abelian case, the electric centre definition is sometimes taken to be
  different from the EHS definition with the middle term in
  \eqref{eq:entanglemententropy} being absent. We will not be very careful
  about such distinctions. Our considerations, studying the probability
distribution $p_i$ will apply to both of these cases.}, by studying the
correlation functions of the electric field on the boundary, we will find, both
for the Abelian and non-Abelian cases, that the distribution is typically 
determined by very high momentum modes localized close to the boundary. As
a result, we will argue that the contribution form these terms drops out of the
mutual information of disjoint regions or the relative entropy between two
states which only have finite energy excitations about the vacuum.
\section{Classical Term in the Continuum Limit}
\subsection{$U(1)$ Abelian Theory}
Let us begin by considering a  3+1-dimensional free $U(1)$ theory. Since the
theory is free, the classical term is determined by the two-point function of
the normal component of the electric field,
\begin{equation}
  G_{rr} = \langle E_rE_r\rangle,
\end{equation}
with the two electric fields $E_r$ being inserted at two different points on
the boundary which is an $S^2$ of radius $R$. The classical term is then given
by
\begin{equation}
  -\sum p_i\log(p_i) = \alpha_1\frac{A}{\epsilon^2}
  - \frac{1}{2}\log(\mathrm{det}G^{-1}_{rr}),
  \label{eq:classicalu1}
\end{equation}
with $\epsilon$ being a short distance cut-off. Thus, up to a non-universal
area lawa divergence, $G_{rr}$ determines the classical term. Let us note in
passing that the first term on the RHS of \eqref{eq:classicalu1} depends on the
measure for the sum over over all electric field configurations. Starting from
the lattice and passing to the continuum gives a well defined measure.
\par Since the two electric fields in $G_{rr}$ are inserted at points on the
boundary $S^2$ it is a function of the angular separation of the two points.
In fact, it turns out to be divergent and some care needs to be exercised in 
regulating this divergence and defining it. After introducing a short distance
cut-off $\Delta$ for the radial momentum, the resulting answer was shown to be
\begin{equation}
  G_{rr}^{lm} = \frac{1}{\pi R^4}\left(\log\frac{R^2}{\Delta^2}\right)l(l+1).
\end{equation}
Here we have carried out a Fourier transform to go from the angular separation
variables to the angular momentum variables labelled by the integers $(l,m)$,
as per standard conventions. Note that the result diverges as
$\Delta\rightarrow 0$, due to the contribution of modes with very high values
of the radial momentum, which dominate the result in this limit. The factor of
$l(l+1)$ means that the Green function is proportional to the two-dimenstional
Laplacian for a free scalar on the boundary. 
\par From $G_{rr}$, we can calculate the classical piece \eqref{eq:classicalu1}
which turns out to be,
\begin{equation}
  -\sum_i p_i\log(p_i) = \alpha_2\frac{A}{\epsilon^2}
  - \frac{1}{6}\log\left(\frac{A}{\epsilon^2}\right) + \cdots,
  \label{classicalu1full}
\end{equation}
where the ellipsis refer to non-universal finite pieces. The logarithmic term
arises from the two-dimenstional scalar Laplacian mentioned above, while
the pre-factor $\log(R^2/\Delta^2)/\pi R^4$ contributes to the non-universal
area law divergent pieces above.
\par It is important to emphasise that we have introduced two cut-offs above,
$\Delta$ and $\epsilon$, and worked in the limit where $\Delta$ goes to zero
first. This was true in \eqref{eq:classicalu1full}, for example, since the
angular momentum $l$ along the $S^2$ was being kept fixed while
$\Delta\rightarrow 0$. It is in this limit that $G_{rr}$ depends on the
two-dimenstional Laplacian for a free scalar with a logarithmic dependence on
the radial cut-off $\Delta$. The behaviour of modes whose momentum along $S^2$
is comparable to the radial cutoff is more complicated and less universal. In
the discussion below, when we analyse the contribution of the classical term to
the mutual information and relative entropy, we will comment on such modes as
well and show that their contribution too drops out from these quantities.
\par Having understood the case of a spherical region, we can now turn to the 
more general situation. Let us begin by first considering the half space $z<0$,
with the remaining spatial coordinates, $(x,y)$ taking values in
$(-\infty,\infty)$. The boundary of this region is the two-plane $z=0$. The
normal component of the electric field is along the $z$ direction. It is easy
to see by standard quantisation of the electromagnetic field that 
\begin{equation}
  \langle E_z(\mathbf{x_1})E_z(\mathbf{x_2})\rangle = \int
  \frac{d^3k}{(2\pi)^2}\frac{\omega_k}{2}\left(1-\frac{k_z^2}{k^2}\right)e^{i\mathbf{k}\cdot(\mathbf{x}_1
    - \mathbf{x}_2)}
\end{equation}
(The two-point function is computed at equal time.) Here $\omega_k
= k = \sqrt{k^2_z + k^2_x + k^2_y}$. Setting the two points to be at the
boundary, $z=0$, we get
\begin{equation}
  \langle E_z(\mathbf{k}_{\|})\E_z(\mathbf{k}_{\|}')\rangle
  = (2\pi)^2\delta^{(2)}(\mathbf{k}_{\|}
  + \mathbf{k}_{\|}')k^2_{\|}\int{\frac{dk_z}{4\pi}\frac{1}{k}}.
\end{equation}
We see that the integral on the RHS is logarithmically divergent since $k\sim
k_z$ for large $k_z$. Thus, as noted above, the result needs to be regulated by
introducing a cut-off for momentum along the $z$ direction, normal to the
boundary. One efficient way to do this is to take the two points located at
slightly different values in the $z$ direction, with $\Delta z = \delta$,
instead of both of them being exactly at the boundary. Repeating the 
calculation above now gives,
\begin{equation}
  \langle E_z(\mathbf{k}_{\|})\E_z(\mathbf{k}_{\|}')\rangle
  = (2\pi)^2\delta^{(2)}(\mathbf{k}_{\|}
  + \mathbf{k}_{\|}')k^2_{\|}\int{\frac{dk_z}{4\pi}\frac{e^{ik_z\Delta}}{k}}.
\end{equation}
We see that the divergence in the integral at large $k_z$ is regulated
resulting in
\begin{equation}
  \langle E_z(\mathbf{k}_{\|})\E_z(\mathbf{k}_{\|}')\rangle
  \sim (2\pi)^2\delta^{(2)}(\mathbf{k}_{\|}
  + \mathbf{k}_{\|}')k^2_{\|}\log(k_{\|}\Delta).
\end{equation}
We can see that the result above is analogous to what was obtained for the
spherical region case. In particular, the result diverges e logarithmically as 
$\Delta\rightarrow 0$ and is proportional to $k^2_{\|}$ which is the eigenvalue
of the two-dimenstional scalar Laplacian on the boundary. One difference is
that in the spherical case, the logarithmic divergence due to the modes
with high value of the normal component is cut-off by the radius $R$, whereas
in the infinite plane case, where this cut-off is not availiable, it is
cut-off by $k_{\|}$.
\par From this example of the half plane and the spherical region, it is now
clear that we expect the two-point function in any compact region a result
analogous to the sphere case, namely going like $\log(R/\Delta)$, with $\Delta$
being the cut-off fir the normal component of momentum, and $R$ being an IR
scale provided by the size of the region of interest, and also being
proportional to the two-dimenstional Laplacian along the boundary.






\section{Entanglement and Dualities}
\section{Conclusions}

