%\chapterauthor{Second Author}{Second Author Affiliation}
\chapter{Entaglement Entropy}
\chapterauthor{Ivo Iliev\\Sofia University}
\adjustmtc
\minitoc
\section{Introduction to Entaglement entropy}
Entaglement entropy is an important way to quantify quantum correlations in
a system. For gauge theories, the definition of entaglement entropy 
turns out to be subtle. Physicall, this is because there are non-local
excitations in the system - for example, loops of electric or magnetic
flux created by Wilson or 't Hooft loop operators, which can cut across
the boundary of the region of interest. More precisely, it is because
the Hilbert space of gauge invariant states, $\hilspace_{g\mathrm{inv}}$, does
not admit a tensor product decomposition between the region of interest
and its complement.
\par A definition of the entaglement entropy has been given for a general gauge
theory based on an extended Hilbert space (EHS) construction. For a non-Abelian
theory without matter, on a spatial latice, this takes the rather compact form
\begin{equation}
  S_{\mathrm{EE}} = -\sum_i{p_i\log p_i}+\sum_i{\log(d^i_a)
    -\sum_i{p_i\mathrm{Tr}_{\hilspace_i}\bar{\rho}_i\log(\bar{\rho}_i)}.
    }
    \label{eq:entanglemententropy}
\end{equation}
The index $i$ in the summation denotes various sectors which are determined
by the value of the normal component of the electric field at all the
boundary points. The probability to be in the sector $i$ is given by $p_i$
,and $d_a^i$, with index $a$ denoting the particular boundary point, is the
dimension of the representation specifying the outgoing electric flux at that
boundary point in the $i^{\mathrm{th}}$ sector. The last term is a weighted sum
with $\bar{\rho}_i$ being the reduced density matrix in the $i^{\mathrm{th}}$
sector and the trace being taken over the Hilbert space $\hilspace_i$ in this
sector. It has been argued that this EHS definition agrees with the
replica trick method of calculating the entaglement entropy. In the discussion
below, we will sometimes refer to the first two terms in
\eqref{eq:entanglemententropy} as the classical terms and the last term as the
quantum term.
\par It should be obvious that the middle term in
\eqref{eq:entanglemententropy}, the one depending on $d_a^i$ is absent in the
Abelian theory. One can argue that the lack of a tensor product decomposition
of the Hilbert space is related to the presence of a non-trivial centre. While
the Hilbert of gauge-invariant states does not admit a  tensor product
decomposition, it could be written as a sum of tensor product terms,
\begin{equation}
  \hilspace_{g\mathrm{inv}} = \bigoplus_i\hilspace_i\otimes\hilspace_i',
  \label{eq:hilbertspacetensorproductterms}
\end{equation}
where each factor $\hilspace_i\otimes\hilspace_i'$ is the Hilbert space in 
a sector corresponding to a particular value for the centre. Different choices
for the algebra of gauge-invariant operators lead to different choices of
centres and to a different value of the entaglement entropy in general. In
particular, keeping all gauge-invariant operators in the region of interest in
the algebra gives rise to the electric centre which is specified by the normal
component of the electric field along the boundary. It is this definition
which agrees with the EHS definition for the Abelian case. Other choices of
centres can also be made. In particular, removing all the tangental components
of the electric field at the boundary from the algebra gives rise to a magnetic
centre.
\par Here, we are interested in studying the behaviour of the first two terms 
of \eqref{eq:entanglemententropy}, which correspond to the non-extractable
contributions to the entanglement entropy, in the continuum limit. These terms
depend on the distribution $p_i$ which determines the probability for being in
the various superselection sectors. For the electric centre case\footnote{In
  the non-Abelian case, the electric centre definition is sometimes taken to be
  different from the EHS definition with the middle term in
  \eqref{eq:entanglemententropy} being absent. We will not be very careful
  about such distinctions. Our considerations, studying the probability
distribution $p_i$ will apply to both of these cases.}, by studying the
correlation functions of the electric field on the boundary, we will find, both
for the Abelian and non-Abelian cases, that the distribution is typically 
determined by very high momentum modes localized close to the boundary. As
a result, we will argue that the contribution form these terms drops out of the
mutual information of disjoint regions or the relative entropy between two
states which only have finite energy excitations about the vacuum.
\section{Classical Term in the Continuum Limit}
\subsection{$U(1)$ Abelian Theory}
Let us begin by considering a  3+1-dimensional free $U(1)$ theory. Since the
theory is free, the classical term is determined by the two-point function of
the normal component of the electric field,
\begin{equation}
  G_{rr} = \langle E_rE_r\rangle,
\end{equation}
with the two electric fields $E_r$ being inserted at two different points on
the boundary which is an $S^2$ of radius $R$. The classical term is then given
by
\begin{equation}
  -\sum p_i\log(p_i) = \alpha_1\frac{A}{\epsilon^2}
  - \frac{1}{2}\log(\mathrm{det}G^{-1}_{rr}),
  \label{eq:classicalu1}
\end{equation}
with $\epsilon$ being a short distance cut-off. Thus, up to a non-universal
area lawa divergence, $G_{rr}$ determines the classical term. Let us note in
passing that the first term on the RHS of \eqref{eq:classicalu1} depends on the
measure for the sum over over all electric field configurations. Starting from
the lattice and passing to the continuum gives a well defined measure.
\par Since the two electric fields in $G_{rr}$ are inserted at points on the
boundary $S^2$ it is a function of the angular separation of the two points.
In fact, it turns out to be divergent and some care needs to be exercised in 
regulating this divergence and defining it. After introducing a short distance
cut-off $\Delta$ for the radial momentum, the resulting answer was shown to be
\begin{equation}
  G_{rr}^{lm} = \frac{1}{\pi R^4}\left(\log\frac{R^2}{\Delta^2}\right)l(l+1).
\end{equation}
Here we have carried out a Fourier transform to go from the angular separation
variables to the angular momentum variables labelled by the integers $(l,m)$,
as per standard conventions. Note that the result diverges as
$\Delta\rightarrow 0$, due to the contribution of modes with very high values
of the radial momentum, which dominate the result in this limit. The factor of
$l(l+1)$ means that the Green function is proportional to the two-dimenstional
Laplacian for a free scalar on the boundary. 
\par From $G_{rr}$, we can calculate the classical piece \eqref{eq:classicalu1}
which turns out to be,
\begin{equation}
  -\sum_i p_i\log(p_i) = \alpha_2\frac{A}{\epsilon^2}
  - \frac{1}{6}\log\left(\frac{A}{\epsilon^2}\right) + \cdots,
  \label{eq:classicalu1full}
\end{equation}
where the ellipsis refer to non-universal finite pieces. The logarithmic term
arises from the two-dimenstional scalar Laplacian mentioned above, while
the pre-factor $\log(R^2/\Delta^2)/\pi R^4$ contributes to the non-universal
area law divergent pieces above.
\par It is important to emphasise that we have introduced two cut-offs above,
$\Delta$ and $\epsilon$, and worked in the limit where $\Delta$ goes to zero
first. This was true in \eqref{eq:classicalu1full}, for example, since the
angular momentum $l$ along the $S^2$ was being kept fixed while
$\Delta\rightarrow 0$. It is in this limit that $G_{rr}$ depends on the
two-dimenstional Laplacian for a free scalar with a logarithmic dependence on
the radial cut-off $\Delta$. The behaviour of modes whose momentum along $S^2$
is comparable to the radial cutoff is more complicated and less universal. In
the discussion below, when we analyse the contribution of the classical term to
the mutual information and relative entropy, we will comment on such modes as
well and show that their contribution too drops out from these quantities.
\par Having understood the case of a spherical region, we can now turn to the 
more general situation. Let us begin by first considering the half space $z<0$,
with the remaining spatial coordinates, $(x,y)$ taking values in
$(-\infty,\infty)$. The boundary of this region is the two-plane $z=0$. The
normal component of the electric field is along the $z$ direction. It is easy
to see by standard quantisation of the electromagnetic field that 
\begin{equation}
  \langle E_z(\mathbf{x_1})E_z(\mathbf{x_2})\rangle = \int
  \frac{d^3k}{(2\pi)^2}\frac{\omega_k}{2}\left(1-\frac{k_z^2}{k^2}\right)e^{i\mathbf{k}\cdot(\mathbf{x}_1
    - \mathbf{x}_2)}
    \label{eq:generaltwopoint}
\end{equation}
(The two-point function is computed at equal time.) Here $\omega_k
= k = \sqrt{k^2_z + k^2_x + k^2_y}$. Setting the two points to be at the
boundary, $z=0$, we get
\begin{equation}
  \langle E_z(\mathbf{k}_{\|})E_z(\mathbf{k}_{\|}')\rangle
  = (2\pi)^2\delta^{(2)}(\mathbf{k}_{\|}
  + \mathbf{k}_{\|}')k^2_{\|}\int{\frac{dk_z}{4\pi}\frac{1}{k}}.
\end{equation}
We see that the integral on the RHS is logarithmically divergent since $k\sim
k_z$ for large $k_z$. Thus, as noted above, the result needs to be regulated by
introducing a cut-off for momentum along the $z$ direction, normal to the
boundary. One efficient way to do this is to take the two points located at
slightlA different values in the $z$ direction, with $\Delta z = \delta$,
instead of both of them being exactly at the boundary. Repeating the 
calculation above now gives,
\begin{equation}
  \langle E_z(\mathbf{k}_{\|})E_z(\mathbf{k}_{\|}')\rangle
  = (2\pi)^2\delta^{(2)}(\mathbf{k}_{\|}
  + \mathbf{k}_{\|}')k^2_{\|}\int{\frac{dk_z}{4\pi}\frac{e^{ik_z\Delta}}{k}}.
  \label{eq:twopointelectric}
\end{equation}
We see that the divergence in the integral at large $k_z$ is regulated
resulting in
\begin{equation}
  \langle E_z(\mathbf{k}_{\|})E_z(\mathbf{k}_{\|}')\rangle
  \sim (2\pi)^2\delta^{(2)}(\mathbf{k}_{\|}
  + \mathbf{k}_{\|}')k^2_{\|}\log(k_{\|}\Delta).
  \label{eq:twopointdivergence}
\end{equation}
We can see that the result above is analogous to what was obtained for the
spherical region case. In particular, the result diverges e logarithmically as 
$\Delta\rightarrow 0$ and is proportional to $k^2_{\|}$ which is the eigenvalue
of the two-dimenstional scalar Laplacian on the boundary. One difference is
that in the spherical case, the logarithmic divergence due to the modes
with high value of the normal component is cut-off by the radius $R$, whereas
in the infinite plane case, where this cut-off is not availiable, it is
cut-off by $k_{\|}$.
\par From this example of the half plane and the spherical region, it is now
clear that we expect the two-point function in any compact region a result
analogous to the sphere case, namely going like $\log(R/\Delta)$, with $\Delta$
being the cut-off fir the normal component of momentum, and $R$ being an IR
scale provided by the size of the region of interest, and also being
proportional to the two-dimenstional Laplacian along the boundary.
\par We are now ready to consider the mutual information and relative entropy
in this theory. Consider the mutual information in the vacuum state first.
Suppose there are two compact disjoint spatial regions $A,B$. The mutual
information is given by 
\begin{equation}
  I(A,B) = S_A + S_B - S_{AB}
\end{equation}
where $S_{AB}$ is the entaglement between the region $A\cup B$ abd the rest. We
are interested in the classical term's contribution to $I(A,B)$. Since the
$U(1)$ theory under consideration is free, this is determined by the two-point 
function, as discussed above, with $p(E_n)$ the probability to be in the
sector where the normal component of the electric field takes value $E_n$ being
given by
\begin{equation}
  p(E_n(\mathbf{x})) = N\; \mathrm{exp}\left(-\frac{1}{2}\int d^2x d^2
  y E_n(\mathbf{x})G^{-1}_{nn}(\mathbf{x}-\mathbf{y})E_n(\mathbf{y})\right).
  \label{eq:twopointvacuum}
\end{equation}
Here $N$ is a normalization, $E_n$ denotes the normal component of the electric
field, $\mathbf{x}\mathbf{y}$ are two points on the boundary and $G_{nn}$ is
the Green function for the normal component $E_n$.
\par When we are dealing with the two disjoint regions $A$ and $B$, we have two
boundaries which are also disjoint. Thus the two point function which appears
on the RHS is evaluated when the two points $\mathbf{x},\mathbf{y}$ are both on
the boundary of $A$ or on the boundary of $B$, or when one point is on the
boundary of $A$ and the other on the boundary of $B$. We saw on general grounds
in the previous subsection that when the two points are on the same boundary
the two-point function diverges and, after a cut-off is introduced, is
proportional to $\log(R/\Delta)$, where $R$ is an IR scale set by the size
of the region $A$ or $B$. There is no such divergence when the two points
are located on the two different boundaries of $A$ and $B$ respectively, since
in this case the two points cannot come close to each other. Hence, the
two-point function which appears in the exponent in the equation above will
be dominated by the contribution when the two points are on the same boundary;
with the contribution from when they are on separate boundaries being suppresed
parametrically by a factor of $[\log(R/\Delta)]^{-1}$. As a result, the
probability, $p(E_n^A,E_n^B)$, for the $E_n$ to take the value $E_n^A, E_n^B$
on the two boundaries $\partial A, \partial B$ will simply be the product
\begin{equation}
  p(E_n^A, E_n^B) = p(E_n^A)p(E_n^B),
  \label{eq:mutinfo}
\end{equation}
where $p(E_n^A)$, for example, is the probability that $E_n$ takes value
$E_n^A$ on $\partial A$ regardless of any value $E_n$ takes on $\partial B$.
\eqref{eq:mutinfo} is true up to subleading terms which vanish in the continuum
limit when $\Delta/R\rightarrow 0$. It then follows that the classical term
will cancel out and not contribute to the mutual information in the continuum
limit.
\par Similarly, we can consider the relative entropy between two states. Here
we will be interested in states  which only carry a finite energy above the
ground state. Let the corresponding density matrices for some spatial
region be $\rho^1$ and $\rho^2$ in these two cases; the relative entropy is
given by 
\begin{equation}
  S(\rho^1\mid\rho^2) = \mathrm{Tr}\left(\rho^1\log\rho^1\right)
  - \mathrm{Tr}\left(\rho^1\log\rho^2\right)
\end{equation}
It is easy to see that this becomes
\begin{equation}
  S(\rho^1\mid \rho^2) = \sum_i p^1_i\log(\frac{p_i^1}{p^2_i}) + \sum_i
  p_i^1\mathrm{Tr}_{\hilspace_i}\log(\frac{\bar{\rho}^1_i}{\bar{\rho}_i^2}),
\end{equation}
where the bared rhos are the normalised density matrices in the
$i^{\mathrm{th}}$ sector in state 1 or 2, respectively. We can see, by
an argument analogous to the one above for mutual information that the first
term, which is due to the classical contribution to the entanglement, again
vanishes in the continuum limit. The argument goes as follows. The probability
$p_i^{1,2}$ is determined by the correlation functions of $E_n$ in state $1,2$.
The two-point function in the two states to the leading order will be the same,
and in turn, equal to that in the vacuum \eqref{eq:twopointvacuum}, since it's
dominated by very high momentum modes whose behaviour will be the same as in
the vacuum for states that which only carry a finity energy above the vacuum.
For a region of size $R$, this two-point function goes like $\log(R/\Delta)$
and therefore diverges when $\Delta\rightarrow 0$. Connected higher point
correlations can arise is states which are not the vacuum, but these will
be finite and thus subdominant compared to the two-point function. Therefore,
$p^1_i, p^2_i$ will be the same up to corrections which vanish as
$\left[\log(R/\Delta)\right]^{-1}$, and as a result,
\begin{equation}
  \log\left(\frac{p_i^1}{p_i^2}\right)\sim \frac{1}{\log(\frac{R}{\Delta})}.
\end{equation}
Since $p_i^1$ is normalised so that $\sum_i p_i^1 = 1$, we see that the first
term will be of order $\left[\log(R/\Delta)\right]^{-1}$ and thus will vanish.
\par Before we proceed, some comments are worth to be made. We mentioned that
in general, the two-point function would depend on the boundary Laplacian. For
a spherical region, the zero mode of the Laplacian needs to be excluded when
computing the determinant, due to the Gauss law constraint. More generally, all
zero modes need to be handled with care, but such rigor goes beyond the scope
of this chapter.
\par The divergent behaviour of the two-point function going like
$\log(R/\Delta)$ is true for modes which carry momentum along the boundary that
is much smaller than $\Delta^{-1}$, (i.e., $k_{\|}\ll\Delta^{-1}$). We can also
consider modes which have $k_{\|}\sim\Delta^{-1}$. In this case, the two-point
function is again dominated by the contribution where the two points lie on the
same boundary. Consider, without loss of generality, two regions corresponding
to $z<0$ and $z>L$. When the two points are on the two boundaries, $z=0, z=L$,
the two-point function becomes
\begin{equation}
  \langle E_z(\mathbf{k}_{\|})E_z(\mathbf{k}_{\|}')\rangle
  = (2\pi)^2\delta^{(2)}(\mathbf{k}_{\|}+\mathbf{k}_{\|}')k^2_{\|}\int
  \frac{dk_z}{4\pi}\frac{e^{ik_zL}}{\sqrt{k^2_{\|}+k^2_z}}.  
\end{equation}
Integrating, we get
\begin{equation}
  \langle E_z(\mathbf{k}_{\|})E_z(\mathbf{k}_{\|}')\rangle
  = (2\pi)^2\delta^{(2)}(\mathbf{k}_{\|}+\mathbf{k}_{\|}')k^2_{\|}K_0(k_{\|}L),
\end{equation}
where $K_0$ is called the modifed Bessel function of the second kind. For
$k_{\|}\sim\Delta^{-1}\gg L^{-1}$, we get
\begin{equation}
  \langle E_z(\mathbf{k}_{\|})E_z(\mathbf{k}_{\|}')\rangle
  = (2\pi)^2\delta^{(2)}(\mathbf{k}_{\|}+\mathbf{k}_{\|}')\left(\frac{1}{\Delta^3
  L}\right)^{1/2}e^{-L/\Delta},
\end{equation}
so that the two-point function (where the two points lie on the two boundaries
for modes with $k_{\|}\sim\Delta^{-1}$) is exponentially suppressed, compared
to the case when the two points lie on the same boundary. Thus, the
contribution of these modes will also drop out in the classical part of the
mutual information. Similarly, if we are considering two states whose behaviour
at the cut-off scale $\Delta$ is the same as that of the vacuum, then the 
contribution of modes with $k_{\|}\sim\Delta^{-1}$ will also drop out
in the relative entropy of these two states.
\par Let us also briefly consider the case of the magnetic centre. In this
case, the different superselection sectors are specified by the normal
component of the magnetic field, \textbf{B},  and the probability of being
in a superselection sector is specified by the two point function of the normal
component of \textbf{B}. For the planar boundary considered above at $z=0$, it
is easy to see that the two-point function is given by,
\begin{equation}
  \langle B_z(\mathbf{k}_{\|})B_z(\mathbf{k}_{\|}')\rangle
  = (2\pi)^2\delta^{(2)}(\mathbf{k}_{\|}+\mathbf{k}_{\|}')\int\frac{dk_z}{4\pi}\frac{e^{ik_z\Delta}}{k}.
\end{equation}
This is obviously the same as the two-point function for the electric field
\eqref{eq:twopointelectric}.
It is also straightforward to generalize this discussion for a gauge field to 
other dimensions. In $d+1$ dimensions, \eqref{eq:generaltwopoint} is replaced
by
\begin{equation}
  \langle E_z(\mathbf{x_1})E_z(\mathbf{x_2})\rangle = \int
  \frac{d^dk}{(2\pi)^d}\frac{\omega_k}{2}\left(1-\frac{k^2_z}{k^2}\right)e^{i\mathbf{k}\cdot(\mathbf{x_1}-\mathbf{x_2})},
\end{equation}
where we are still considering the region $z<0$ with a boundary which now has
extent in $d-1$ spatial dimensions.
\par It then follows that
\begin{align}
  \langle E_z(\mathbf{k}_{\|})E_z(\mathbf{k}_{\|}')\rangle
  &=(2\pi)^{d-1}\delta^{d-1}(\mathbf{k}_{\|}+\mathbf{k}_{\|}')k^2_{\|}\int
  \frac{dk_z}{4\pi}\frac{e^{ik_z\Delta}}{k}\nonumber\\
  &\sim (2\pi)^{d-1}\delta^{d-1}(\mathbf{k}_{\|}+\mathbf{k}_{\|}')k^2_{\|}\log(k_{\|}\Delta),
\end{align}
showing that the logarithmic divergence, and boundary Laplacian are universal
features in all dimensions $d\geq 2$.\footnote{In $d=2$, despite the fact that
  the compact theory is confined on large length-scales, it is not on small
length-scales, and thus we expect this result to continue to hold as well.}The
logarithmic divergence then implies that our arguments go through for the
$U(1)$ theory in any dimension greater than two. It also follows that the
mutual information and relative entropy are independent of the classical
term.
Our next consideration is the logical extension of the free $U(1)$ theory - we
want to study theories with charged matter. On the lattice, charged matter adds
degrees of freedom that live on the spatial lattice sites in the standard
manner. The discussion above can be readily extended to such a case. The Gauss
law constraint still results in a non-trivial centre, and the full Hilbert
space admits a decomposition of the form given in
\eqref{eq:hilbertspacetensorproductterms}. The label $i$ denotes sectors where
the centre takes a fixed value. Also, the extended Hilbert space in this case
is obtained by taking the tensor product of the Hilbert spaces of the gauge
degrees of freedom (living on the links) and the matter degrees of freedom
(living on the sites). Passing to the continuum in the electric centre case, or
equivalently, the extended Hilbert space case, the different sectors are
specified by the different values taken by the normal component of the electric
field on the boundary of the region of interest. The probability $p_i$ is a 
functional of these boundary values, $p_i\equiv p_i\left[E_n\right]$. The
theory is no longer free, therefore there are non-zero correlation functions of
higher order, besides the two-point function.
\par As long as the interactions become weak in the UV sector, at the scale
of the lattice, one would expect that the theory is perturbative at said scale
and that the divergence seen in the free field case would dominate over the
perturbative corrections due to the interactions. Thus, the leading correlation
would again be the two-point function, which has the same behaviour as in
\eqref{eq:twopointdivergence} and would diverge in the continuum limit. This
should be the case in $d=2$, since in that case the gauge coupling is 
superrenormalisable. In $d=3$, the gauge coupling grows logarithmically and
becomes strong in the UV; it also becomes strong in the UV in $d>3$, where
the gauge coupling is non-renormalisable. Therefore, it is unclear what happens
in the continuum limit for those cases, or even if such a limit exists for
them. One interesting possibility is that the theory flows to a non-trivial
fixed point in the UV. In this case, the behaviour of the two-point function
of the electric field would be determined by its anomalous dimension at said UV
fixed point. The short distance contribution would take the form
\begin{equation}
  \langle E_z(\mathbf{k}_{\|})E_z(\mathbf{k}_{\|}')\rangle \sim
\delta^{(d-1)}(\mathbf{k}_{\|}+\mathbf{k}_{\|}')k^2_{\|}\int\frac{dk_z}{k^{(d-2(\delta-1))}}e^{ik_z\Delta}
\end{equation}
where $\delta$ is the anomalous dimension. If $\delta>(d+1)/2$, so that
the anomalous dimension exceeds the engineering dimension, then the
correlations will be even more strongly dominated by the short distance modes
and one expects the arguments for the independence of the mutual information
and relative entropy from the classical term to apply. For $\delta < (d+1)/2$,
the situation is less clear. If the correlations of $E_n$ are finite and
non-divergent, then it could be that the classical piece continues to
contribute in the continuum limit to the mutual information and the relative
entropy.
\par Suppose that the theory is strong coupled in the UV and flows to a weakly
coupled one, which is close to the free non-interacting theory, in the IR at an
energy scacle of order $E\sim \Lambda$. In such cases, one would still expect
thath the contribution to the classical term from modes with energies
$E\leq\Lambda$ is approximately governed by thr two-point function and thus,
the contribution of these modes should drop out in the mutual information or
the relative entropy. It will be worth trying to make this intuitive argument
more precise.

\subsection{Non-Abelian Theory}
Let us now turn our attention to the non-Abelian case. In this case, in the
continuum, the different sectors in the electric centre or the EHS definition
are specified by the  value of $\mathrm{Tr}(E_n^2)$ and other Casismirs (i.e.,
appropriate local gauge invariant operators) on the boundary, where $E_n$ is
the normal component of the electric field. The probability to be in
a particular
sector is a functional of $\mathrm{Tr}(E_n^2)$ and the other Casismirs and we
denote it for ease of notation as $p\left[E_n^2\right]$.
\par Let us start by considering the free $SU(N)$ Yang-Mills theory, say, in
3+1 dimensions, and consider the region of space $z<0$. The boundary lies at
$z=0$. It is easy to show that in this case the two-point function is given by
\begin{gather}
  \langle E_z(\mathbf{k}_{\|})^2E_z(\mathbf{k}_{\|}')^2\rangle
  \sim
  (N^2-1)\delta^{(2)}(\mathbf{k}_{\|}+\mathbf{k}_{\|}')\int{d^2\bar{k}_{\|}\bar{k}_{\|}^2(\mathbf{k}_{\|}-\bar{\mathbf{k}}_{\|})^2}\\
  \int{\frac{dk_z}{\sqrt{k_z^2+\bar{k}^2_{\|}}}}e^{ik_z\Delta}\int\frac{dk_z}{\sqrt{k_z^2'+(\mathbf{k}_{\|}-\bar{\mathbf{k}}_{\|})^2}}e^{ik_x'\Delta}
  \end{gather}
where we have separated the two points  in the $z$ direction by a value $\Delta
z= \Delta$. We see that now the integral over $\bar{k}_{\|}$, the momentum 
along the boundary, is also divergent, since $E_z^2$ is a more singular
operator than $E_z$. Introducing a cut-off along the boundary directions and
not distinguishing it form $\Delta$, the cut-off along the normal direction, we
get 
\begin{equation}
  \langle E_z(\mathbf{k}_{\|})^2E_z(\mathbf{k}_{\|}')^2\rangle
  \sim (N^2-1)\delta^{(2)}(\mathbf{k}_{\|}+\mathbf{k}_{\|}')\frac{1}{\Delta^6},
  \label{eq:twopointnonabel}
\end{equation}
up to logarithmic corrections, which we have not kept carefully. This is much
more singular than the two-point function of $E_z$ in the Abelian case.
\par It is also easy to calculate the higher point correlators for $E_z^2$ in
the free case. The contributions to the $n$-point function can be classified in
terms of the number of disconnected components. The dominant contribution comes
from the maximally disconnected component going like the product $\langle
E_z^2(1)E_z^2(2)\rangle\langle E_z^2(3)E_z^2(4)\rangle\cdots\langle E_z^2(n-1)E_z^2(n)\rangle$
for $n$ even, which diverges as $\Delta^{-2(n+1)}$. In contrast, the fully
connected contribution goes like $\Delta^{-6}$, as before, and is therefore
subdominant. As a result, the two-point correlation function dominates the
correlation functions and hence the probability distribution and therefore the
classical term is determined by the two-point function
\eqref{eq:twopointnonabel}. This two-point function is dominated by the UV
modes at the cut-off. As for the higher-order Casismirs, the two-point function
is even more singular and therefore dominated even more strongly.
\par On turning on the gauge coupling, these correlations will change. However,
Yang-Mills theory is known to be asymptotically free in 3+1 dimensions and
superrenormalisable in 2+1 dimensions. Therefore, in these cases, we expect
the behaviour at the scale of the lattice to continue to be that of the
free theory to good aproximation, and the two-point and higher point
correlations to diverge, as described above. These divergences of course occur
because points on the boundary can come close together. When we consider
two disconnected boundaries, this means that the correlations will be dominated
by their value when the points lie on the same boundary. As a result, in the
continuum limit, the joint probability should satisfy the condition
\begin{equation}
  p((E^2_n)^A,(E^2_n)^B) = p((E^2_n)^A)p((E^2_n)^B),
  \label{eq:mutinfononabel}
\end{equation}
analogous to \eqref{eq:mutinfo} in the Abelian case, and the contribution of
the classical piece to the mutual information should therefore vanish. Note
that this conclusion is equally true in the electric centre definition, where
the classical piece is given by the first term on the RHS of
\eqref{eq:entanglemententropy}, and in the EHS definition, where it is given
by the sum of the first two terms on the RHS of \eqref{eq:entanglemententropy}.
Both these terms vanish if \eqref{eq:mutinfononabel} is true. Similarly, one
can argue that the relative entropy for states which only differ from the
vacuum with finite energy excitations, will also not recieve a contribution
from the classical piece.
\par In higher dimensions $d>3$, the theory is non-renormalisable and the 
situation is more interesting. A continuum limut might exist with the theory
flowing to a non-trivial fixed point in the UV, and as in the discussion above
for the Abelian case, the anomalous dimension of $\mathrm{Tr}(E^2)$ would 
then determine the short distance nature of the two-point and higher point
correlators. Similarly, adding matter can change the bahaviour of the theory.
An important example of this type is if the resulting theory becomes
conformally invariant. Once again, the anomalous dimension of
$\mathrm{Tr}(E^2)$ plays an important role in determining the nature of the
short distance correlators. If the correlators are smooth at short distances,
and do not diverge as $\Delta\rightarrow 0$, then the classical piece could
potentially contribute to both the mutual information and the relative
entropy.
\subsection{$p$-Form Abelian Gauge Theory}
Our discussion of the Abelian $U(1)$ gauge theory can be generalised to the
case of a general $p$-form Abelian theory in $d+1$ dimensions. The action
is given by
\begin{equation}
  S = - \frac{1}{2(p+1)!}\int d^{d+1}x H_{\mu_1\mu_2\cdots\mu_{p+1}}H^{\mu_1\mu_2\cdots\mu_{p+1}}
\end{equation}
where
\begin{equation}
  H_{\mu_1\mu_2\cdots\mu_{p+1}}
  = (p+1)\partial_{[\mu_1}B_{\mu_2\mu_3\cdots\mu_{p+1}]},
\end{equation}
and the square brackets indicate complete anti-symmetrisation.
\par The EHS is obtained by working the the following gauge
\begin{equation}
  B_{0\mu_1\cdots\mu_{p-1}} = 0.
\end{equation}
The resulting constrains, analogous to the Gauss law, are
\begin{equation}
  \partial^i H_{0i\mu_1\cdots\mu_{p-1}} = 0.
\end{equation}
The different superselection sectors, for the electric centre choice are
therefore specified by the value for the normal component of the 
electric field, $H_{0n\mu_1\cdots\mu_p}\equiv H_{0i\mu_1\cdots\mu_p}n^i$.
\par The resulting classical term is then determined by the two-point 
function of the normal component. For a planar boundary, which we continue to
denote as $z=0$, we have
\begin{multline}
  \langle H_{0zi_1i_2\cdots i_{p-1}}(\mathbf{k}_{\|})H_{0zj_1j_2\cdots
    j_{p-1}}(\mathbf{k}_{\|}')\rangle\sim\delta^{(d-1)}(\mathbf{k}_{\|}+\mathbf{k}_{\|}')k^2_{\|}(\delta^T_{i_1[j_1}\delta^T_{i_2j_2}\cdots\delta^T_{i_{p-1}j_{p-1}]})\\
    \times\int\frac{dk_z}{\sqrt{k_z^2+k^2_{\|}}}e^{ik_z\Delta}
  \end{multline}
Here
\begin{equation}
  \delta^T_{ij}\equiv\delta_{ij} - \frac{(k_{\|})_i(k_{\|})_j}{k^2_{\|}}
\end{equation}
is the delta function transverse to the spatial momentum along the boundary
$\mathbf{k}_{\|}$, and the square brackets in the product of the delta
functions indicates complete anti-symmetrisation with respect to the
second indices $j_1,j_2,\cdots,j_{p-1}$.
\par We see that the two-point function generally involves a Laplacian for
a $(p-1)$-form living on the boundary which is $(d-1)$-dimensional. The
contribution is once again dominated by short distance modes and
logarithmically dependent on the cut-off $\Delta$. As a result, we see that the
contribution of the classical piece drops out of the mutual information or
relative entropy.\footnote{As in the gauge field case, the logarithmic
  dependence on $\Delta$ is true for modes with $k_{\|}\ll\Delta^{-1}$.
  However, it also follows that these modes have zero contribution to the
mutual information and the relative entropy in the classical piece.}
\par Let us also note that when we work with a region which is not the half
space and the boundary is no longer planar, the logarithmic divergence will be
cut off by the size of the region of interest. Also, in those more general
cases, one should be more careful about the zero modes and their contributions.
\section{Entanglement and Dualities}
In this section, we will study some aspects of the duality between a theory
with a $p$-form gauge potential in $d+1$ dimensions and the dual theory, which
is equipped with a $(d-p-1)$-form potential. This is a broad generalisation of
the electric-magnetic duality in 3+1 dimensions for a gauge field, and is also
closely related to the Kramers-Wannier duality on the lattice. Our interest
will be to study how the different choices for the centre of the algebra of
observables transform under this duality and the accompanying changes in the
entaglement entropy and related quantities. In particular, we will consider two
choices of centers for the $p$-form theory, called the electric and magnetic 
centers and study how they map under duality. We will find that the magnetic 
centre choice for a region $R$ in the $p$-form case maps to an algebra 
of observables in the dual theory which is closesly related but not identical
to the  electric centre of the  dual $(d-p-1)$-form theory in a suitable region
$\tilde{R}$ of the dual lattice. In the continuum limit, these differences
become uninmortant for ultraviolet insensitive quantities, like the relative
entropy and mutual information,  as we mentioned above, and the two dual
theories therefore agree.

\section{Conclusions}
In this section :
