%\chapterauthor{Second Author}{Second Author Affiliation}`
\chapter{Supersymmetry}
\chapterauthor{Ivo Iliev\\Sofia University}
\adjustmtc
\minitoc
\section{Supermultiplets}

\begin{definition}[Supermultiplet]
  Representations of the supersymmetric algebra (superalgebra) are called
  supermultiplets.
\end{definition}

Indeed, these representations can be thought of as multiplets where we assemble
together several different representations of the Lorentz algebra, since the
latter is a subalgebra of the superalgebra.

\subsection{Massless supermultiplets}
If $P^2=0$, then we can take $P_\mu$ to a canonical form by applying boost and
rotations until it reads

\begin{equation}
  \sigma^{\mu}_{\alpha\dot{\alpha}}P_\mu = \left(\sigma^0+\sigma^3\right)E = 
  \begin{bmatrix}
    0 & 0\\
    0 & 2E
  \end{bmatrix}
\end{equation}

The supersymmetric algebra becomes,
\begin{equation}
  \begin{bmatrix}
    \{Q_1,\bar{Q}_{\dot{1}}\} & \{Q_1,\bar{Q}_{\dot{2}}\} \\

    \{Q_2,\bar{Q}_{\dot{1}}\} & \{Q_2,\bar{Q}_{\dot{2}}\} \\

  \end{bmatrix}
  =
  \begin{bmatrix}
    0 & 0\\
    0 & 4E
  \end{bmatrix}
\end{equation}
intended as acting on the states of the multiplet we are looking for. In
particular,
\begin{equation}
\{Q_1, \bar{Q}_{\dot{1}}\} = 0
\end{equation}
which implies that 
\begin{equation}
  ||Q_1|\omega\rangle||^2 = 0 = ||\bar{Q}_{\dot{1}}|\omega\rangle||^2
\end{equation}
and thus

\begin{equation}
  Q_1|\omega\rangle = 0 = \bar{Q}_{\dot{1}}|\omega\rangle.
\end{equation}
This means that as operators $Q_1$ and $\bar{Q}_{\dot{1}}$ annihilate the
multiplet. 
\par The only non-trivial anticommutation relation that is left is:
\begin{equation}
  \{Q_2,\bar{Q}_{\dot{2}}\} = 1
\end{equation}
If we call
\begin{equation}
  \alpha = \frac{1}{2\sqrt{E}}Q_2,\quad \alpha^\dagger
  = \frac{1}{2E}\bar{Q}_{\dot{2}}
\end{equation}
then the anticommutation relation that is left is:
\begin{equation}
  \{\alpha,\alpha^\dagger\} = 1
\end{equation}
with $\{\alpha,\alpha\} = 0$.

We can build the representation starting from a state $|\lambda\rangle$ such
that
\begin{equation}
  \alpha|\lambda\rangle = 0
\end{equation}
Lets suppose that it has \textit{helicity} $\lambda$:
\begin{equation}
  M_{12}\lambda \equiv J_3|\lambda\rangle = \lambda|\lambda\rangle.
\end{equation}
It is easy to compute the helicity of $\alpha^{\dagger}|\lambda\rangle$:
\begin{equation}
  M_{12} \bar{Q}_{\dot{2}}|\lambda\rangle
  = \left(\bar{Q}_{\dot{2}}M_{12}+\frac{1}{2}\bar{Q}_{\dot{2}}\right)|\lambda\rangle
  = (\lambda+\frac{1}{2}\bar{Q}_{\dot{2}})|\lambda\rangle 
\end{equation}
In the last line, we have used the fact that
$\left[M_{12},\bar{Q}_{\dot{2}}\right]=\frac{1}{2}\bar{Q}_{\dot{2}}$. Thus we
found out that
\begin{equation}
  \alpha^\dagger|\lambda\rangle  = |\lambda+ \frac{1}{2}\rangle 
\end{equation}
Since $(\alpha^\dagger)^2 = 0$, this stops here. Hence we have
\begin{equation}
  \alpha^\dagger|\lambda+\frac{1}{2}\rangle = 0.
\end{equation}
\par
Massless multiplets are thus composed of one boson and one fermion. Since
physical particles must come in CPT conjugate representation (or, there are no
spin-$\frac{1}{2}$ one dimensional representations of the massless little group
of the Lorentz group), one must add the CPT conjugate multiplet where
helicities are flipped.
\begin{example}[Examples of massless supermultiplets]
  \mbox{}
  \begin{itemize}
    \item The \textit{scalar} multiplet is obtained by setting $\lambda =0$.
      Then we have 
      \begin{equation}
        \alpha^\dagger\ket{0} = \ket{\frac{1}{2}}
      \end{equation}
      The full multiplet is composed of two states with $\lambda = 0 $ and
      a doublet with $\lambda = \pm\frac{1}{2}$. These are the degrees of
      freedom of a complex scalar and a Weyl (chiral) fermion.
    \item The \textit{vector} multiplet is obtained starting from a $\lambda
      = \frac{1}{2}$ state. We get
      \begin{equation}
        \alpha^\dagger\ket{\frac{1}{2}} = \ket{1}.
      \end{equation}
      To this we add the CPT conjugate multiplet, to obtain two pairs of
      states, one with $\lambda = \pm\frac{1}{2}$ and the other with $\lambda
      = \pm 1$. These are the degrees of freedom of a Weyl fermion and of
      a massless vector. The latter is usually interpreted as a gauge boson.
    \item Another multiplet is obtained starting from $\lambda = \frac{3}{2}$:
      \begin{equation}
        \alpha^\dagger\ket{\frac{3}{2}} = \ket{2}.
      \end{equation}
      Adding the CPT conjugate, one has a pair of bosonic degrees of freedom
      with $\lambda = \pm 2$, which we interpret as the \textit{graviton}, and
      a pair of fermionic degrees of freedom with $\lambda =\pm\frac{3}{2}$,
      which correspond to a massless spin-$\frac{3}{2}$ Rarita-Schwinger field,
      also called the \textit{gravition}, since it is the SUSY partner of the
      graviton, as was just shown.
  \end{itemize}
\end{example}
\subsection{Supermultiplets of extended supersymmetry}
Very briefly we will mention that having extended SUSY, the massless
supermultiplets are longer. Let's take the algebra to be:
\begin{equation}
  \{Q_\alpha^I,\bar{Q}_{\dot{\alpha}}^J\} = 2\sigma_{\alpha\dot{\alpha}}^\mu
  P_\mu\delta^{IJ},
\end{equation}
where for simplicity we suppose that $Z^{IJ}=0$ for these states. For massless
states, $P_\mu = (E,0,0,E)$ and therefore as before we have that
\begin{equation}
  \{Q_1^I,\bar{Q}_{\dot{1}}^J\} = 0,
\end{equation}
which implies the (operator) equations $Q_1^I=0$ and $\bar{Q}_{\dot{1}}^I=0$,
for $I=1,\cdots,\sn$. The non-trivial relations are then:
\begin{equation}
\{Q_2^I, \bar{Q}_{\dot{2}}^J\} = 4E\delta^{IJ}
\end{equation}
Of course we can define
\begin{equation}
  \alpha_I = \frac{1}{2\sqrt{E}}Q_2^I
\end{equation}
and obtain the canonical anticommutation relations for $\sn$ fermionic
oscillators
\begin{equation}
  \{\alpha_I,\alpha_J^\dagger\} = \delta_{IJ}
\end{equation}
\par If we now start with a state $\ket{\lambda}$ with helicity $\lambda$ which
satisfies $\alpha_I\ket{\lambda} = 0$, we build a multiplet as follows:
\begin{gather}
  \alpha_I^\dagger\ket{\lambda} = \ket{\lambda + \frac{1}{2}}_I,\nonumber\\
  \alpha_I\dagger\alpha_J\dagger\ket{\lambda} = \ket{\lambda
  + 1}_{[IJ]},\nonumber\\
  \vdots\nonumber\\
  \alpha_1^\dagger\cdots\alpha_{\sn}^\dagger\ket{\lambda}
  = \ket{\lambda+\frac{\sn}{2}}
\end{gather}
It is very important to note that there are $\sn$ states with helicity
$\lambda + \frac{1}{2}, \frac{1}{2}\sn(\sn-1)$ states with
helicity $\lambda + 1$ and so on, until we reach a single state with helicity $\lambda
+\frac{\sn}{2}$ (it is totally antisymmetric in $\sn$ indices
I). In total, the supermultiplet is composed of $2^\sn$ states, half of
them bosonic and half of them fermionic.
\par Interestingly, in this case we can now have self-CPT conjugate multiplets.
Take for example $\sn = 4$ and start from $\lambda = -1$. Then $\lambda
+ \frac{\sn}{2} = 1$ and the multiplet spans states of opposite helicities,
thus filling complete representations of the Lorenz group. Indeed, it contains
one pair of states with $\lambda\pm1$ (a vector, i.e a gauge boson), 4 pairs of
states with $\lambda=\pm\frac{1}{3}$ (4 Weyl Fermions) and 6 states with
$\lambda = 0$ (6 real scalars, or equivalently 3 complex scalars).
\par Another example is $\sn = 8$ supersymmetry. Here if we start with $\lambda
= -2$ we end up with $\lambda + \frac{\sn}{2} = 2$. Thus in this case we have
the graviton in the self-CPT conjugate multiplet, corresponding to the pair of
states with $\lambda\pm2$. In addition, we have 8 massless gravitini with
$\lambda\pm\frac{3}{2}$, 28 massless vectors with $\lambda = \pm 1$, 56
massless Weyl fermions with $\lambda = \pm\frac{1}{2}$ and finally 70 real
scalars with $\lambda = 0$. This is the content of $\sn = 8$
\textit{supergravity}, which is the only multiplet of $\sn = 8$ supersymmetry
with $|\lambda|<2$. The latter condition is necessary in order to have
consistent couplings (higher spin fields cannot be coupled in a consistent way
with gravity and lower spin fields).
\par From the theoretical standpoint, this is a very nice result, because we
have a theory where \textit{everything is determined} from symmetry alone: the
complete spectrum and all the couplings. Unfortunately, this theory is also
completely unphysical. To mention one problem, it has no room for fermions in
complex representations of the gauge group, which are present in the Standard
Model.

\subsection{Massive supermultiplets}
When $P^2=M^2>0$, by boosts and rotation $P_\mu$ can be put in the following
form
\begin{equation}
  P_\mu = (M,0,0,0)
\end{equation}
Then we have
\begin{equation}
  \sigma_{\alpha\dot{\alpha}}^\mu P_\mu = M\sigma^0 = 
  \begin{bmatrix} 
    M & 0\\
    0 & M
  \end{bmatrix}
\end{equation}
so that the superalgebra reads
\begin{equation}
  \{Q_\alpha, \bar{Q}_{\dot{\alpha}}\} = 2 M\delta_{\alpha\dot{\alpha}}
\end{equation}
Note that $[M_{12}, Q_1] = i(\sigma_{12})_1\ ^1Q_1= \frac{1}{2}Q_1$, thus it is
$Q_1$ that raises the helicity, in the same way as $\bar{Q}_{\dot{2}}$. We make
the redefinition
\begin{gather}
  \alpha_1 = \frac{1}{\sqrt{2M}}\bar{Q}_{\dot{1}},\quad \alpha_1^\dagger
  = \frac{1}{\sqrt{2M}}Q_1, \\
  \alpha_2 = \frac{1}{\sqrt{2M}}Q_2,\quad \alpha_2^\dagger
  = \frac{1}{\sqrt{2M}}\bar{Q}_{\dot{2}},
\end{gather}
so that we have the canonical anticommutation relations of two fermionic
oscillators:
\begin{equation}
  \{\alpha_a,\alpha_b^\dagger\}=\delta_{ab},\quad a,b = 1,2.
\end{equation}
\par If we start with $\alpha_a\ket{\lambda} = 0$, $M_{12}\ket{\lambda}
= \lambda\ket{\lambda}$, then we build the multiplet as:
\begin{gather}
  \alpha_1^\dagger\ket{\lambda} = \ket{\lambda + \frac{1}{2}}_1,\\
  \alpha_2^\dagger\ket{\lambda} = \ket{\lambda+\frac{1}{2}}_2,\\
  \alpha_1^\dagger\alpha_2^\dagger\ket{\lambda} = \ket{\lambda+1}.
\end{gather}
There are 4 states now (compared to the 2 in the massless case), two bosons and
two fermions. 
\begin{example}[Examples of massive supermultiplets]
  \mbox{}
  \begin{itemize}
    \item In the case of the \textit{massive scalar multiplet}, we start from
      $\lambda =  -\frac{1}{2}$ and obtain two states with $\lambda=0$ and one
      state with $\lambda=\frac{1}{2}$. These are the degrees of freedom of one
      massive complex scalar and one massive Weyl fermion. Note that the latter
      might not be familiar. Indeed, one cannot write the usual Dirac mass term
      for a Weyl fermion. Instead, one can write what is called a Majorana mass
      term:
      \begin{equation}
        \mathcal{L}\supset m\epsilon^{\alpha\beta}\psi_\alpha\psi_\beta + h.c.
      \end{equation}
      Note that the total degrees of freedom of a massless scalar multiplet is
      the same as that of a massive one.
    \item For a \textit{massive vector multiplet}, start from $\lambda = 0$ to
      obtain 2 states with $\lambda = \frac{1}{2}$ and one state with $\lambda
      = 1$. To this we add the CPT conjugate multiplet so that in the end we
      have one pair with $\lambda = \pm1$, two pairs with
      $\lambda=\pm\frac{1}{2}$ and two states with $\lambda = 0$. According to
      the massive little group, this corresponds to 1 massive vector (with
      $\lambda = \pm1,0)$, 1 real scalar and 1 massive Dirac fermion. Note
      however that the content in degrees of freedom is the same as that of one
      massless vector multiplet together with one massless scalar multiplet.
      This hints that the consistent way to treat massive vectors in
      a supersymmetric field theory will be through a SUSY version of the
      Brout-Englert-Higgs mechanism.
    \end{itemize}

  \end{example}
\section{General}
\begin{definition}[R-symmetry]
In supersymmetric theories, an R-symmetry is the symmetry transforming
different supercharges into each other. In the simplest case of the $\sn=1$
supersymmetry, such R-symmetry is isomorphic to a global $U(1)$ group or its discrete subgroup. For extended supersymmetry theories, the R-symmetry group becomes a global non-abelian group.  
\end{definition}

\begin{remark}
In the case of the discrete subgroup $\mathbb{Z}_2$, the R-symmetry is called \textit{R-parity}
\end{remark}

\begin{definition}[Extended supersymmetry]
In supersymmetric theories, when $\sn>1$ the algebra is said to have
\textit{extended supersymmetry}.
\end{definition}

\section{Bogomol'nyi-Prasad-Sommerfield (BPS) states}
\begin{definition}[BPS state]
A massive representation of an extended supersymmetry algebra that has mass
equal to the supersymmetry central charge $Z$ is called an \textit{BPS} state.
\end{definition}

Quantum mechanically speaking, if the supersymmetry remains unbroken, exact
solutions to the modulus of $Z$ exist. Their importance arises as the
multiplets shorten for generic representations, with stability and mass formula
exact.

\begin{example}[$d=4$,$\sn=2$]
The generators for the odd part of the superalgebra have relations:
\begin{gather}
  \{Q_\alpha^A,\bar{Q}_{\dot{\beta}B}\} = 2\sigma^m_{\alpha\dot{\beta}}P_m\delta^A_B \\
  \{Q_\alpha^A,Q_{\beta B}\} = 2\epsilon_{\alpha\beta}\epsilon^{AB}\bar{Z}\\
  \{\bar{Q}_{\dot{\alpha}A},\bar{Q}_{\dot{\beta}B}\}
  = -2\epsilon_{\dot{\alpha}\dot{\beta}}\epsilon_{AB}Z,
\end{gather}
where $\alpha\dot{\beta}$ are the Lorentz group indices and $A,B$ are the
$R$-symmerty indices.
If we take linear combinations of the above generators as follows:
\begin{gather}
  R_\alpha^A = \xi^{-1}Q_\alpha^A
  + \xi\sigma^0_{\alpha\dot{\beta}}\bar{Q}^{\dot{\beta}B}\\
  T_\alpha^A = \xi^{-1}Q_\alpha^A - \xi\sigma^0_{\alpha\dot{\beta}}\bar{Q}^{\dot{\beta}B}
\end{gather}
and consider a state $\psi$ which has momentum $(M,0,0,0)$, we have:
\begin{equation}
  \left(R_1^1+(R_1^1)^\dagger\right)^2\psi = 4(M+Re(Z\xi^2))\psi,
\end{equation}
but because this is the square of a Hermitian operator, the right hand side
coefficient must be positive for all $\xi$. In particular, the strongest result from this is
\begin{equation}
  M\geq|Z|
\end{equation}
\end{example}
\section{Supersymmetric theories on curved manifolds}
\textit{Remark:} Supersymmetric theories may be defined only on backgrounds admitting solutions to certain Killing spinor equations,
\begin{align}
\left(\nabla_\mu-iA_\mu\right)\zeta + iV_\mu\zeta + iV^\nu\sigma_{\mu\nu}\zeta = 0\\
\left(\nabla_\mu+iA_\mu\right)\tilde{\zeta} - iV_\mu\tilde{\zeta} -  iV^\nu\tilde{\sigma}_{\mu\nu}\tilde{\zeta} = 0
\end{align}
which in four dimensions and Euclidean signature are
equivalent to the requirement that the manifold is complex and the metric Hermitian.

\section{Supersymmetric Chern-Simons-matter theories}

In this section we will introduce the basic building blocks of supersymmetric
Chern-Simons-matter theories. We will work in Euclidean space, and we will put
the theories on the three-sphere, since we are eventually interested in
computing the free energy of the gauge theory in this curved space. 
\subsection{Conventions}
In Euclidean space, the fermions $\psi$ and $\bar{\psi}$ are independent and 
they transform and they transform in the same representation of the Lorentz
group. Their index structure is:
\begin{equation}
  \psi^\alpha, \quad\bar{\psi}^\alpha .
\end{equation}
We will take $\gamma_\mu$ to be the Pauli matrices, which are hermitian, and
\begin{equation}
  \gamma_{\mu\nu} = \frac{1}{2}\left[\gamma_\mu,\gamma_nu\right]
= i\epsilon_{\mu\nu\rho}\gamma^{\rho}
\end{equation}
We introduce the usual symplectic product through the antisymmetric matrix
\begin{equation}
  C_{\alpha\beta} = 
  \begin{pmatrix}
        0 & C\\
        -C & 0
  \end{pmatrix}.
\end{equation}
We set $C=-1$ and denote the matrix by $\epsilon_{\alpha\beta}$. The product
is
\begin{equation}
  \bar{\epsilon}\lambda = \bar{\epsilon}^\alpha
  C_{\alpha\beta}\lambda^\beta .
\end{equation}
Notice that
\begin{equation}
  \bar{\epsilon}\gamma^\mu\lambda = \bar{\epsilon}^\beta
  C_{\beta\gamma}(\gamma^\mu)^\gamma_\alpha\lambda^\alpha .
\end{equation}
It is easy to check that
\begin{equation}
\bar{\epsilon}\lambda = \lambda\bar{\epsilon},\quad
\bar{\epsilon}\gamma^\mu\lambda = -\lambda\gamma^\mu\bar{\epsilon},
\end{equation}
and in particular
\begin{equation}
  (\gamma^\mu\bar{\epsilon})\lambda = -\bar{\epsilon}\gamma^\mu\lambda .
\end{equation}
We also have the following Fierz identities
\begin{equation}
  \bar{\epsilon}(\epsilon\psi) + \epsilon(\bar{\epsilon}\psi)
  + (\bar{\epsilon}\epsilon)\psi = 0
\end{equation}
and
\begin{equation}
  \epsilon(\bar{\epsilon}\psi) + 2(\bar{\epsilon}\epsilon)\psi
  + (\bar{\epsilon}\gamma_\mu\psi)\gamma^\mu\epsilon = 0.
\end{equation}
\subsection{Vector multiplet and supersymmetric Chern-Simons theory}
  We first start with theories based on vector multiplets. The three
  dimensional Euclidean $\mathcal{N}=2$ vector superfield $V$ has the following
  content
  \begin{equation}
    V: \quad A_\mu,\sigma,\lambda,\bar{\lambda}, D,
  \end{equation}
  where $A_\mu$ is a gauge field, $\sigma$ is an auxiliary scalar field,
  $\lambda,\bar{\lambda}$ twi-component complex Dirac spinors, and $D$ is an 
  auxiliary scalar. This is just the dimensional reduction of the
  $\mathcal{N}=1$ vector multiplet in 4 dimensions, and $\sigma$ is the
  reduction of the fourth component of $A_\mu$. All fields are valued in
  the Lie algebra $\mathfrak{g}$ of the gauge group $G$. For $G=U(N)$ our 
  convention is that $\mathfrak{g}$ are Hermitian matrices. It follows
  that the gauge covariant derivative is given by
  \begin{equation}
    \partial_\mu + i\left[A_\mu,.\;\right]
  \end{equation}
  while the gauge field strength is
  \begin{equation}
    F_{\mu\nu} = \partial_\mu A_\nu - \partial_\nu A_\mu
    + i\left[A_\mu,A_\nu\right].
  \end{equation}
The transformations of the fields are generated by two independent complex 
structures $\epsilon$, $\bar{\epsilon}$. They are given by,
\begin{align}
  \delta A_\mu &= \frac{i}{2}(\bar{\epsilon}\gamma_\mu\lambda
  - \bar{\lambda}\gamma_\mu\epsilon),\nonumber\\
  \delta\sigma &= \frac{1}{2}({\bar{\epsilon}}\lambda
  - \bar{{\lambda}}\epsilon),\nonumber\\
  \delta\lambda &= -\frac{1}{2}\gamma^{\mu\nu}\epsilon F_{\mu\nu} - D\epsilon
  + i\gamma^\mu\epsilon D_\mu\sigma + \frac{2i}{3}\sigma\gamma^\mu
  D_\mu\epsilon,\nonumber\\
  \delta\bar{\lambda} &= - \frac{1}{2}\gamma^{\mu\nu}F_{\mu\nu} + D\bar{\epsilon}
  - i\gamma^\mu\bar{\epsilon}D_\mu\sigma
  - \frac{2i}{3}\sigma\gamma^{\mu}D_\mu\bar{\epsilon},\nonumber\\
  \delta D &= -\frac{i}{2}\bar{\epsilon}\gamma^\mu D_\mu\lambda
  - \frac{i}{2}D_\mu\bar{\lambda}\gamma^\mu\epsilon
  + \frac{i}{2}\left[\bar{\epsilon}\lambda,\sigma\right]
  + \frac{i}{2}\left[\bar{\lambda}\epsilon, \sigma\right]
  - \frac{i}{6}(D_\mu\bar{\epsilon}\gamma^\mu\lambda
  + \bar{\lambda}\gamma^{\mu}D_\mu\epsilon),
\end{align}
and we split naturally
\begin{equation}
\delta = \delta_\epsilon + \delta_{\bar\epsilon}.
\end{equation}
Here we follow the conventions of \cite{Hama11}, be we change the sign of the gauge
connection: $A_\mu\rightarrow - A_\mu$. The derivative $D_\mu$ is covariant
with respect to both the gauge field and the spin connection. On all the
fields, except $D$, the commutator
$\left[\delta_{\epsilon},\delta_{\bar{\epsilon}}\right]$ becomes a sum
of translation, gauge transformation, Lorentz rotation, dilation and
$R$-rotation:
\begin{align}
  \left[\delta_\epsilon, \delta_{\bar{\epsilon}}\right]A_\mu
  &= iv^\nu\partial_\nu A_\mu + i\partial_\mu\nu^\nu A_\nu
  - D_\mu\Lambda,\nonumber\\
  \left[\delta_\epsilon, \delta_{\bar{\epsilon}}\right]\sigma
  &= i\nu^\mu\partial_\mu\sigma + i\left[\Lambda, \sigma\right] + \rho\sigma,
  \nonumber\\
  \left[\delta_\epsilon, \delta_{\bar{\epsilon}}\right]\lambda
  &= i\nu^\mu\partial_\mu\lambda
  + \frac{i}{4}\Theta_{\mu\nu}\gamma^{\mu\nu}\lambda + i\left[\Lambda,
  \lambda\right] + \frac{3}{2}\rho\lambda + \alpha\lambda,\nonumber\\
  \left[\delta_\epsilon, \delta_{\bar{\epsilon}}\right]\bar{\lambda}
  &= i\nu^\mu \partial_\mu\bar{\lambda}
  + \frac{i}{4}\Theta_{\mu\nu}\bar{\lambda} + i\left[\Lambda,
  \bar{\lambda}\right] + \frac{3}{2}\rho\bar{\lambda}
  - \alpha\bar{\lambda},\nonumber\\
  \left[\delta_\epsilon, \delta_{\bar{\epsilon}}\right]D &=i\nu^\mu\partial_\mu
  D + i\left[\Lambda, D\right] + 2\rho
  D + \frac{1}{3}\sigma(\bar{\epsilon}\gamma^\mu\gamma^{\nu}D_\mu D_\nu\epsilon
  -\epsilon\gamma^\mu\gamma^\nu D_\mu D_\nu\bar{\epsilon}),
\end{align}
where
\begin{align}
  \nu^\mu &= \bar{\epsilon}\gamma^\mu\epsilon,\nonumber\\
  \Theta^{\mu\nu} &= D^{[\mu}v^{\nu]}
  + \nu^\lambda\omega^{\mu\nu_\lambda},\nonumber\\
  \Lambda &= \nu^\mu i A_\mu + \sigma\bar{\epsilon}\epsilon\nonumber\\
  \rho &= \frac{i}{3}(\bar{\epsilon}\gamma^{\mu}D_\mu\epsilon + D_\mu
  \bar{\epsilon}\gamma^\mu\epsilon),\nonumber\\
  \alpha &= \frac{i}{3}(D_\mu\bar{\epsilon}\gamma^{\mu}\epsilon
  - \bar{\epsilon}\gamma^\mu D_\mu\epsilon).
\end{align}
Here, $\omega_\lambda^{\mu\nu}$ is the spin connection. As a check, let us
calculate the commutator acting on $\sigma$. We have,
\begin{align}
  \left[\delta_\epsilon, \delta_{\bar{\epsilon}}\right]\sigma
  &= \delta_\epsilon\left(\frac{1}{2}\bar{\epsilon}\lambda\right)
  - \delta_{\bar{\epsilon}}\left(-\frac{1}{2}\bar{\lambda}\epsilon\right)\nonumber\\
  &=\frac{1}{2}\bar{\epsilon}\left(-\frac{1}{2}\gamma^{\mu\nu}\epsilon
    F_{\mu\nu}
  - D\epsilon + i\gamma^\mu\epsilon D_\mu \sigma\right)
  + \frac{i}{3}\bar{\epsilon}\gamma^{\mu}(D_\mu\epsilon)\sigma\nonumber\\
  &+ \frac{1}{2}(-\frac{1}{2}\gamma^{\mu\nu}\bar{\epsilon}F_{\mu\nu}
  + D\epsilon - i\gamma^{\mu}\bar{\epsilon}D_\mu\sigma)\epsilon
  - \frac{i}{3}\gamma^\mu(D_\mu\bar{\epsilon})\epsilon\sigma\nonumber\\
  &= i\bar{\epsilon}\gamma^\mu\epsilon D_\mu\sigma + \rho\sigma.
\end{align}
In order for the supersymmetry algebra to close, the last term in the RHS of
$\left[\delta_\epsilon, \delta_{\bar{\epsilon}}\right]D$ must vanish. This is
the case if the Killing spinors satisfy:
\begin{equation}
  \gamma^\mu\gamma^\nu D_\mu D_\nu\epsilon = h\epsilon,\quad
  \gamma^\mu\gamma^\nu D_\mu D_\nu \bar{\epsilon} = h\bar{\epsilon}
  \label{eq:spinorrequirement}
\end{equation}
for some scalar function $h$. A sufficient condition for this is to simply have
\begin{equation}
  D_\mu\epsilon = \frac{i}{2r}\gamma_\mu\epsilon, \quad
  D_\mu\bar{\epsilon}=\frac{i}{2r}\gamma y_\mu\bar{\epsilon}
\end{equation}
and
\begin{equation}
  h = - \frac{9}{4r^2}
  \label{eq:equationh}
\end{equation}
where $r$ is the radius of the three-sphere. This condition is satisfied by one
of the Killing spinors on the three-sphere (the one which is constant in the
left-invariant frame). Notice that, with this choice $\rho$ vanishes.
\par The Euclidean SUSY Chern-Simons (CS) action, in flat space, is given by
\begin{align}
  S_{SCS} &= - \int d^3x \mathrm{Tr}\left(A\wedge dA + \frac{2i}{3}A^3
    - \bar{\lambda}\lambda + 2D\sigma\right)\\
    &= -\int d^3x
    \mathrm{Tr}\left(\epsilon^{\mu\nu\rho}\left(A_\mu\partial_\nu A_\rho
      + \frac{2i}{3}A_\mu A_\nu A_\rho\right)
    - \bar{\lambda}{\lambda}+2D\sigma\right).
  \end{align}

  Here $\mathrm{Tr}$ denotes the trace in the fundamental representation. The
  part of the action involving the gauge connection $A$ is the standard,
  bosonic CS action in three dimensions. This action was first considered
  from the point of view of QFT, in \cite{Deser82}, where the total action for
  a non-abelian gauge field was the sum of the standard Yang-Mills action and
  the CS action. In \cite{Witten89}, the CS action was considered by itself and shown
  to lead to a topological gauge theory.
  \par We can check that the SUSY CS action is invariant under the
  supersymmetry generated by $\delta_\epsilon$ (the proof for
  $\delta_{\bar{\epsilon}}$ is similar). The  SUSY variation of the integrand
  of the action is
  \begin{multline}
    (2\delta A_\mu\partial_\nu A_\rho + 2i\delta A_\mu A_\nu
    A_\rho)\epsilon^{\mu\nu\rho} - \bar{\lambda}\delta\lambda + 2(\delta
    D)\sigma + 2D\delta\sigma = \\-i\bar{\lambda}\gamma_\mu\epsilon\partial_\nu
    A_\rho\epsilon^{\mu\nu\rho} + \bar{\lambda}\gamma_\mu\epsilon A_\nu A_\rho
    \epsilon^{\mu\nu\rho} - \bar{\lambda}(-\frac{1}{2}\gamma^{\mu\nu}F_{\mu\nu}
    - D + i\gamma^\mu D_\mu\sigma)\epsilon
    - \frac{2i}{3}\bar{\lambda}\gamma^\mu D_\mu\epsilon\sigma\\
    -i(D_\mu\bar{\lambda})\gamma^\mu\sigma\epsilon
    + i\left[\bar{\lambda}\epsilon, \sigma\right]\sigma
    - \frac{i}{3}\bar{\lambda}\gamma^\mu D_\mu\epsilon\sigma
    - \bar{\lambda}\epsilon D.
  \end{multline}
  It is obvious that the terms involving $D$ cancel. Let us consider the terms
  involving the gauge field. Using the gamma matrix comutator, we find
  \begin{equation}
    \frac{1}{2}\bar{\lambda}\gamma^{\mu\nu} F_{\mu\nu}\epsilon = i\bar{\lambda}
    \gamma_\rho\epsilon\epsilon^{\mu\nu\rho}\partial_\mu A_\nu
    - \bar{\lambda}\gamma_\rho\epsilon\epsilon^{\mu\nu\rho}A_\mu A_\nu
  \end{equation}
  which cancels the first two terms in the variation. Let us now look at the
  remaining terms. The covariant derivative of $\bar{\lambda}$ is
  \begin{equation}
    D_\mu \bar{\lambda} = \partial_\mu\bar{\lambda}
    + \frac{i}{2r}\gamma_\mu\bar{\lambda} + i\left[A_\mu, \bar{\lambda}\right].
  \end{equation}
  If we integrate by parts the term involving the derivative of $\lambda$ we
  find in total
  \begin{multline}
    i\bar{\lambda}\gamma^\mu\epsilon\partial_\mu\sigma
    + i\bar{\lambda}\gamma^\mu\partial_\mu\epsilon\sigma
    + \frac{1}{2r}(\gamma^\mu\bar{\lambda})\gamma_\mu\epsilon
    + \left[A_\mu,\bar{\lambda}\right]\gamma^\mu\epsilon\sigma=\\
    =i\bar{\lambda}\gamma^\mu\epsilon\partial_\mu\sigma
    + i\bar{\lambda}\gamma^\mu D_\mu \epsilon \sigma + \left[A_\mu,
    \bar{\lambda}\right]\gamma^\mu\epsilon\sigma
  \end{multline}
  The derivative of $\sigma$ cancels against the corresponding term in the
  covariant derivative of $\sigma$. Putting all together, we find
  \begin{equation}
    i\bar{\lambda}\gamma^\mu(D_\mu\epsilon)\sigma
    - i \bar{\lambda}\gamma^\mu(D_\mu\epsilon)\sigma + \left[A_\mu,
    \bar{\lambda}\right]\gamma^\mu\epsilon\sigma
      + \bar{\lambda}\gamma^\mu\epsilon\left[A_\mu,
      \sigma\right]+i\left[\bar{\lambda}\epsilon,\sigma\right]\sigma.
\end{equation}
The last three terms cancel due to the cyclic property of the trace. This
proves the invariance of the SUSY CS theory.
\par In the path integral, the SUSY CS action enters the form
\begin{equation}
 \mathrm{exp}\left(\frac{ik}{4\pi}S_{\mathrm{SCS}}\right)
 \label{eq:susycspathintegral}
\end{equation}
where $k$ plays the role of the inverse coupling constant and it is referred to
as the level of the CS theory. In a consistend quantum theory,  $k$ must be an
integer. This is due to the fact that the Chern-Simons action for the
connection $A$ is not invariant under large gauge transformations, but changes
by an integer times $8\pi^2$. The quantization of $k$ guarantees that
\eqref{eq:susycspathintegral} remains invariant.
\par Of course, there  is another Lagrangian for vector multiplets, namely
the Yang-Mills Lagrangian,
\begin{equation}
  \mathcal{L}_{\mathrm{YM}} = \mathrm{Tr}\left[\frac{1}{4}F_{\mu\nu}F^{\mu\nu}
    + \frac{1}{2}D_\mu\sigma D^\mu\sigma
    + \frac{1}{2}\left(D+\frac{\sigma}{r}\right)^2
    + \frac{i}{2}\bar{\lambda}\gamma^\mu D_\mu \lambda
    + \frac{i}{2}\bar{\lambda}\left[\sigma,\lambda\right]
  - \frac{1}{4r}\bar{\lambda}\lambda\right].
  \label{eq:ymlagrangian}
\end{equation}
In the flat space limit $r\rightarrow\infty$, this becomes the standard
(Euclidean) super Yang-Mills theory in three dimensions. The Lagrangian
\eqref{eq:ymlagrangian} is not only invariant under the SUSY transformations,
but it can also be written as a superderivative,
\begin{equation}
  \bar{\epsilon}\epsilon\mathcal{L}_{\mathrm{YM}}
  = \delta_{\bar{\epsilon}}\delta_{\epsilon}\mathrm{Tr}\left(\frac{1}{2}\bar{\lambda}\lambda
  - 2D\sigma\right).
\end{equation}
This will be important later on.
\subsection{Supersymmetric matter multiplets}
We will now add supersymmetric matter, i.e. a chiral multiplet $\Phi$ in
a representation $R$ of the gauge group. Its components are
\begin{equation}
  \Phi :\quad \phi,\bar{\phi},\psi,\bar{\psi}, F,\bar{F}.
\end{equation}
The supersymmetry transformations are
\begin{align}
  \delta\phi &= \bar{\epsilon}\psi,\nonumber\\
  \delta\bar{\psi} &= \epsilon\bar{\psi},\nonumber\\
  \delta\psi &= i\gamma^\mu\epsilon D_\mu\phi + i\epsilon\sigma\phi
  + \frac{2\Delta i}{3}\gamma^\mu D_\mu\epsilon\phi
  + \bar{\epsilon}F,\nonumber\\
  \delta{\bar{\psi}} &= i\gamma^\mu\bar{\epsilon} D_\mu \bar{\psi}
  + i\bar{\phi}\sigma\bar{\epsilon} + \frac{2\Delta i}{3}\bar{\phi}\gamma^\mu
  D_\mu\bar{\epsilon} + \bar{F}\epsilon,\nonumber\\
  \delta F &= \epsilon(i\gamma^\mu D_\mu \psi - i\sigma\psi - i\lambda\phi)
  + \frac{i}{3}(2\Delta - 1)D_\mu\epsilon\gamma^\mu\psi,\nonumber\\
  \delta\bar{F} &= \bar{\epsilon}(i\gamma^\mu D_\mu \bar{\psi}
  - i\bar{\psi}\sigma + i\bar{\psi}\bar{\lambda}) + \frac{i}{3}(2\Delta
  -1)D_\mu\bar{\epsilon}\gamma^\mu\bar{\psi}
\end{align}
where $\Delta$ is the possible anomalous dimension of $\phi$. For theories
with $\mathcal{N}\geq 3$ supersymmetry, the field has the canonical dimension
\begin{equation}
  \Delta = \frac{1}{2},
\end{equation}
but in general this is not the case.
\par The commutators of these transformations are given by
\begin{align}
  \left[\delta_\epsilon,\delta_\bar{\epsilon}\right]\phi
  &= i\nu^\mu\partial_\mu\phi + i\Lambda\phi + \Delta\rho\phi
  - \Delta\alpha\phi\\
  \left[\delta_\epsilon,\delta_\bar{\epsilon}\right]\bar{\phi}
  &= i\nu^\mu\partial_\mu\bar{\phi} - i\Lambda\bar{\phi} + \Delta\rho\bar{\phi}
  +\Delta\alpha\bar{\phi}\\
  \left[\delta_\epsilon,\delta_\bar{\epsilon}\right]\psi
  &= i\nu^\mu\partial_\mu\psi+\frac{1}{4}\Theta_{\mu\nu}\gamma^{\mu\nu}\psi
  + i\Lambda\psi + \left(\Delta + \frac{1}{2}\right)\rho\psi
  + (1-\Delta)\alpha\psi\\
  \left[\delta_\epsilon,\delta_\bar{\epsilon}\right]\bar{\psi}
  &= i\nu^\mu\partial_\mu\bar{\psi}
  + \frac{1}{4}\Theta_{\mu\nu}\psi^{\mu\nu}\bar{\psi} - i\bar{\psi}\Lambda
  + \left(\Delta + \frac{1}{2}\right)\rho\bar{\psi} + \left(\Delta
  -1\right)\alpha\bar{\psi},\\
  \left[\delta_\epsilon,\delta_\bar{\epsilon}\right]F &= i\nu^\mu\partial_\mu F + i\Lambda F + \left(\Delta + 1\right)\rho F + \left(2-\Delta\right)\alpha F,\\
      \left[\delta_\epsilon,\delta_\bar{\epsilon}\right]\bar{F}
      &= i\nu^\mu\partial_\mu\bar{F} - i\bar{F}\Lambda + \left(\Delta
  + 1\right)\rho\bar{F} + \left(\Delta -2\right)\alpha\bar{F}.
\end{align}
The lowest components of the superfields are  assigned the dimension $\Delta$
and $R$-charge $\mp\Delta$. The supersymmetry algebra closes off-shell
when the Killing spinors $\epsilon,\bar{\epsilon}$ satisfy
\eqref{eq:spinorrequirement} and $h$ is given by \eqref{eq:equationh}.
As a check, we compute
\begin{align}
  \left[\delta_\epsilon,\delta_\bar{\epsilon}\right]\phi &=\\
                                                         &=\bar{epsilon}\left(i\gamma^\mu\epsilon
  D_\mu\phi + i\epsilon\sigma\phi
+ \frac{2i\Delta}{3}\gamma^\mu(D_\mu\epsilon)\phi\right) = i\nu^\mu D_\mu\phi
+ i\sigma\bar{\epsilon}\epsilon
+ \frac{2i\Delta}{3}\left(\bar{\epsilon}\gamma^\mu D_\mu\epsilon\right),
\end{align}
which is the wished-for result.
\par Let us now consider supersymmetric Lagrangians for the matter
hypermultiplet. If the fields have their canonical dimensions, the
Lagrangian:
\begin{equation}
  \mathcal{L} = D_\mu\bar{\phi} D^\mu \phi - i\bar{\psi}\gamma^\mu D_\mu \psi
  + \frac{3}{4r^2}\bar{\phi}\phi + i\bar{\psi}\sigma\psi
  + i\bar{\psi}\lambda\phi - i\bar{\phi}\bar{\lambda}\psi + i\bar{\psi} D\phi
  + \bar{\phi}\sigma^2\phi + \bar{F}F
  \label{eq:hyperlagrangian}
\end{equation}
is invariant under supersymmetry if the Killing spinors
$\epsilon,\bar{\epsilon}$ satisfy \eqref{eq:spinorrequirement}, with $h$ given
as in \eqref{eq:equationh}. The quadratic part of the Lagrangian for $\phi$
gives indeed the standard conformal coupling for  a scalar field. We recall
that the action for a massless scalar field in a curved space of $n$ dimensions
contains a coupling to the curvature $R$ given by
\begin{equation}
  S = \int \mathrm{d}^n x\sqrt{g}(g^{\mu\nu}\partial_\mu\phi\partial_\nu\phi
  + \xi R\phi^2),
\end{equation}
where $\xi$ is a constant. This action is conformally invariant when
\begin{equation}
\xi = \frac{1}{4}\frac{n-2}{n-1}.
\end{equation}
If the  spacetime is an $n$-sphere of radius $r$, the curvature is
\begin{equation}
  R = \frac{n(n-1)}{r^2}
\end{equation}
and the conformal coupling of the scalar leads to an effective mass
term of the form
\begin{equation}
  \frac{n(n-2)}{4r^2}\phi^2
\end{equation}
which in $n=3$ dimensions gives the quadratic term for $\phi$ in
\eqref{eq:hyperlagrangian}. 
\par If the fields have non-canonical dimensions, the Lagrangian
\begin{multline}
  \mathcal{L_\mathrm{mat}} = D_\mu\bar{\phi} D^\mu\phi + \bar{\phi}\sigma^2\phi
  + \frac{i(2\Delta-1)}{r}\bar{\phi}\sigma\phi
  + \frac{\Delta(2-\Delta)}{r^2}\bar{\phi}\phi + i\bar{\phi}D\phi + \bar{F}F\\
  -i\bar{\psi}\gamma^\mu D_\mu \psi + i\bar{\psi}\sigma\psi - \frac{2\Delta
  - 1}{2r}\bar{\psi}\psi + i\bar{\psi}\lambda\phi
  - i\bar{\phi}\bar{\lambda}\psi
  \label{eq:noncanonicallagrangian}
\end{multline}
is supersymmetry, provided the parameters $\epsilon,\bar{\epsilon}$ satisfy the
Killing spinor  conditions \eqref{eq:spinorrequirement}. The Lagrangian in
\eqref{eq:noncanonicallagrangian} is not only invariant under the
supersymmetries $\delta_{\epsilon,\bar{\epsilon}}$ but it can be written as
a total superderivative,
\begin{equation}
  \bar{\epsilon}\epsilon\mathcal{L_\mathrm{mat}}
  = \delta_{\bar{\epsilon}}\delta_{\epsilon}\left(\bar{\psi}\psi
    - 2i\bar{\phi}\sigma\phi + \frac{2(\Delta -1)}{r}\bar{\phi}\phi\right).
  \end{equation}
\subsection{ABJM theory}
The theory proposed by Aharony, Bergman, Jafferis and Maldacena in \cite{Aharony08, Aharony08v2} to describe $N$ M2 branes
is a praticular example of a supersymmetric Chern-Simons theory. It consists of
two copies of Chern-Simons theory with gauge groups $U(N_1)$, $U(N_2)$, and
opposite levels $k, -k$. In addition, we have four matter supermultiplets
$\Phi_i$, $i=1,\dots,4$, in the bifundamental representation of the 
gauge group $U(N_1)\times U(N_2)$. This theory can be represented
as a quiver\footnote{A quiver is a directed graph where loops and multiple
  arrows between two vertices are allowed, i.e. a multidigraph. They are
  commonly used in representation theory: a representation $V$ of a
  quive assigns a vector space $V(x)$ to each vertex $x$ of the
quiver and a linear map $V(a)$ to each arrow $a$.}, with two nodes representing the Chern-Simons theories,
 and four edges between the nodes representing the matter supermultiplets, see
 the figure. In addition, there is a superpotential invloving the matter
 fields, which after integrating out the auxiliary fields in the
 Chern-Simons-matter system, reads (on $\mathbb{R}^3$.
 \begin{equation}
   W = \frac{4\pi}{k}\mathrm{Tr}\left(\Phi_1\Phi_2^\dagger\Phi_3\Phi_4^\dagger
   - \Phi_1\Phi_2^\dagger\Phi_3\Phi_4^\dagger\right),
 \end{equation}
 where we have used the standard superspace notation for $\mathcal{N}=1$
 supermultiplets.
 \section{A brief review of Chern-Simons theory}
 Since one crucial ingredient in the theories we are considering is
 Chern-Simons theory, we review here some results concerning the perturbative
 of structure of this theory on general three-manifolds. Chern-Simons theory on
 three-manifolds is an important subject in itself, hence we present a rather
 formal treatment with at least some pedagogical value.
 \subsection{Perturbative approach}
 In this section, we will denote the bosonic Chern-Simons action by
 \begin{equation}
   S = - \frac{1}{4\pi}\int_M\mathrm{Tr}(A\wedge\mathrm{d}A
   + \frac{2i}{3}A\wedge A\wedge A)
 \end{equation}
 where we use the conventions for Hermitian connections, and we included the
 factor $1/4\pi$ in the action for notational convenience. The group of gauge
 transformations $\mathcal{G}$ acts on the gauge connections as
 follows,
 \begin{equation}
   A\rightarrow A^U = UAU^{-1} - iU\mathrm{d}U^{-1}, \quad U\in\mathcal{G}.
 \end{equation}
 We will assume that the theory is defined on a compact three-manifold $M$. The
 partition function is defined as 
 \begin{equation}
   Z(M)
   = \frac{1}{\mathrm{vol}(\mathcal{G})}\int\left[\mathcal{D}A\right]e^{ikS}
   \label{eq:partitionfunctionchernsimons}
 \end{equation}
 where we recall that $k\in\mathbb{Z}$.
 \par There are many different approaches to the calculation of
 \eqref{eq:partitionfunctionchernsimons}, but the obvious strategy is to use
 perturbation theory. Notice that, since the theory is defined on a compact
 manifold, there are no IR divergences and we just have to deal with UV
 divergences, as in standard QFT. Once these are treated appropriately, the
 partition function \eqref{eq:partitionfunctionchernsimons} is a well-defined
 observable. In perturbation theory we evaluate
 \eqref{eq:partitionfunctionchernsimons} by expanding around saddle points.
 These are flat connections, which are in one-to-one correspondance with group
 homomorphisms
 \begin{equation}
   \pi_1(M)\rightarrow G
 \end{equation}
 modulo conjugation. For example, if $M = \mathbb{S}^3/\mathbb{Z}_p$ is the
 lens space $L(p,1)$, one has $\pi_1(L(p,1))=\mathbb{Z}_p$, and flat
 connections are labelled by homomorphisms $\mathbb{Z}_p \rightarrow G$. Let us
 assume that thesea rare a discrete set of points( this happens, for example.
 if $M$] is a rotational homology sphere, since in that case $\pi_1(M)$ is
 a finite group). We weill label the falt connnections with an index $c$, and
 a flat connection will be denoted by $A^{(c)}$. Each flat connection
 leads to a covariant derivative:
 \begin{equation}
   \mathrm{d}_{A^{(c)}} = \mathrm{d} + i[A^(c),\cdot],
   \label{eq:covariantderivative}
 \end{equation}
 and flatness implies that
 \begin{equation}
   \mathrm{d}^2_{A^{(c)}} = iF_{A^{(c)}} = 0.
 \end{equation}
 Therefore the covariant derivative leads to a complex
 \begin{equation}
   0\rightarrow\Omega^{0}(M,\mathbf{g})\xrightarrow{\mathrm{d}_{A^{(c)}}}\Omega^{1}(M,\mathbf{g})\xrightarrow{\mathrm{d}_{A^{(c)}}}\Omega^{3}(M,\mathbf{g}).
   \label{eq:covariantderivativecomplex}
 \end{equation}
 The first two terms in this complex have a natural interpretation
 in the context of gauge theories: $\Omega^{0}(M,\mathbf{g})$ is the Lie
 algebraalgebra of the  group of gauge transformations, and we can
 write a gauge transformation as 
 \begin{equation}
   U = e^{i\phi}\quad \phi\in\Omega^{0}(M,\mathbf{g})
 \end{equation}
 The elements of $\Omega^{0}(M,\mathbf{g})$ generate infinitesimal
 gauge transformations,
 \begin{equation}
   \delta A = - \mathrm{d}_A\phi
 \end{equation}
 The second term, $\Omega^{1}(M,\mathbf{g})$, can be identifeid with the
 tangent space to the space of gauge connections. The first map in the complex
 \eqref{eq:covariantderivativecomplex} is interpreted as (minus) an
 infinitesimal gauge transformation in the background of $A^{(c)}$.
 \par One can recall that the space of $\mathbf{g}$ valued forms on $M$ has
 a natural inner product given by
 \begin{equation}
   \langle a,b \rangle = \int_M\mathrm{Tr}(a\wedge \ast b),
 \end{equation}
where $\ast$ is the Hodge operator. With respect to this product, we can define
an adjoint operator on $\mathbf{g}$-valued $p$-forms in the same
way that is done for the usual de Rham operator,
\begin{equation}
  \mathrm{d}^\dagger_{A^{(c)}} = (-1)^{3(1+p)+1}(\ast \mathrm{d}_{A^{(c)}})\ast .
\end{equation}
We then have the orthogonal decompositions
\begin{gather}
  \Omega^{0}(M,\mathbf{g})
  = \mathrm{Ker}\:\mathrm{d}_{A^{(c)}}\oplus\mathrm{Im}\:\mathrm{d}^\dagger_{A^{(c)}}\nonumber\\
  \Omega^{1}(M,\mathbf{g})
  = \mathrm{Ker}\:\mathrm{d}^\dagger_{A^{(c)}}\oplus\mathrm{Im}\:\mathrm{d}_{A^{(c)}}
  \label{eq:orthogonaldecomposition}
\end{gather}
These decompositions are easily proved. For the first one, for example, we just
have to note that
\begin{equation}
  a\in\mathrm{Ker}\: \mathrm{d}_{A^{(c)}}\Rightarrow \langle
  \mathrm{d}_{A^{(c)}}a,\phi\rangle = \langle a,
  \mathrm{d}_{A^{(c)}}^\dagger\phi\rangle = 0,\quad \forall\phi
\end{equation}
therefore
\begin{equation}
  (\mathrm{Ker}\:\mathrm{d}_{A^{(c)}})^\bot
= \mathrm{Im}\:\mathrm{d}^\dagger_{A^{(c)}}.
\end{equation}
One also has the analogue of the Laplace-Beltrami operator acting on
$p$-forms
\begin{equation}
  \Delta^p_{A^{(c)}} = \mathrm{d}_{A^{(c)}}^\dagger\mathrm{d}_{A^{(c)}}
      + \mathrm{d}_{A^{(c)}}\mathrm{d}_{A^{(c)}}^\dagger .
    \end{equation}
In the following we will assume that
\begin{equation}
  H^{1}(M,\mathbf{g}) = 0.
\end{equation}
This means that the conection $A^{(c)}$ is \textit{isolated}. However we will
consider the possibility that $A^{(c)}$ has a non-trivial isotropy group
$\mathcal{H}_c$. We recall that the isotropy group of a connection  $A^{(c)}$
is the subgroup of gauge transformations which leave $A^{(c)}$ invariant,
\begin{equation}
  \mathcal{H}_c = \left\{\phi \in \mathcal{G}\:\big\vert\phi(A^{(c)}) =A^{(c)}\right\}.
\end{equation}
The Lie algebra of this group is given by zero-forms annihilated
by the covariant derivative \eqref{eq:covariantderivative},
\begin{equation}
  \mathrm{Lie}(\mathcal{H}_c) = H^{0}(M,\mathbf{g})
  = \mathrm{Ker}\:\mathrm{d}_{A^{(c)}},
  \label{eq:liealgebra}
\end{equation}
which is in general non-trivial. A  connection is \textit{irreducible} if its
isotopy group is equal to the center of the group. In particular, if
$A^{(c)}$ is irreducible one has
\begin{equation}
  H^0(M,\mathbf{g})=0
\end{equation}
It can be shown that the isotropy group $\mathcal{H}_c$ consists 
of constant gauge translations that leave $A^{(c)}$ invariant,
\begin{equation}
  \phi A^{(c)}\phi^{-1} = A^{(c)}.
\end{equation}
They are in one-to-one correspondence with a subgroup of $G$ which
we will denote by $H_c$.
\par In the semiclassical approximation, $Z(M)$ is written as a sum
of terms associated to saddle-points:
\begin{equation}
  Z(M) = \sum_c Z^{(c)}(M),
\end{equation}
where $c$ labels the different flat connections $A^{(c)}$ on $M$. Each of the
$Z^{(c)}(M)$ will be a perturbative series in $1/k$ of the form
\begin{equation}
  Z^{(c)}(M)
  =  Z^{(c)}_{\mathrm{1-loop}}(M)\exp\left\{\sum_{l=1}^{\infty}S_l^{(c)}k^{-l}\right\},
\end{equation}
where $S_l^{(c)}$ is the $(l+1)$-loop contribution around the 
flat connection $A^{(c)}$. In order to derive this expansion, we
split the connection into a "background", which is the flat connection
$A^{(c)}$, plus a "fluctuation" $B$:
\begin{equation}
  A = A^{(c)}+B
\end{equation}
Expanding around this, we find
\begin{equation}
  S(A) = S(A^{(c)})+S(B),
  \label{eq:expansionloop}
\end{equation}
where 
\begin{equation}
  S(B) = -\frac{1}{4}\int_M \mathrm{Tr}(B\wedge\mathrm{d}_{A^{(c)}}B
  + \frac{2i}{3}B^3).
\end{equation}
The first term in \eqref{eq:expansionloop} is the classical
Chern-Simons invariant of the connection $A^{(c)}$. Since Chern-Simonstheory is
a gauge theory, in order to proceed we have to fix the gauge. We will
follow the detailed analysis of \cite{Adams97}. Our gauge choice
will be the standard, covariant gauge,
\begin{equation}
  g_{A^{(c)}}(B) = \mathrm{d}^\dagger_{A^{(c)}}B=0
\end{equation}
where $g_{A^{(c)}}$ is the gauge fixing function. We recall that in the
standard Fadeev-Popov (FP) gauge fixing one first defines
\begin{equation}
  \Delta^{-1}_{A^{(c)}}(B) = \int\mathcal{D}U\delta(g_{A^{(c)}}(B^U)),
  \label{eq:fpgaugefixing}
\end{equation}
and then inserts into the path integral
\begin{equation}
1 = \left[\int\mathcal{D}U\delta(g_{A^{(c)}}(B^U))\right]\Delta_{A^{(c)}}(B).
\end{equation}
The key new ingredient here is the presence of a  non-trivial isotropy
group $\mathcal{H}_c$ for the flat connection $A^{(c)}$. When there is
a non-trivial isotropy group, the gauge-fixing condition does not fix
the gauge completely, since
\begin{equation}
  g_{A^{(c)}}(B^\phi) = \phi g_{A^{(c)}}(B)\phi^{-1},\quad
  \phi\in\mathcal{H}_c,
  \label{eq:fakegaugefixing}
\end{equation}
i.e. the basic assumption that $g(A)=0$ only cuts the gauge orbit
once is not true, and there is a residual symmetry by the isotropy
group. Another way to see this is that the standard FP determinant
vanishes due to zero modes. In fact, the standard calculation of
\eqref{eq:fpgaugefixing} (which is valid if the standard isotropy
group of $A^{(c)}$ is trivial) gives
\begin{equation}
  \Delta^{-1}_{A^{(c)}}(B) = \abs{\mathrm{det}\frac{\delta
      g_{A^{(c)}}(B^U)}{\delta U}}^{-1}
      = \abs{\mathrm{det}\;\mathrm{d}^\dagger_{A^{(c)}}\mathrm{d}_A}^{-1}.
      \label{eq:fpdeterminant}
    \end{equation}
Note that when $\mathcal{H}_c\neq 0$, the operator $\mathrm{d}_{A^{(c)}}$ has
zero modes due to the nonvanishing of \eqref{eq:liealgebra}, and the FP
procedure is ill-defined. The correct way to proceed in the calculation of
\eqref{eq:fpgaugefixing} is to split the integration over the
gauge group into two pieces. The first piece is the integration
over the isotropy group. Due to \eqref{eq:fakegaugefixing}, the integrand does
not depend on it, and we obtain a factor of $\mathrm{Vol}(\mathcal{H}_c)$. The
second piece gives an integration over the remaining part of the gauge
transformations, which has as its Lie algebra
\begin{equation}
  \left(\mathrm{Ker}\;\mathrm{d}_{A^{(c)}}\right)^\bot .
\end{equation}
The integration over this piece leads to the standard FP determinant
\eqref{eq:fpdeterminant} but with the zero modes removed. We then
find,
\begin{equation}
  \Delta^{-1}_{A^{(c)}}(B)
  = \mathrm{Vol}(\mathcal{H_c})\abs{\det\diff^\dagger_{A^{(c)}}\diff_A}^{-1}_{(\mathrm{Ker}\;\diff_{A^{(c)}})^\bot}
\end{equation}
This phenomenon was first observed by Rozansky in \cite{Rozansky94}, and
developed in this language in \cite{Adams97}. As usual, the determinant
appearing here can be written as a path integral over ghost fields,
with action
\begin{equation}
  S_{\mathrm{ghosts}}(C,\bar{C},B) = \langle
  \bar{C},\diff^\dagger_{A^{(c)}}\diff_A C\rangle,
\end{equation}
where $C,\bar{C}$ are Grassmannian fields taking values in
\begin{equation}
  \left(\mathrm{Ker}\diff_{A^{(c)}}\right)^\bot .
\end{equation}
The action for the ghosts can be divided into a kinetic term plus an
interaction term between the ghost fields and the fluctuation $B$:
\begin{equation}
  S_{\mathrm{ghosts}}(C,\bar{C},B) = \langle\bar{C},\Delta^0_{A^{(c)}}C\rangle
  + i\langle\bar{C},\diff^\dagger_{A^{(c)}}\left[B,C\right]\rangle .
\end{equation}
The modified FP gauge-fixing leads then to the path integral
\begin{multline}
  Z^{(c)}(M) = e^{ikS(A^{(c)})}\int_{\Omega^1(M,\mathbf{g})}\mathcal{D}B
  e^{ikS(B)}\Delta_{A^{(c)}}(B)\delta\left(\diff^\dagger_{A^{(c)}}B\right)=\\
  = \frac{e^{ikS(A^{(c)})}}{\mathrm{Vol}(\mathcal{H}_c)}\int_{\Omega^1(M,\mathbf{g})}\mathcal{D}B\delta(\diff^\dagger_{A^{(c)}}B)\int_{(\mathrm{Ker}\;\diff_{A^{(c)}})^\bot}\mathcal{D}C\mathcal{D}\bar{C}e^{ikS(B)-S_{\mathrm{ghosts}}(C,\bar{C},B)}.
\end{multline}
Finally, we analyze the delta constraint on $B$. Due to the decomposition of
$\Omega^1(M,\mathbf{g})$ in \eqref{eq:orthogonaldecomposition}, we can write
\begin{equation}
  B = \diff_{A^{(c)}}\phi + B',
  \label{eq:bfielddecompose}
\end{equation}
where
\begin{equation}
  \phi\in\left(\mathrm{Ker}\;\diff_{A^{(c)}}\right)^\bot, \quad
  B'\in\mathrm{Ker}\;\diff^\dagger_{A^{(c)}}.
\end{equation}
The presence of the operator $\diff_{A^{(c)}}$ in the change of variables
\eqref{eq:bfielddecompose} leads to a non-trivial Jacobian. Indeed,
we have
\begin{equation}
  \abs{\abs{B}}^2 = \langle\phi,\Delta^0_{A^{(c)}}\phi\rangle
  + \abs{\abs{B'}}^2
\end{equation}
and the measure in the functional integral becomes
\begin{equation}
  \mathcal{D}B
  = \left({\det} ' \Delta^0_{A^{(c)}}\right)^{\frac{1}{2}}\mathcal{D}\phi\mathcal{D}B',
  \label{eq:chernsimonsjacobian}
\end{equation}
where the $'$ indicates, as usual, that we are removing the  zero nodes. Notice
that the operator in the last equation is positive-definite, so the square root
of its determinant is well-defined. We also have that
\begin{equation}
  \delta\left(\diff_{A^{(c)}}^\dagger B\right)
  = \delta\left(\Delta^0_{A^{(c)}}\phi\right)
  = \left({\det}'{\Delta^0_{A^{(c)}}}^{-1}\right)\delta(\phi),
\end{equation}
which is a straighforward generalization of the standard formula
\begin{equation}
  \delta(ax) = \frac{1}{\abs{a}}\delta(x).
\end{equation}
We conclude that the delta function, together with the Jacobian in
\eqref{eq:chernsimonsjacobian}, lead to the following factor in
the path integral:
\begin{equation}
  ({\det}'\Delta^0_{A^{(c)}})^{-\frac{1}{2}}.
\end{equation}
In addition, the delta function sets $\phi=0$. The only thing that remains is
the integration over $B'$ which we relabel $B'\rightarrow B$. The  final result
for the gauge-fixed path integral is then
\begin{equation}
  Z^{(c)}(M)
  = \frac{e^{ikS(A^{(c)})}}{\mathrm{Vol}(\mathcal{H}_c)}\left({\det}'\Delta^0_{A^{(c)}}\right)^{-\frac{1}{2}}\int_{(\mathrm{Ker}\;\diff^\dagger_{A^{(c)}})^\bot}\mathcal{D}C\mathcal{D}\bar{C}e^{ikS(B)-S_{\mathrm{ghosts}}(C,\bar{C},B)}
\end{equation}
This is the starting point to perform gauge-fixed perturbation theory in:
Chern-Simons theory. 
\subsection{The one-loop contribution}
We now consider the one-loop contribution of a saddle-point to the path
integral. This has been studied in many papers [41,58,87,88]. We will follow
the detailed presentation in [2]. Before proceeding, we should specify what is
the regularization mathod that we will use to define the functional determinants
appearing in our calculation. A natural and useful regularization for quantum
field theories in curved space is zeta-functional regularization. We recall
that the zeta function of a self-adjoint operator $T$ with eigenvalues
$\lambda_n>0$ is defined as
\begin{equation}
  \zeta_T(s) = \sum_n{\lambda_n^{-s}.}
\end{equation}
Under appropriate conditions, this defines a meromorphic\footnote{A meromorphic
  function on an open subset $D$ of the complex plane (or a higher dimensional
  complex disk) is a function that is holomorhic on all of $D$ except for
a discrete set of isoleted poles.} function on the complex $s-$plane which is
regular at $s=0$. Since
\begin{equation}
  =\zeta'_T(0) = \sum_n{\log{\lambda_n}}
\end{equation}
