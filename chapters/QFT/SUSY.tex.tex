\chapterauthor{Ivo Iliev}{Sofia University}
%\chapterauthor{Second Author}{Second Author Affiliation}
\chapter{Supersymmetry}
\section{Supermultiplets}

\begin{definition}[Supermultiplet]
  Representations of the supersymmetric algebra (superalgebra) are called
  supermultiplets.
\end{definition}

Indeed, these representations can be thought of as multiplets where we assemble
together several different representations of the Lorentz algebra, since the
latter is a subalgebra of the superalgebra.

\subsection{Massless supermultiplets}
If $P^2=0$, then we can take $P_\mu$ to a canonical form by applying boost and
rotations until it reads

\begin{equation}
  \sigma^{\mu}_{\alpha\dot{\alpha}}P_\mu = \left(\sigma^0+\sigma^3\right)E = 
  \begin{bmatrix}
    0 & 0\\
    0 & 2E
  \end{bmatrix}
\end{equation}

The supersymmetric algebra becomes,
\begin{equation}
  \begin{bmatrix}
    \{Q_1,\bar{Q}_{\dot{1}}\} & \{Q_1,\bar{Q}_{\dot{2}}\} \\

    \{Q_2,\bar{Q}_{\dot{1}}\} & \{Q_2,\bar{Q}_{\dot{2}}\} \\

  \end{bmatrix}
  =
  \begin{bmatrix}
    0 & 0\\
    0 & 4E
  \end{bmatrix}
\end{equation}

intended as acting on the states of the multiplet we are looking for. In
particular,
\begin{equation}
\{Q_1, \bar{Q}_{\dot{1}}\} = 0
\end{equation}
which implies that 
\begin{equation}
  ||Q_1|\omega\rangle||^2 = 0 = ||\bar{Q}_{\dot{1}}|\omega\rangle||^2
\end{equation}
and thus

\begin{equation}
  Q_1|\omega\rangle = 0 = \bar{Q}_{\dot{1}}|\omega\rangle.
\end{equation}
This means that as operators $Q_1$ and $\bar{Q}_{\dot{1}}$ annihilate the
multiplet. 
\par The only nontrivial anticommutation relation that is left is:
\begin{equation}
  \{Q_2,\bar{Q}_{\dot{2}}\} = 1
\end{equation}
If we call
\begin{equation}
  \alpha = \frac{1}{2\sqrt{E}}Q_2,\quad \alpha^\dagger
  = \frac{1}{2E}\bar{Q}_{\dot{2}}
\end{equation}
then the anticommutation relation that is left is:
\begin{equation}
  \{\alpha,\alpha^\dagger\} = 1
\end{equation}
with $\{\alpha,\alpha\} = 0$.

We can build the representation starting from a state $|\lambda\rangle$ such
that
\begin{equation}
  \alpha|\lambda\rangle = 0
\end{equation}
Lets suppose that it has \textit{helicity} $\lambda$:
\begin{equation}
  M_{12}\lambda \equiv J_3|\lambda\rangle = \lambda|\lambda\rangle.
\end{equation}
It is easy to compute the helicity of $\alpha^{\dagger}|\lambda\rangle$:
\begin{equation}
  M_{12} \bar{Q}_{\dot{2}}|\lambda\rangle
  = \left(\bar{Q}_{\dot{2}}M_{12}+\frac{1}{2}\bar{Q}_{\dot{2}}\right)|\lambda\rangle
  = (\lambda+\frac{1}{2}\bar{Q}_{\dot{2}})|\lambda\rangle 
\end{equation}
In the last line, we have used the fact that
$\left[M_{12},\bar{Q}_{\dot{2}}\right]=\frac{1}{2}\bar{Q}_{\dot{2}}$. Thus we
found out that
\begin{equation}
  \alpha^\dagger|\lambda\rangle  = |\lambda+ \frac{1}{2}\rangle 
\end{equation}
Since $(\alpha^\dagger)^2 = 0$, this stops here. Hence we have
\begin{equation}
  \alpha^\dagger|\lambda+\frac{1}{2}\rangle = 0.
\end{equation}
\par
Massless multiplets are thus composed of one boson and one fermion. Since
physical particles must come in CPT conjugate representation (or, there are no
spin-$\frac{1}{2}$ one dimensional representations of the massless little group
of the Lorentz group), one must add the CPT conjugate multiplet where
helicities are flipped.
\begin{example}[Examples of massless supermultiplets]
  \mbox{}
  \begin{itemize}
    \item The \textit{scalar} multiplet is obtained by setting $\lambda =0$.
      Then we have 
      \begin{equation}
        \alpha^\dagger\ket{0} = \ket{\frac{1}{2}}
      \end{equation}
      The full multiplet is composed of two states with $\lambda = 0 $ and
      a doublet with $\lambda = \pm\frac{1}{2}$. These are the degrees of
      freedom of a complex scalar and a Weyl (chiral) fermion.
    \item The \textit{vector} multiplet is obtained starting from a $\lambda
      = \frac{1}{2}$ state. We get
      \begin{equation}
        \alpha^\dagger\ket{\frac{1}{2}} = \ket{1}.
      \end{equation}
      To this we add the CPT conjugate multiplet, to obtain two pairs of
      states, one with $\lambda = \pm\frac{1}{2}$ and the other with $\lambda
      = \pm 1$. These are the degrees of freedom of a Weyl fermion and of
      a massless vector. The latter is usually interpreted as a gauge boson.
    \item Another multiplet is obtained starting from $\lambda = \frac{3}{2}$:
      \begin{equation}
        \alpha^\dagger\ket{\frac{3}{2}} = \ket{2}.
      \end{equation}
      Adding the CPT conjugate, one has a pair of bosonic degrees of freedom
      with $\lambda = \pm 2$, which we interpret as the \textit{graviton}, and
      a pair of fermionic degrees of freedom with $\lambda =\pm\frac{3}{2}$,
      which correspond to a massless spin-$\frac{3}{2}$ Rarita-Schwinger field,
      also called the \textit{gravition}, since it is the SUSY partner of the
      graviton, as was just shown.
  \end{itemize}
\end{example}
\subsection{Supermultiplets of extended supersymmetry}
Very briefly we will mention that having extended SUSY, the massless
supermultiplets are longer. Let's take the algebra to be:
\begin{equation}
  \{Q_\alpha^I,\bar{Q}_{\dot{\alpha}}^J\} = 2\sigma_{\alpha\dot{\alpha}}^\mu
  P_\mu\delta^{IJ},
\end{equation}
where for simplicity we suppose that $Z^{IJ}=0$ for these states. For massless
states, $P_\mu = (E,0,0,E)$ and therefore as before we have that
\begin{equation}
  \{Q_1^I,\bar{Q}_{\dot{1}}^J\} = 0,
\end{equation}
which implies the (operator) equations $Q_1^I=0$ and $\bar{Q}_{\dot{1}}^I=0$,
for $I=1,\cdots,\sn$. The nontrivial relations are then:
\begin{equation}
\{Q_2^I, \bar{Q}_{\dot{2}}^J\} = 4E\delta^{IJ}
\end{equation}
Of course we can define
\begin{equation}
  \alpha_I = \frac{1}{2\sqrt{E}}Q_2^I
\end{equation}
and obtain the canonical anticommutation relations for $\sn$ fermionic
oscillators
\begin{equation}
  \{\alpha_I,\alpha_J^\dagger\} = \delta_{IJ}
\end{equation}
\par If we now start with a state $\ket{\lambda}$ with helicity $\lambda$ which
satisfies $\alpha_I\ket{\lambda} = 0$, we build a multiplet as follows:
\begin{gather}
  \alpha_I^\dagger\ket{\lambda} = \ket{\lambda + \frac{1}{2}}_I,\nonumber\\
  \alpha_I\dagger\alpha_J\dagger\ket{\lambda} = \ket{\lambda
  + 1}_{[IJ]},\nonumber\\
  \vdots\nonumber\\
  \alpha_1^\dagger\cdots\alpha_{\sn}^\dagger\ket{\lambda}
  = \ket{\lambda+\frac{\sn}{2}}
\end{gather}
It is very important to note that there are $\sn$ states with helicity
$\lambda + \frac{1}{2}, \frac{1}{2}\sn(\sn-1)$ states with
helicity $\lambda + 1$ and so on, until we reach a single state with helicity $\lambda
+\frac{\sn}{2}$ (it is totally antisymmetric in $\sn$ indices
I). In total, the supermultiplet is composed of $2^\sn$ states, half of
them bosonic and half of them fermionic.
\par Interestingly, in this case we can now have self-CPT conjugate multiplets.
Take for example $\sn = 4$ and start from $\lambda = -1$. Then $\lambda
+ \frac{\sn}{2} = 1$ and the multiplet spans states of opposite helicities,
thus filling complete representations of the Lorenz group. Indeed, it contains
one pair of states with $\lambda\pm1$ (a vector, i.e a gauge boson), 4 pairs of
states with $\lambda=\pm\frac{1}{3}$ (4 Weyl Fermions) and 6 states with
$\lambda = 0$ (6 real scalars, or equivalently 3 complex scalars).
\par Another example is $\sn = 8$ supersymmetry. Here if we start with $\lambda
= -2$ we end up with $\lambda + \frac{\sn}{2} = 2$. Thus in this case we have
the graviton in the self-CPT conjugate multiplet, corresponding to the pair of
states with $\lambda\pm2$. In addition, we have 8 massless gravitini with
$\lambda\pm\frac{3}{2}$, 28 massless vectors with $\lambda = \pm 1$, 56
massless Weyl fermions with $\lambda = \pm\frac{1}{2}$ and finally 70 real
scalars with $\lambda = 0$. This is the content of $\sn = 8$
\textit{supergravity}, which is the only multiplet of $\sn = 8$ supersymmetry
with $|\lambda|<2$. The latter condition is necessary in order to have
consistent couplings (higher spin fields cannot be coupled in a consistent way
with gravity and lower spin fields).
\par From the theoretical standpoint, this is a very nice result, because we
have a theory where \textit{everything is determined} from symmetry alone: the
complete spectrum and all the couplings. Unfortunately, this theory is also
completely unphysical. To mention one problem, it has no room for fermions in
complex representations of the gauge group, which are present in the Standard
Model.

\subsection{Massive supermultiplets}
When $P^2=M^2>0$, by boosts and rotation $P_\mu$ can be put in the following
form
\begin{equation}
  P_\mu = (M,0,0,0)
\end{equation}
Then we have
\begin{equation}
  \sigma_{\alpha\dot{\alpha}}^\mu P_\mu = M\sigma^0 = 
  \begin{bmatrix} 
    M & 0\\
    0 & M
  \end{bmatrix}
\end{equation}
so that the superalgebra reads
\begin{equation}
  \{Q_\alpha, \bar{Q}_{\dot{\alpha}}\} = 2 M\delta_{\alpha\dot{\alpha}}
\end{equation}
Note that $[M_{12}, Q_1] = i(\sigma_{12})_1\ ^1Q_1= \frac{1}{2}Q_1$, thus it is
$Q_1$ that raises the helicity, in the same way as $\bar{Q}_{\dot{2}}$. We make
the redefinition
\begin{gather}
  \alpha_1 = \frac{1}{\sqrt{2M}}\bar{Q}_{\dot{1}},\quad \alpha_1^\dagger
  = \frac{1}{\sqrt{2M}}Q_1, \\
  \alpha_2 = \frac{1}{\sqrt{2M}}Q_2,\quad \alpha_2^\dagger
  = \frac{1}{\sqrt{2M}}\bar{Q}_\dot{2},
\end{gather}
so that we have the canonical anticommutation relations of two fermionic
oscillators:
\begin{equation}
  \{\alpha_a,\alpha_b^\dagger\}=\delta_{ab},\quad a,b = 1,2.
\end{equation}
\par If we start with $\alpha_a\ket{\lambda} = 0$, $M_{12}\ket{\lambda}
= \lambda\ket{\lambda}$, then we build the multiplet as:
\begin{gather}
  \alpha_1^\dagger\ket{\lambda} = \ket{\lambda + \frac{1}{2}}_1,\\
  \alpha_2^\dagger\ket{\lambda} = \ket{\lambda+\frac{1}{2}}_2,\\
  \alpha_1^\dagger\alpha_2^\dagger\ket{\lambda} = \ket{\lambda+1}.
\end{gather}
There are 4 states now (compared to the 2 in the massless case), two bosons and
two fermions. 
\begin{example}[Examples of massive supermultiplets]
  \mbox{}
  \begin{itemize}
    \item In the case of the \textit{massive scalar multiplet}, we start from
      $\lambda =  -\frac{1}{2}$ and obtain two states with $\lambda=0$ and one
      state with $\lambda=\frac{1}{2}$. These are the degrees of freedom of one
      massive complex scalar and one massive Weyl fermion. Note that the latter
      might not be familiar. Indeed, one cannot write the usual Dirac mass term
      for a Weyl fermion. Instead, one can write what is called a Majorana mass
      term:
      \begin{equation}
        \mathcal{L}\supset m\epsilon^{\alpha\beta}\psi_\alpha\psi_\beta + h.c.
      \end{equation}
      Note that the total degrees of freedom of a massless scalar multiplet is
      the same as that of a massive one.
    \item For a \textit{massive vector multiplet}, start from $\lambda = 0$ to
      obtain 2 states with $\lambda = \frac{1}{2}$ and one state with $\lambda
      = 1$. To this we add the CPT conjugate multiplet so that in the end we
      have one pair with $\lambda = \pm1$, two pairs with
      $\lambda=\pm\frac{1}{2}$ and two states with $\lambda = 0$. According to
      the massive little group, this corresponds to 1 massive vector (with
      $\lambda = \pm1,0)$, 1 real scalar and 1 massive Dirac fermion. Note
      however that the content in degrees of freedom is the same as that of one
      massless vector multiplet together with one massless scalar multiplet.
      This hints that the consistent way to treat massive vectors in
      a supersymmetric field theory will be through a SUSY version of the
      Brout-Englert-Higgs mechanism.
    \end{itemize}
\section{General}
\begin{definition}[R-symmetry]
In supersymmetric theories, an R-symmetry is the symmetry transforming different supercharges into each other. In the simplest case of the $\sn=1$ supersymmetry, such R-symmetry is isomorphic to a global $U(1)$ group or it's discrete subgroup. For extended supersymmetry theories, the R-symmetry group becomes a global non-abelian group.  
\end{definition}

\begin{remark}
In the case of the discrete subgroup $\mathbb{Z}_2$, the R-symmetry is called \textit{R-parity}
\end{remark}

\begin{definition}[Extended supersymmetry]
In supersymmetric theories, when $\sn>1$ the algebra is said to have
\textit{extended supersymmetry}.
\end{definition}

\section{Bogomol'nyi-Prasad-Sommerfield (BPS) states}
\begin{definition}[BPS state]
A massive representation of an extended supersymmetry algebra that has mass
equal to the supersymmetry central charge $Z$ is called an \textit{BPS} state.
\end{definition}

Quantum mechanically speaking, if the supersymmetry remains unbroken, exact
solutions to the modulus of $Z$ exist. Their importance arises as the
multiplets shorten for generic representations, with stability and mass formula
exact.

\begin{example}[$d=4$,$\sn=2$]
The generators for the odd part of the superalgebra have relations:
\begin{gather}
  \{Q_\alpha^A,\bar{Q}_{\dot{\beta}B}\} = 2\sigma^m_{\alpha\dot{\beta}}P_m\delta^A_B \\
  \{Q_\alpha^A,Q_{\beta B}\} = 2\epsilon_{\alpha\beta}\epsilon^{AB}\bar{Z}\\
  \{\bar{Q}_{\dot{\alpha}A},\bar{Q}_{\dot{\beta}B}\}
  = -2\epsilon_{\dot{\alpha}\dot{\beta}}\epsilon_{AB}Z,
\end{gather}
where $\alpha\dot{\beta}$ are the Lorentz group indices and $A,B$ are the
$R$-symmerty indices.
If we take linear combinations of the above generators as follows:
\begin{gather}
  R_\alpha^A = \xi^{-1}Q_\alpha^A
  + \xi\sigma^0_{\alpha\dot{\beta}}\bar{Q}^{\dot{\beta}B}\\
  T_\alpha^A = \xi^{-1}Q_\alpha^A - \xi\sigma^0_{\alpha\dot{\beta}}\bar{Q}^{\dot{\beta}B}
\end{gather}
and consider a state $\psi$ which has momentum $(M,0,0,0)$, we have:
\begin{equation}
  \left(R_1^1+(R_1^1)^\dagger\right)^2\psi = 4(M+Re(Z\xi^2))\psi,
\end{equation}
but because this is the square of a Hermitian operator, the right hand side
coefficient must be positive for all $\xi$. In particular, the strongest result from this is
\begin{equation}
  M\geq|Z|
\end{equation}
\end{example}
%%%\begin{table}
%%%    \tabletitle{Examples for illustrating attacks}
%%%    \begin{tabular}{|c|c|c|c|}
%%%        \hline
%%%        \textbf{Job} & \textbf{Sex} & \textbf{Age} & \textbf{Disease} \\
%%%        \hline
%%%        Engineer & Male & 35 & Hepatitis \\
%%%        Engineer & Male & 38 & Hepatitis \\
%%%        Lawyer & Male & 38 & HIV \\
%%%        Writer & Female & 30 & Flu \\
%%%        Writer & Female & 30 & HIV \\
%%%        Dancer & Female & 30 & HIV \\
%%%        Dancer & Female & 30 & HIV \\
%%%        \hline
%%%    \end{tabular}
%%%    \label{table:rawpatient}
%%%\end{table}
\section{Supersymmetric theories on curved manifolds}
\textit{Remark:} Supersymmetric theories may be defined only on backgrounds admitting solutions to certain Killing spinor equations,
\begin{align}
\left(\nabla_\mu-iA_\mu\right)\zeta + iV_\mu\zeta + iV^\nu\sigma_{\mu\nu}\zeta = 0\\
\left(\nabla_\mu+iA_\mu\right)\tilde{\zeta} - iV_\mu\tilde{\zeta} 0 iV^\nu\tilde{\sigma}_{\mu\nu}\tilde{\zeta} = 0
\end{align}
which in four dimensions and Euclidean signature are
equivalent to the requirement that the manifold is complex and the metric Hermitian.

%\begin{Glossary}

%\end{Glossary}
