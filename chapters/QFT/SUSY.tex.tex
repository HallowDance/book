\chapterauthor{Ivo Iliev}{Sofia University}
%\chapterauthor{Second Author}{Second Author Affiliation}
\chapter{Supersymmetry}
\section{Supermultiplets}

\begin{definition}[Supermultiplet]
  Representations of the supersymmetric algebra (superalgebra) are called
  supermultiplets.
\end{definition}

Indeed, these representations can be thought of as multiplets where we assemble
together several different representations of the Lorentz algebra, since the
latter is a subalgebra of the superalgebra.

\subsection{Massless supermultiplets}
If $P^2=0$, then we can take $P_\mu$ to a canonical form by applying boost and
rotations until it reads

\begin{equation}
  \sigma^{\mu}_{\alpha\dot{\alpha}}P_\mu = \left(\sigma^0+\sigma^3\right)E = 
  \begin{bmatrix}
    0 & 0\\
    0 & 2E
  \end{bmatrix}
\end{equation}

The supersymmetric algebra becomes 
\begin{equation}
  \begin{bmatrix}
    \{Q_1,\bar{Q}_{\dot{1}}\} & \{Q_1,\bar{Q}_{\dot{2}}\} \\

    \{Q_2,\bar{Q}_{\dot{1}}\} & \{Q_2,\bar{Q}_{\dot{2}}\} \\

  \end{bmatrix}
  =
  \begin{bmatrix}
    0 & 0\\
    0 & 4E
  \end{bmatrix}
\end{equation}

intended as acting on the states of the multiplet we are looking for. In
particular,
\begin{equation}
\{Q_1, \bar{Q}_{\dot{1}}\} = 0
\end{equation}
which implies that 
\begin{equation}
  ||Q_1|\omega\rangle||^2 = 0 = ||\bar{Q}_{\dot{1}}|\omega\rangle||^2
\end{equation}
and thus

\begin{equation}
  Q_1|\omega\rangle = 0 = \bar{Q}_{\dot{1}}|\omega\rangle.
\end{equation}
This means that as operators $Q_1$ and $\bar{Q}_{\dot{1}}$ anihilate the
multiplet. 
\par The only nontrivial anticommutation relation that is left is:
\begin{equation}
  \{Q_2,\bar{Q}_{\dot{2}}\} = 1
\end{equation}
If we call
\begin{equation}
  \alpha = \frac{1}{2\sqrt{E}}Q_2,\quad \alpha^\dagger
  = \frac{1}{2E}\bar{Q}_{\dot{2}}
\end{equation}
then the anticommutation relation that is left is:
\begin{equation}
  \{\alpha,\alpha^\dagger\} = 1
\end{equation}
with $\{\alpha,\alpha\} = 0$.

We can build the representation starting from a state $|\lambda\rangle$ such
that
\begin{equation}
  \alpha|\lambda\rangle = 0
\end{equation}
Lets suppose that it has \textit{helicity} $\lambda$:
\begin{equation}
  M_{12}\lambda \equiv J_3|\lambda\rangle = \lambda|\lambda\rangle.
\end{equation}
It is easy to compute the helicity of $\alpha^{\dagger}|\lambda\rangle$:
\begin{equation}
  M_{12} \bar{Q}_{\dot{2}}|\lambda\rangle
  = \left(\bar{Q}_{\dot{2}}M_{12}+\frac{1}{2}\bar{Q}_{\dot{2}}\right)|\lambda\rangle
  = (\lambda+\frac{1}{2}\bar{Q}_{\dot{2}})|\lambda\rangle 
\end{equation}
In the last line, we have used the fact that
$\left[M_{12},\bar{Q}_{\dot{2}}\right]=\frac{1}{2}\bar{Q}_{\dot{2}}$. Thus we
found out that
\begin{equation}
  \alpha^\dagger|\lambda\rangle  = |\lambda+ \frac{1}{2}\rangle 
\end{equation}
Since $(\alpha^\dagger)^2 = 0$, this stops here. Hence we have
\begin{equation}
  \alpha^\dagger|\lambda+\frac{1}{2}\rangle = 0.
\end{equation}
\par
Massless multiplets are thus composed of one boson and one 1
\section{General}
\begin{definition}[R-symmetry]
In supersymmetric theories, an R-symmetry is the symmetry transforming different supercharges into each other. In the simplest case of the $\mathcal{N}=1$ supersymmetry, such R-symmetry is isomorphic to a global $U(1)$ group or it's discrete subgroup. For extended supersymmetry theories, the R-symmetry group becomes a global non-abelian group.  
\end{definition}

\begin{remark}
In the case of the discrete subgroup $\mathbb{Z}_2$, the R-symmetry is called \textit{R-parity}
\end{remark}

\begin{definition}[Extended supersymmetry]
In supersymmetric theories, when $\mathcal{N}>1$ the algebra is said to have
\textit{extended supersymmetry}.
\end{definition}

\section{Bogomol'nyi-Prasad-Sommerfield (BPS) states}
\begin{definition}[BPS state]
A massive representation of an extended supersymmetry algebra that has mass
equal to the supersymmetry central charge $Z$ is called an \textit{BPS} state.
\end{definition}

Quantum mechanically speaking, if the supersymmetry remains unbroken, exact
solutions to the modulus of $Z$ exist. Their importance arises as the
multiplets shorten for generic representations, with stability and mass formula
exact.

\begin{example}[$d=4$,$\mathcal{N}=2$]
The generators for the odd part of the superalgebra have relations:
\begin{gather}
  \{Q_\alpha^A,\bar{Q}_{\dot{\beta}B}\} = 2\sigma^m_{\alpha\dot{\beta}}P_m\delta^A_B \\
  \{Q_\alpha^A,Q_{\beta B}\} = 2\epsilon_{\alpha\beta}\epsilon^{AB}\bar{Z}\\
  \{\bar{Q}_{\dot{\alpha}A},\bar{Q}_{\dot{\beta}B}\}
  = -2\epsilon_{\dot{\alpha}\dot{\beta}}\epsilon_{AB}Z,
\end{gather}
where $\alpha\dot{\beta}$ are the Lorentz group indeces and $A,B$ are the
$R$-symmerty indeces.
If we take linear combinations of the above generators as follows:
\begin{gather}
  R_\alpha^A = \xi^{-1}Q_\alpha^A
  + \xi\sigma^0_{\alpha\dot{\beta}}\bar{Q}^{\dot{\beta}B}\\
  T_\alpha^A = \xi^{-1}Q_\alpha^A - \xi\sigma^0_{\alpha\dot{\beta}}\bar{Q}^{\dot{\beta}B}
\end{gather}
and consider a state $\psi$ which has momentum $(M,0,0,0)$, we have:
\begin{equation}
  \left(R_1^1+(R_1^1)^\dagger\right)^2\psi = 4(M+Re(Z\xi^2))\psi,
\end{equation}
but because this is the square of a Hermitian operator, the right hand side
coefficient must be positive for all $\xi$. In particular, the strongest result from this is
\begin{equation}
  M\geq|Z|
\end{equation}
\end{example}
%%%\begin{table}
%%%    \tabletitle{Examples for illustrating attacks}
%%%    \begin{tabular}{|c|c|c|c|}
%%%        \hline
%%%        \textbf{Job} & \textbf{Sex} & \textbf{Age} & \textbf{Disease} \\
%%%        \hline
%%%        Engineer & Male & 35 & Hepatitis \\
%%%        Engineer & Male & 38 & Hepatitis \\
%%%        Lawyer & Male & 38 & HIV \\
%%%        Writer & Female & 30 & Flu \\
%%%        Writer & Female & 30 & HIV \\
%%%        Dancer & Female & 30 & HIV \\
%%%        Dancer & Female & 30 & HIV \\
%%%        \hline
%%%    \end{tabular}
%%%    \label{table:rawpatient}
%%%\end{table}
\section{Supersymmetric theories on curved manifolds}
\textit{Remark:} Supersymmetric theories may be defined only on backgrounds admitting solutions to certain Killing spinor equations,
\begin{align}
\left(\nabla_\mu-iA_\mu\right)\zeta + iV_\mu\zeta + iV^\nu\sigma_{\mu\nu}\zeta = 0\\
\left(\nabla_\mu+iA_\mu\right)\tilde{\zeta} - iV_\mu\tilde{\zeta} 0 iV^\nu\tilde{\sigma}_{\mu\nu}\tilde{\zeta} = 0
\end{align}
which in four dimensions and Euclidean signature are
equivalent to the requirement that the manifold is complex and the metric Hermitian.

%\begin{Glossary}

%\end{Glossary}
