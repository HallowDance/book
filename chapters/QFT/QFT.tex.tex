%\chapterauthor{Second Author}{Second Author Affiliation}
\chapter{Quantum Field Theory}
\chapterauthor{Ivo Iliev\\Sofia University}
\adjustmtc
\minitoc

\section{Rarita-Schwinger equation}
Consider the following Lagrangian
\begin{equation}
  \mathcal{L} = -\frac{1}{2}\bar{\psi}_\mu\left(\epsilon^{\mu\kappa\rho\nu}\gamma_5\gamma_\kappa\partial_\rho-im\sigma^{\mu\nu}\right)\psi_\nu
\end{equation}
This equation obviously controls the propagation of the wave function of
a spin- object such as the gravitino. The equation of motion for
this Lagrangian are known as the \textit{Rarita-Schwinger equation}:
\begin{equation}
\left(\epsilon^{\mu\kappa\rho\nu}\gamma_5\gamma_\kappa\partial_\rho
  -im\sigma^{\mu\nu}\right)\psi_\nu
\end{equation}
In the massless case, the Rarita-Schwinger equation has a fermionic gauge
symmertrty, it is invariant under the gauge transformation:
\begin{equation}
  \psi_\mu\rightarrow\psi_\mu + \partial_\mu\epsilon,
\end{equation}
where $\epsilon\equiv\epsilon_\alpha$ is an arbitrary spinor field. 
\subsection{Massless case}
Consider a massless Rarita-Schwinger field, described by the Lagrangian
\begin{equation}
  \mathcal{L}_{RS} = \bar{\psi}_\mu\gamma^{\mu\nu\rho}\partial_\nu\psi_{\rho}
\end{equation}
where the sum over spin indices is implicit, $\psi_\mu$ are Majorana spinors
and the quantity $\gamma^{\mu\nu\rho}$ is equal to
\begin{equation}
  \gamma^{\mu\nu\rho}\equiv\frac{1}{3!}\gamma^{[\mu}\gamma^\nu\gamma^{\rho]}
\end{equation}
Varying the Lagrangian yealds after some calculation
\begin{equation}
  \delta\mathcal{L_{RS}}
  = 2\delta\bar{\psi}_\mu\gamma^{\mu\nu\rho}\partial_\nu\psi_\rho
  + \text{boundary terms}
\end{equation}
Now imposing that $\mathcal{L}_{RS} =0$ we get the equation of motion for
a massless Majorana Rarita-Schwinger spinor:
\begin{equation}
  \gamma^{\mu\nu\rho}\partial_\nu\psi_\rho = 0
\end{equation}
\subsection{Massive case}
The description of massive, higher-spin fields through the Rarita-Schwinger
equation is not well defined physically. Coupling the RS Largrangian to
electromagnetism leads to an equation with solutions representing wavefronts,
some of which propagate faster than light. However, it was shown by Das and
Freedman that local supersymmetry can circumvent this problem.
%%%\begin{table}
%%%    \tabletitle{Examples for illustrating attacks}
%%%    \begin{tabular}{|c|c|c|c|}
%%%        \hline
%%%        \textbf{Job} & \textbf{Sex} & \textbf{Age} & \textbf{Disease} \\
%%%        \hline
%%%        Engineer & Male & 35 & Hepatitis \\
%%%        Engineer & Male & 38 & Hepatitis \\
%%%        Lawyer & Male & 38 & HIV \\
%%%        Writer & Female & 30 & Flu \\
%%%        Writer & Female & 30 & HIV \\
%%%        Dancer & Female & 30 & HIV \\
%%%        Dancer & Female & 30 & HIV \\
%%%        \hline
%%%    \end{tabular}
%%%    \label{table:rawpatient}
%%%\end{table}

%\section{Glossary}
%\begin{Glossary}

%\end{Glossary}

