\chapter{Differential Geometry}
\chapterauthor{Ivo Iliev\\Sofia University}
\adjustmtc
\minitoc
\section{General}
\begin{definition}[Einstein Manifold]
A Riemannian manifold $M$ is called an \textit{Einstein manifold} if its Ricci tensor is proportional to the metric, i.e.
\begin{equation}
Ric_g = \lambda g
\end{equation}
\end{definition}
\begin{remark}
An Einsteinian manifold, where $\lambda=0$ is called a \textit{Ricci-flat} manifold.
\end{remark}

\section{Differential Forms}
\epigraph{"Hamiltonian mechanics cannot be understood without differential
forms"}{V. I. Arnold}

In this section we will give a very brief introduction to differential forms
and the general rules for calculus with differential forms. 
\subsection{Exterior forms}
We begin with the more general notion of a \textit{exterior form}, which is
generally a poly-linear map from a vector space to an algebraic filed.
\subsubsection{Definitions}
Let $\mathbb{L}^n$ be an $n$-dimensional real vector space.
\begin{definition}[Exterior algebraic form of degree $k$]
An exterior algebraic form of degree $k$, also known as a $k$-form, is
a function of $k$ vectors which is $k$-linear and anti-symmetric. Namely:
\begin{equation}
  \omega(\lambda_1\xi_1' + \lambda_2\xi_1'',\xi_2,\cdots,\xi_k)
  = \lambda_1\omega(\xi_1',\xi_2,\cdots,\xi_k)
  + \lambda_2\omega(\xi_1'',\xi_2,\cdots,\xi_k),
\end{equation}
and
\begin{equation}
  \omega(\xi_{i_1},\cdots,\xi_{i_k}) = (-1)^\nu\omega(\xi_1,\cdots,\xi_k)
\end{equation}
with $(-1)^\nu = 1$ if the permutation $(i_1,\cdots,i_k)$ is even and  $(-1)^\nu
= -1$ if the same permutation is odd. Here, $\xi_i\in\mathbb{L}^n$ and
$\lambda_i\in\mathbb{R}$.
\end{definition}
The set of all $k$-forms in $\mathbb{L}^n$ forms a real vector space. Indeed,
one has that:
\begin{equation}
   (\omega_1+\omega_2)(\xi) = \omega_1(\xi) + \omega_2(\xi), \quad \xi
   = \{\xi_1,\cdots,\xi_k\}
\end{equation}
and
\begin{equation}
  (\lambda\omega)(\xi) = \lambda\omega(\xi).
\end{equation}
Since $\mathbb{L}^n$ is a vector space, we can always suppose that it is
equipped with a coordinate system. Let us denote these coordinates by
$x_1,\cdots,x_n$. Now, we can think of these coordinates as 1-forms so that
$x_i(\xi) = \xi_i$ and that is the $i$-th coordinate of the vector $\xi$. These
coordinates form a basis of 1-forms in the 1-form vector space (which is also
called the \textit{dual space} $(\mathbb{L}^n)^*$).
\par Any 1-form can then be
written as a linear combination of basis 1-forms:
\begin{equation}
  \omega_1 = a_1x_1 + \cdots + a_nx_n.
\end{equation}

\subsubsection{Exterior Products}
\begin{definition}[Exterior Product]
An exterior product of $k$ 1-forms $\omega_1,\omega_2,\cdots,\omega_k$ is
a $k$-form defined by
\begin{equation}
  (\omega_1\wedge\omega_2\wedge\cdots\wedge\omega_k)(\xi_1,\xi_2,\cdots,\xi_k)
  = det\omega_i(\xi_j).
\end{equation}
\end{definition}
\begin{remark}
  Exterior products of basis 1-forms $x_{i_1}\wedge\cdots\wedge x_{i_k}$ with
  $i_1<\cdots<i_k$ for a basis in the space of $k$-forms. The dimension the latter
  space is obviously $C_n^k$.
\end{remark}
A general $k$-form can be written as
\begin{equation}
  \omega^k = \sum_{1\leq i_1<\cdots<i_k\leq n}{a_{i_1\cdots
  i_k}x_{i_1}\wedge\cdots\wedge x_{i_k}}
\end{equation}
where $a_{i_1\cdots i_k}$ are real numbers.

\begin{definition}[Exterior product of an $k$-form and an $l$-form]
  The exterior product $\omega^k\wedge\omega^l$ of a $k$-form with an $l$-form
  on $\mathbb{L}^n$ is the $k+l$-form on $\mathbb{L}^n$ defined as:
  \begin{equation}
    (\omega^k\wedge\omega^l)(\xi_1,\cdots,\xi_{k+l})
    = \sum{(-1)^\nu\omega^k(\xi_{i_1},\cdots,\xi_{i_k})\wedge\omega^l(\xi_{j_1},\cdosts,\xi_{j_l})}
  \end{equation}
  where $i_1<\cdots<i_k$ and $j_1,<\cdots<j_l$ and the sum is taken over all
  permutations $(i_1,\cdots,i_k,j_1,\cdots,j_l)$ with $(-1)^\nu$ being +1
  for even permutations and -1 for odd permutation.
\end{definition}
One can check that this definition is consistent with the definition for
exterior product of 1-forms. Furthermore it can be shown that the exterior
product is distributive, associative and \textit{skew-commutative}. The latter
means that $\omega^k\wedge\omega^l = (-1)^{kl}\omega^l\wedge\omega^k$. If
$\omega$ is a 1-form (or a form of any odd degree) one can easily show that
$\omega\wedge\omega=0$. We will present a brief version of the proof here:
\begin{proof}
Let $\omega$ be an exterior $k$-form, where $k$ is an odd integer. Then the
exterior product takes the form:
 \begin{equation}
 \omega\wedge\omega = (-1)^{k^2}\omega\wedge\omega
 \end{equation}
 But $k$ is odd, therefore $k^2$ is odd. From that we see that
 \begin{equation}
   \omega\wedge\omega = (-1)^k^2\omega\wedge\omega = -\omega\wedge\omega
 \end{equation}
 From this we conclude that $\omega\wedge\omega = 0$.
\end{proof}
\subsubsection{Behaviour under mappings}
Let $f:\mathbb{L}^m\rightarrow\mathbb{L}^n$ be a linear map and $\omega^k$ an
exterior $k$-form on $\mathbb{L}^n$. We can define a $k$-form $(f^*\omega^k)$
on $\mathbb{L}^m$ by
\begin{equation}
  (f^*\omega^k)(\xi_1,\cdots,\xi_k) = \omega^k(f\xi_1,\cdots,f\xi_k)
\end{equation}
Notice that the obtained mapping of forms $f^*$ acts in \textit{opposite}
direction to $f$. Namely,
$f^*:\Omega_k(\mathbb{L}^n)\rightarrow\Omega_k(\mathbb{L}^b)$, where
$\Omega_k(\mathbb{L}^n)$ is the vector space of $k$-forms on $\mathbb{L}^n$.
\subsection{Differential Forms}
After the general discussion about exterior forms it is time to restrict
ourselves to a more practical object, namely the differential form.
Understanding differential forms or at least having working knowledge about the
topic is paramount for the study of differential geometry and physics on curved
manifolds. Although this may seem abstract at first, we urge the reader to push
through the mathematical definitions and grasp the essence of the idea, that is
how do we perform calculus on curved manifolds without the notion of
coordinates. 
\begin{definition}[Differential Form]
  A differential $k$-form $\omega^k\vert_x$ at a point $x$ of a manifold $M$ is
  an exterior $k$-form on the tangent space $TM_x$ to $M$ at $x$, i.e.,
  a $k$-linear skew-symmetric function of $k$ vectors $\xi_1,\cdots,\xi_k$
  tangent to $M$ at $x$. If such a form is given at every point $x$ of $M$ and
  if it is differentiable, we say that we are given a $k$-form $\omega^k$ on
  the manifold $M$.
\end{definition}
We intoduce the coordinate basis 1-forms $dx_i$ with $i=1,\cdots,n$ $(dimM
= n)$. The notation $dx_i$ is used for the exterior forms emphasizes that these
basis forms act on $TM_x$ at a given point $x$ of $M$. In the neighborhood of
$x$ one can always write the general differential $k$-form as
\begin{equation}
  \omega^k
  = \sum_{i_1<\cdots<i_k}{a_{i_1,\cdots,i_k}(x)dx_{i_1}\wedge\cdots\wedge
  dx_{i_k}},
  \label{eq:generaldifferentialform}
\end{equation}
where $a_{i_1\cdots i_k}$ are smooth functions of $x$. Let's give a simple
example.
\begin{example}
  The differential of some scalar function $f(x)$ defined on a manifold $M$ is
  a differential 1-form. We introduce
  \begin{equation}
    df = \sum_{k=1}^n{\frac{\partial f}{\partial x_k}\bigg\vert_x dx_k},
  \end{equation}
where $dx_k$ are basis differential 1-forms. The value of this 1-form on
a vector $\xi\in TM_x$ is given by
\begin{equation}
  df(\xi) = \sum_{k=1}^n{\frac{\partial f}{\partial x_k}\bigg\vert_x \xi_k}.
\end{equation}
\end{example}
\subsubsection{Behaviour under mappings}
Let $f:M\rightarrow N$ be a differentiable map of a smooth manifold $M$ to
a smooth manifold $N$, and let $\omega$ be a differential $k$-form o $N$. The
mapping $f$ induces the mapping $f_*:TM_x\rightarrow TM_{f(x)}$ of tangent
spaces. The latter mapping $f_*$ is called the \textit{differential} of the map
$f$. The mapping $f_*$ is a mapping of linear spaces and gives rise to the
mapping of forms defined on corresponding tangent spaces. As a result
a well-defined differential $k$-form $(f^*\omega)$ exists on $M$:
\begin{equation}
  (f^*\omega)(\xi_1,\cdots,\xi_k) = \omega(f_*\xi_1,\cdots,f_*\xi_k).
\end{equation}
\subsection{Integration of Differential Forms over Chains}
\subsubsection{Integration of $k$-form over $k$-dimensional cell}
Let $D$ be a bounded convex polyhedron in $\mathbb{L}^k$ and $x_1,\cdots,x_k$
an oriented coordinate system on $\mathbb{L}^k$. Any differential $k$-form on
$\mathbb{L}^k$ can be written as $\omega^k=\phi(x)dx_1\wedge\cdots\wedge dx_k$,
where $\phi(x)$ is a differentiable function on $\mathbb{R}^k$. We define
the integral of the form $\omega^k$ over $D$ as the integral of the function
$\phi(x)$:
\begin{equation}
  \int_D{\omega_k} = \int_D{\phi(x)dx_1dx_2\cdots dx_k}.
\end{equation}

\begin{definition}[$k$-dimensional cell]
A $k$-dimensional cell $\sigma$ of an $n$-dimensional manifold $M$ is
a polyhedron $D$ in $\mathbb{L}^n$ with a differentiable map $f:D\rightarrow
M$.
\end{definition}
One can think about $\sigma$ as a "curvilinear polyhedron" - the image of $D$
on $M$. If $\omega$ is a differentiable $k$-form on $M$, we define the integral
of a form over the cell $\sigma$ as
\begin{equation}
  \int_\sigma{\omega} = \int_D{f^*\omega},
\end{equation}
where $f^*$ is a mapping of $k$-forms induced by $f$.
\par The cell $\sigma$ inherits an orientation from the orientation of
$\mathbb{L}^k$. The $k$-dimensional cell which differs from $\sigma$ only by
the choice of orientation is called the \textit{negative} of $\sigma$ and is
denoted by $-\sigma$ or by $(-1)\sigma$. One can show that under a change of
orientation the integral changes sign:
\begin{equation}
  \int_{-\sigma}{\omega}=-\int_{\sigma}{\omega}.
\end{equation}

\subsubsection{Chains and the Boundary Operator}
It is convenient to generalize our definition of the integral of a form over
\textit{cell} to the integral over a \textit{chain}.
\begin{definition}[Chain of a Manifold]
  A chain of dimension $k$ on a manifold $M$ consists of a finite collection of
  $k$-dimensional oriented cells $\sigma_1,\cdots,\sigma_r$ in $M$ and integers
  $m_1,\cdots,m_r$ called multiplicities. A chains is denoted by
  \begin{equation}
    c_k = m_1\sigma_1+\cdots+m_r\sigma_r .
  \end{equation}
\end{definition}
One can introduce the structure of a commutative group on a set of $k$-chains
on $M$ with natural definitions of addition of chains $c_k + b_k$.
\begin{definition}[Boundary Operator]
  The boundary of a convex oriented $k$-polyhedron $D$ on $\mathbb{L}^k$ is the
  $(k-1)$-chain $\partial D$ on $\mathbb{L}^k$ defined as
  \begin{equation}
    \partial D = \sum_i{\sigma_i}
  \end{equation}
  where the cells $\sigma_i$ are the $(k-1)$-dimensional faces of $D$ with
  orientations inherited from the orientation of $\mathbb{L}^k$.
\end{definition}

One can easily extend this definition to the definition of the \textit{boundary
of a cell} $\partial\sigma$ on $M$ and then to the boundary of a chain. Indeed,
defining:
\begin{definition}[Boudary of a Chain]
  \begin{equation}
    \partial c_k = m_1\partial\sigma_1 + \cdots + m_r\partial\sigma_r
  \end{equation}
\end{definition}
We can see that $\partial c_k$ is a $(k-1)$-chain on $M$. Additionally, we
define a 0-chain as a collection of points with multiplicities. Furthermore we
define the boundary of an oriented interval $\vec{AB}$ as $B-A$. The boundary
of a point is empty.
\par It is straightforward to show that the boundary of the boundary of a cell
is zero. Therefore 
\begin{equation}
  \partial(\partial c_k) = 0
\end{equation}
for any $k$-chain $c_k$. We denote this property as
\begin{equation}
  \partial\partial = \partial^2 = 0
\end{equation}
\subsubsection{Integration of a $k$-form over a $k$-chain}
An integral of a $k$-form over a $k$-chain is then defined as
\begin{equation}
  \int_{c_k}{\omega^k} = \sum{m_i\int_{\sigma_i}{\omega^k}}.
\end{equation}
\subsection{Exterior Differentiation and Stokes Formula}
\begin{definition}[Exterior Derivative of a Form]
  An \textit{exterior derivative} of a differential $k$-form $\omega$ is
  a $(k+1)$-form $\Omega=d\omega^k$. Given a set of coordinates $\{dx_{i_j}\}$,
  we have:
  \begin{equation}
    \Omega = d\omega^k = \sum{da_{i_1\cdots i_k}\wedge
      dx_{i_1}\wedge\cdosts\wedge dx_{i_k}}
    \end{equation}
    implying that we have defined $\omega^k$ as in
    \eqref{eq:generaldifferentialform}. Here, $da$ is a 1-form, the
    differential of the function $a(x)$.
\end{definition}
One can show that the definition does not actually depend on the choice of
coordinates. We can think of the 1-form given by the differential of a scalar
function as of an external derivative of a 0-form.
It is easy to show that $d(df)=0$, if $f$ belongs to the set of 0-forms. Indeed
\begin{proof}
  If f belongs to the set of 0-forms, then by definition
  \begin{equation}
   df = \sum{\frac{\partial a}{\partial x^i}dx^i},
  \end{equation}
  which is a 1-form. Taking an exterior derivative again yields
  \begin{align}
     d(df) &= d\left(\sum{\frac{\partial a}{\partial x^i}dx^i}\right)=\nonumber\\
    &= \sum{\frac{\partial^2}{\partial x^i\partial x^j}dx^i\wedge x^j}
     =\nonumber\\
    &= \frac{\partial^2 g}{dx^i dx^j}dx^i\wedge dx^j + \frac{\partial^2 g}{dx^j
    dx^i}dx^j\wedge dx^i\nonumber\\
    &= 0
  \end{align}
  Here we have used that $\frac{\partial g^2}{dx^i dx^j} = \frac{\partial
  g^2}{dx^j dx^i}$ and $dx^i\wedge dx^j = -dx^j\wedge dx^i$.
\end{proof}
Using this result, one can show that it holds for forms of any degree.
\par Another useful formula is that for differentiating an exterior product of
form:
\begin{equation}
  d(\omega^k\wedge\omega^l) = d\omega^k\wedge d\omega^l + (-1)^k\omega^k\wedge
  d\omega^l
\end{equation}
\par Lastly, if $f:M\rightarrow N$ is a smooth map and $\omega$ is a $k$-form
on $N$, we have
\begin{equation}
  f^*(d\omega) = d(f^*\omega).
\end{equation}
\subsubsection{Stokes' Formula}
One of the most-important formulae in differential geometry is the
\textit{Newton-Leibniz-Gauss-Green-Ostrogradski-Stokes-Poincar\'e} formula:
\begin{equation}
  \boxed{\int_{\partial c}{\omega} = \int_c{d\omega}}
\end{equation}
where $c$ is any $(k+1)$-chain on a manifold $M$ and $\omega$ is any $k$-form
on M. In the case when the boundary $\partial c=0$, we have $\int_c{d\omega}
= 0$, which corresponds to integration of a complete derivative over a closed
surface.
\subsection{Homologies and Cohomologies}
\subsubsection{Closed and Exact Forms}
\begin{definition}[Closed Form]
  A differential form $\omega$ on a manifold $M$ is said to be \textit{closed}
  if $d\omega = 0.$
\end{definition}
In particular, on a 3D Riemannian manifold, we have $d\omega^2_{\vec{A}}
= \left(\nabla\cdot\vec{A}\right)\omega^3 = 0$, which is equivalent to
$\left(\nabla\cdot\vec{A}\right) = 0$, i.e the corresponding vector field is
divergenless. We can apply Stokes' formula for a closed form, getting
\begin{equation}
  \int_{\patial c_{k+1}}{\omega^k} = 0\quad\text{if }dw^k=0.
  \label{eq:stokesclosedform}
\end{equation}
\begin{definition}[Exact Form]
  A  differential form $\omega$ on a manifold $M$ is said to be \texit{exact}
  if there exists such a differential form $\mu$ that $\omega = d\mu$.
\end{definition}
Since $d(d\omega) = 0$ as we've proven all exact forms are closed. However,
there are some closed forms which are not exact. Let's give a short example.
\begin{example}
  Consider the circle $S^1$, parametrized by an angle $\phi\in [0,2\pi]$. One
  can introduce the 1-form $\omega^1$, defined by
  $\omega^1(\partial_t\gamma)=\partial_t\gamma$, where $\partial_t\gamma$ is
  the "velocity" along the path $\phi = \gamma(t)$. Obviously, this "velocity",
  belongs to the tangent space of $S^1$ at the point with coordinate $\phi$.
  The form is closed - $d\omega$ is a 2-form, and we can't have 2-forms on
  a one-dimensional manifold. However:
  \begin{equation}
    \int_{S^1}\omega^1 = \int_{0}^T{dt \partial_t\gamma} = 2\pi,
  \end{equation}
the length of the circle. Although the boundary $\partial S^1$ is zero, the
integral is not zero, and therefore $\omega^1$ is not exact.
\end{example}
\par One can notice that locally the introduced 1-form can be written as
$\omega^1 = d\phi$, which can look like a contradiction. It is easy to see
where the problem lies - writing $\omega^1 = d\phi$ is not valid for $\phi=0$.
We've come across an example of a general result, namely
\begin{theorem}[Poincar\'e's Lemma]
  Any closed form is locally exact.
\end{theorem}
The existence of locally but not globally exact closed forms is related to some
topological properties of the underlying manifold $M$.

\begin{subsubsection}{Cycles and Boundaries}
  \begin{definition}[Cycle on a Manifold]
    A \texit{cycle} on a manifold $M$ is a chain whose boundary is equal to
    zero.
  \end{definition}
  Using Stokes theorem, we have
  \begin{equation}
    \int_{c_{k+1}}{d\omega^k} =0\quad \text{if } \partial c_{k+1} =0.
  \end{equation}
  Chains that can be considered boundaries of some other chains are called
  \textit{boundaries}. Since $\parital\partial = 0$, all boundaries are cycles.
  However, not all cycles are boundaries. The existence of cycles that are not
  boundaries is again related to some topological properties of the manifold.
  A fairly simple example is found on the 2-torus. The 2-torus is the direct
  product of 2 circles - $T^2=S^1\cross S^1$. Each of the $S^1$ is a cycle, but
  none of them are boundaries. 

  \begin{subsubsection}{Homologies and Cohomologies}
  The set of all $k$-forms on $M$ is a vector space, the set of all
  \textit{closed} $k$-forms is a subspace of that space and the set of
  differentials of $(k-1)$-forms (the \text{exact} $k$-forms) are a subspace of
  the subspace of closed forms. We can now define:
  \begin{definition}
    The quotient space:
  \begin{equation}
    \frac{(\text{closed forms})}{(\text{exact forms})} = H^k(M,\mathbb{R})
  \end{equation}
    is called the \textit{k-th cohomology group} of the manifold $M$. An
    element of this group is a class of closed forms, differing from each other
    only by an exact form.
  \end{definition}
  For the circle $S^1$, we have $H^1(S^1,\mathbb{R}) = \mathbb{R}$.
  \begin{definition}[Betti Number]
    The dimension of $H^k$ is called the $k$-th \texit{Betti number} of $M$.
  \end{definition}
  Obviously, the first Betti number of $S^1$ is 1. The cohomology groups of $M$ are important \textit{topological}
  properties of $M$.

  \begin{definition}[Homologous Cycles]
  Let us now consider two $k$-cyclces $a$ and $b$, such that their 
  difference is a boundary of a $(k+1)$-chain, i.e. $a-b = \partial c_{k+1}$.
  Such cycles are called \texit{homologous}.
\end{definition}
  Let us have two $k$-cycles, $a$ and $b$, homologous to each other and
  a closed form $\omega^k$. From \eqref{eq:stokesclosedform} we can see that 
  \begin{equation}
    \int_a{\omega^k} = \int_b{\omega^k}.
  \end{equation}
  In other words, homologous cycles can be replaced with one another for
  integration paths.
  \begin{definition}[Homology Group]
    The quotient group
    \begin{equation}
      \frac{(\text{cycles})}{(\text{boundaries})} = H_k(M)
    \end{equation}
    is called the $k$-th homology group of $M$. An element of this group is
    a class of cycles homologous to each other. The rank of this group is also
    equal to the $k$-th Betti number of $M$.
  \end{definition}
\subsection{Homologies and Homotopies}
There are important relations between homology and homotopy groups of
a topological space $M$.
\par Let us suppose that $\pi_1(M)$ and $H_1(M)$ are the fundamental and the
first homology group of $M$, respectively. Then $H_1(M)
= \pi_1(M)/[\pi_1,\pi_1]$, where $[\pi_1,\pi_1]$ is the commutator in the
fundamental group. In particular, if $\pi_1(M)$ is Abelian, then $\pi_1(M)
= H_1(M)$. For the higher homotopy groups, there is another result, known as
the Gurevich theorem.
\begin{theorem}[Gurevich theorem]
  If $\pi_k(M) = 0$ for all $k<n$, then
  \begin{equation}
    \pi_n(M) = H_n(M).
  \end{equation}
\end{theorem}
As a general rule of thumb, homology (and cohomology) groups are usually easier
to calculate than homotopy groups.
    \section{Contact Manifolds}
\begin{definition}[Riemannian Cone]
Given a Riemannian manifold $(M,g)$, its \textit{Riemannian cone} is a product
\begin{equation}
(M\times\mathbb{R}^{>0})
\end{equation}
of $M$ with the half-line $\mathbb{R}^{>0}$ equipped with the \textit{cone metric}
\begin{equation}
t^2g+dt^2,
\end{equation}
where t is a parameter in $\mathbb{R}^{>0}$
\end{definition}

\begin{definition}[Contact Manifold]
A manifold $M$, equipped with a 1-form $\theta$ is contact if and only if the 2-form
\begin{equation}
t^2d\theta + 2tdt\cdot\theta
\end{equation}
on its cone is symplectic.
\end{definition}

\begin{definition}[Sasakian Manifold]
A contact Riemannian manifold is called a Sasakian manifold, if its Riemannian cone with the cone metric is a K\"ahler manifold with K\"ahler form
\begin{equation}
t^2d\theta + 2t dt\cdot \theta .
\end{equation}
\end{definition}

\begin{example}
Consider the manifold $\mathbb{R}^{2n+1}$ with coordinates $(\vec{x},\vec{y},z)$, endowed with contact form 
\begin{equation}
\theta = \frac{1}{2}dz + \Sigma_i{y_idx_i}
\end{equation}
and Riemannian metric

\begin{equation}
g = \Sigma_i{(dx_i)^2 + (dy_i)^2 + \theta^2}
\end{equation}
\end{example}

\begin{definition}[Sasaki-Einstein Manifold]
A Sasaki-Einstein manifold is a Riemanian manifold $(S,g)$ that is both Sasakian and Einstein
\end{definition}

\begin{example}
The odd dimensional sphere $S^{2n-1}$, equipped with its standard Einstein metric is a Sasaki-Einstein manifold. In this case, the K\"ahler cone is $\mathbb{C}^2\backslash\{0\}$, equipped with its flat metric.
\end{example}

\section{Symplectic Geometry}
\begin{theorem}[Duistermaat-Heckman Formula]
For a compact symplectic manifold $M$ of dimension $2n$ with symplectic form $\omega$ and with a Hamiltonian $U(1)$ action whose moment map is denoted by $\mu$, the following formula holds:
\begin{equation}
\int_M{\frac{\omega^n}{n!}e^{-\mu}} = \sum_i{\frac{e^{-\mu(x_i)}}{e(x_i)}}
\end{equation}
Here, $x_i$ are the fixed points of the $U(1)$ action and they are assumed to be isolated, and $e(x_i)$ is the product of the weights of the $U(1)$ action on the tangent space at $x_i$.

\end{theorem}

\section{Complex Manifolds}
\begin{definition}[Hermitian Metric]
If a Riemannian metric $g$ of a complex manifold $M$ satisfies 
\begin{equation}
g_p(J_pX, J_pY) = g_p(X,Y)
\end{equation}
at each point $p\in M$ and for any $X,Y\in T_p M$, $g$ is said to be a \textit{Hermitian metric}. Here, $J_p$ denotes the almost complex structure on $M$.
\end{definition}

\begin{definition}[Hermitian Manifold]
The pair $(M,g)$ is called a \textit{Hermitian manifold}
\end{definition}

\begin{theorem}
A complex manifold always admits a Hermitian metric.
\end{theorem}

\begin{proof}
Let $g$ be any Riemannian metric of a complex manifold $M$. Define a new metric $\hat{g}$ by
\begin{equation}
\hat{g}_p(X,Y)\equiv \frac{1}{2}\left[g_p(X,Y) + g_p(J_pX, J_pY)\right].
\end{equation}
Clearly $\hat{g}_p(J_pX,J_pY) = \hat{g}_p(X,Y)$. Moreover, $\hat{g}$ is positive definite provided that $g$ is. Hence, $\hat{g}$ is a Hermitian metric on M
\end{proof}

\begin{definition}[K\"ahler Form]
  Let $(M,g)$ be a Hermitian manifold. Define a tensor field $\Omega$ whose
  action on $X,Y\in T_pM$ is
\begin{equation}
\Omega_p(X,Y) = g_p(J_pX,Y)
\end{equation}
Note that $\Omega$ is anti-symmetric, $\Omega(X,Y) = g(JX,Y) = g(J^2X,JY)
= -g(JY,X) = -\Omega(Y,X)$. Hence, $\Omega$ defines a two-form, called the
\textit{Kh\"aler form} of a Hermitian metric g.
\end{definition}

\begin{definition}[K\"ahler Manifold]
  A \textit{K\"ahler manifold} is a Hermitian manifold $(M,g)$ whose K\"ahler
  form $\Omega$ is closed, $d\Omega=0$. The metric $g$ is called the
  \textit{K\"ahler metric} of M.
\end{definition}

\begin{remark}
Not all complex manifolds admit K\"ahler metrics
\end{remark}

\begin{theorem}
A Hermitian manifold $(M,g)$ is a K\"ahler manifold if and only if the almost
  complex structure $J$ satisfies
\begin{equation}
\nabla_\mu J = 0
\end{equation}
where $\nabla_\mu$ is the Levi-Cevita connection associated with $g$.
\end{theorem}

\begin{proof}
We first note that for any $r$-form $\omega$, d$\omega$ is written as
\begin{equation}
  d\omega = \nabla\omega \equiv
  \frac{1}{r!}\nabla_\mu\omega_{\nu_1\cdots\nu_r}dx^\mu\wedge
  dx^{\nu_1}\wedge\cdots\wedge dx^{\nu}    
\end{equation}
Now we prove that $\nabla_\mu J = 0$ if and only if $\nabla_\mu\Omega = 0$. We
verify the following equalities:
\begin{align}
(\nabla_Z\Omega)(X,Y)
  &= \nabla_Z\left[\Omega(X,Y)\right]-\Omega(\nabla_ZX,Y)-\Omega(X,\nabla_z Y)\\
  &= \nabla_Z\left[g(JX,Y)\right] - g(J\nabla_Z X,Y) - g(JX, \nabla_Z Y)\\
  &= (\nabla_Z g)(JX,Y) + g(\nabla_ZJX,Y) - g(J\nabla_ZX,Y)\\
  &= g(\nabla_ZJX-J\nabla_ZX,Y) = g((\nabla_ZJ)X,y)
\end{align}
where $\nabla_Z g = 0$ has been used. Since this is true for any $X,Y,Z$, if
  follows that $\nabla_Z\Omega = 0$ if and only if $\nabla_Z J = 0$.
\end{proof}
The last theorem shows that the Riemann structure is compatable with the
Hermitian structure in the K\"ahler manifold.

We can also characterize K\"ahler manifolds as Hermitian manifolds for which
the Cristoffel symbols of the Levi-Chevita connection are pure. In other words,
$\Gamma_{jk}^i$ and $\Gamma_{\bar{j}\bar{k}}^{\bar{i}}$ may be non zero, but
all "mixed" symbols like $\Gamma_{jk}^{\bar{i}}$, for example, vanish. This
means that (anti-)holomorphic vectors get parallel transported to
(anti-)holomorphic vectors.

K\"ahler manifolds are manifolds on which we can always find
a \textit{holomorphic} change of coordinates which, at some given point, sets
the metric to its cannonical form, and its first derivatives to zero.

Equivalently, an $n$-dimensional K\"ahler manifold are precisely
$2n$-dimensional Riemannian manifolds with holonomy group contained in $U(1)$
\begin{definition}[Hopf Surface]
Let $\mathbb{Z}$ act on $\mathbb{C}^n \backslash \{0\}$ by $\left(z_1,\cdots,z_n\right)\rightarrow \left(\lambda^k z_1,\cdots,\lambda^k z_n\right)$ for $k\in\mathbb{Z}$. For $0 < \lambda < 1$ the action is free and discrete. The quotient complex manifold $X = \left(\mathbb{C}^n\backslash\{0\}\right)/\mathbb{Z}$ is diffeomorphic to $S^1\times S^{2n-1}$. For $n=1$ this manifold is isomorphic to a complex torus $\mathbb{C}/\Gamma$. The lattice $\Gamma$ can be determined explicitly.\\
In other words,  a Hopf manifold is obtained as a quotient of the complex vector space (with zero deleted) $\mathbb{C}^n \backslash \{0\}$ by a free action of the group $\Gamma \cong \mathbb{Z}$ of integers, with the generator $\gamma$ of $\Gamma$ acting by holomorphic contractions.
\end{definition}


%\section{Glossary}
%\begin{Glossary}

%\end{Glossary}
