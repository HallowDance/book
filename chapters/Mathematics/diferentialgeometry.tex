\chapterauthor{Ivo Iliev}{Sofia University}
%\chapterauthor{Second Author}{Second Author Affiliation}
\chapter{Differential Geometry}

\section{General}
\begin{definition}[Einstein Manifold]
A Riemannian manifold $M$ is called an \textit{Einstein manifold} if its Ricci tensor is proportional to the metric, i.e.
\begin{equation}
Ric_g = \lambda g
\end{equation}
\end{definition}
\begin{remark}
An Einsteinian manifold, where $\lambda=0$ is called a \textit{Ricci-flat} manifold.
\end{remark}


\section{Contact Manifolds}
\begin{definition}[Riemannian Cone]
Given a Riemannian manifold $(M,g)$, its \textit{Riemannian cone} is a product
\begin{equation}
(M\times\mathbb{R}^{>0})
\end{equation}
of $M$ with the half-line $\mathbb{R}^{>0}$ equipped with the \textit{cone metric}
\begin{equation}
t^2g+dt^2,
\end{equation}
where t is a parameter in $\mathbb{R}^{>0}$
\end{definition}

\begin{definition}[Contact Manifold]
A manifold $M$, equipped with a 1-form $\theta$ is contact if and only if the 2-form
\begin{equation}
t^2d\theta + 2tdt\cdot\theta
\end{equation}
on its cone is symplectic.
\end{definition}

\begin{definition}[Sasakian Manifold]
A contact Riemannian manifold is called a Sasakian manifold, if its Riemannian cone with the cone metric is a K\"ahler manifold with K\"ahler form
\begin{equation}
t^2d\theta + 2t dt\cdot \theta .
\end{equation}
\end{definition}

\begin{example}
Consider the manifold $\mathbb{R}^{2n+1}$ with coordinates $(\vec{x},\vec{y},z)$, endowed with contact form 
\begin{equation}
\theta = \frac{1}{2}dz + \Sigma_i{y_idx_i}
\end{equation}
and Riemannian metric

\begin{equation}
g = \Sigma_i{(dx_i)^2 + (dy_i)^2 + \theta^2}
\end{equation}
\end{example}

\begin{definition}[Sasaki-Einstein Manifold]
A Sasaki-Einstein manifold is a Riemanian manifold $(S,g)$ that is both Sasakian and Einstein
\end{definition}

\begin{example}
The odd dimensional sphere $S^{2n-1}$, equipped with its standard Einstein metric is a Sasaki-Einstein manifold. In this case, the K\"ahler cone is $\mathbb{C}^2\backslash\{0\}$, equipped with its flat metric.
\end{example}

\section{Symplectic Geometry}
\begin{theorem}[Duistermaat-Heckman Formula]
For a compact symplectic manifold $M$ of dimension $2n$ with symplectic form $\omega$ and with a Hamiltonian $U(1)$ action whose moment map is denoted by $\mu$, the following formula holds:
\begin{equation}
\int_M{\frac{\omega^n}{n!}e^{-\mu}} = \sum_i{\frac{e^{-\mu(x_i)}}{e(x_i)}}
\end{equation}
Here, $x_i$ are the fixed points of the $U(1)$ action and they are assumed to be isolated, and $e(x_i)$ is the product of the weights of the $U(1)$ action on the tangent space at $x_i$.

\end{theorem}

\section{Complex Manifolds}
\begin{definition}[Hermitian Metric]
If a Riemannian metric $g$ of a complex manifold $M$ satisfies 
\begin{equation}
g_p(J_pX, J_pY) = g_p(X,Y)
\end{equation}
at each point $p\in M$ and for any $X,Y\in T_p M$, $g$ is said to be a \textit{Hermitian metric}. Here, $J_p$ denotes the almost complex structure on $M$.
\end{definition}

\begin{definition}[Hermitian Manifold]
The pair $(M,g)$ is called a \textit{Hermitian manifold}
\end{definition}

\begin{theorem}
A complex manifold always admits a Hermitian metric.
\end{theorem}

\begin{proof}
Let $g$ be any Riemannian metric of a complex manifold $M$. Define a new metric $\hat{g}$ by
\begin{equation}
\hat{g}_p(X,Y)\equiv \frac{1}{2}\left[g_p(X,Y) + g_p(J_pX, J_pY)\right].
\end{equation}
Clearly $\hat{g}_p(J_pX,J_pY) = \hat{g}_p(X,Y)$. Moreover, $\hat{g}$ is positive definite provided that $g$ is. Hence, $\hat{g}$ is a Hermitian metric on M
\end{proof}

\begin{definition}[K\"ahler Form]
  Let $(M,g)$ be a Hermitian manifold. Define a tensor field $\Omega$ whose
  action on $X,Y\in T_pM$ is
\begin{equation}
\Omega_p(X,Y) = g_p(J_pX,Y)
\end{equation}
Note that $\Omega$ is anti-symmetric, $\Omega(X,Y) = g(JX,Y) = g(J^2X,JY)
= -g(JY,X) = -\Omega(Y,X)$. Hence, $\Omega$ defines a two-form, called the
\textit{Kh\"aler form} of a Hermitian metric g.
\end{definition}

\begin{definition}[K\"ahler Manifold]
  A \textit{K\"ahler manifold} is a Hermitian manifold $(M,g)$ whose K\"ahler
  form $\Omega$ is closed, $d\Omega=0$. The metric $g$ is called the
  \textit{K\"ahler metric} of M.
\end{definition}

\begin{remark}
Not all complex manifolds admit K\"ahler metrics
\end{remark}

\begin{theorem}
A Hermitian manifold $(M,g)$ is a K\"ahler manifold if and only if the almost
  complex structure $J$ satisfies
\begin{equation}
\nabla_\mu J = 0
\end{equation}
where $\nabla_\mu$ is the Levi-Cevita connection associated with $g$.
\end{theorem}

\begin{proof}
We first note that for any $r$-form $\omega$, d$\omega$ is written as
\begin{equation}
  d\omega = \nabla\omega \equiv
  \frac{1}{r!}\nabla_\mu\omega_{\nu_1\cdots\nu_r}dx^\mu\wedge
  dx^{\nu_1}\wedge\cdots\wedge dx^{\nu}    
\end{equation}
Now we prove that $\nabla_\mu J = 0$ if and only if $\nabla_\mu\Omega = 0$. We
verify the following equalities:
\begin{align}
(\nabla_Z\Omega)(X,Y)
  &= \nabla_Z\left[\Omega(X,Y)\right]-\Omega(\nabla_ZX,Y)-\Omega(X,\nabla_z Y)\\
  &= \nabla_Z\left[g(JX,Y)\right] - g(J\nabla_Z X,Y) - g(JX, \nabla_Z Y)\\
  &= (\nabla_Z g)(JX,Y) + g(\nabla_ZJX,Y) - g(J\nabla_ZX,Y)\\
  &= g(\nabla_ZJX-J\nabla_ZX,Y) = g((\nabla_ZJ)X,y)
\end{align}
where $\nabla_Z g = 0$ has been used. Since this is true for any $X,Y,Z$, if
  follows that $\nabla_Z\Omega = 0$ if and only if $\nabla_Z J = 0$.
\end{proof}
The last theorem shows that the Riemann structure is compatable with the
Hermitian structure in the K\"ahler manifold.

We can also characterize K\"ahler manifolds as Hermitian manifolds for which
the Cristoffel symbols of the Levi-Chevita connection are pure. In other words,
$\Gamma_{jk}^i$ and $\Gamma_{\bar{j}\bar{k}}^{\bar{i}}$ may be non zero, but
all "mixed" symbols like $\Gamma_{jk}^{\bar{i}}$, for example, vanish. This
means that (anti-)holomorphic vectors get parallel transported to
(anti-)holomorphic vectors.

K\"ahler manifolds are manifolds on which we can always find
a \textit{holomorphic} change of coordinates which, at some given point, sets
the metric to its cannonical form, and its first derivatives to zero.

Equivalently, an $n$-dimensional K\"ahler manifold are precisely
$2n$-dimensional Riemannian manifolds with holonomy group contained in $U(1)$
\begin{definition}[Hopf Surface]
Let $\mathbb{Z}$ act on $\mathbb{C}^n \backslash \{0\}$ by $\left(z_1,\cdots,z_n\right)\rightarrow \left(\lambda^k z_1,\cdots,\lambda^k z_n\right)$ for $k\in\mathbb{Z}$. For $0 < \lambda < 1$ the action is free and discrete. The quotient complex manifold $X = \left(\mathbb{C}^n\backslash\{0\}\right)/\mathbb{Z}$ is diffeomorphic to $S^1\times S^{2n-1}$. For $n=1$ this manifold is isomorphic to a complex torus $\mathbb{C}/\Gamma$. The lattice $\Gamma$ can be determined explicitly.\\
In other words,  a Hopf manifold is obtained as a quotient of the complex vector space (with zero deleted) $\mathbb{C}^n \backslash \{0\}$ by a free action of the group $\Gamma \cong \mathbb{Z}$ of integers, with the generator $\gamma$ of $\Gamma$ acting by holomorphic contractions.
\end{definition}


%\section{Glossary}
%\begin{Glossary}

%\end{Glossary}
