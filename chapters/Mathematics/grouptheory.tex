%\chapterauthor{Second Author}{Second Author Affiliation}
\chapter{Group Theory}
\chapterauthor{Ivo Iliev\\Sofia University}
\adjustmtc
\minitoc
\epigraph{"The Universe is an enormous direct product of resentations of
symmetry groups."}{Hermann Weyl}
\section{Basic Definitions}
\begin{definition}[(Group) Homomorphism]
Let $(G, *)$ and $(H,\cdot)$ be two groups. A (group) homomorphism from $G$ to $H$ is a function $h:G\rightarrow H$ such that for all $x,y$ in $G$ it holds that
$h(x*y) = h(u)\cdot h(v)$
\end{definition}
\begin{definition}[Coset]
Let $G$ be a group and $H$ is a subgroup of $G$. Consider an element $g \in G$. Then, $gH = \{ gh : h\in H\}$ is the \textit{left coset} of $H$ in $G$ with respect to $g$, and $Hg = \{hg : h\in H\}$ is the \textit{right coset} of $H$ in $G$ with respect to $g$
\end{definition}
\begin{remark}
In general the left and right cosets are not groups.
\end{remark}

\begin{definition}[Normal Subgroup]
A subgroup $H$ of $G$ is called \textit{normal} if and only if the left and right sets of cosets coincide, that is if $gH = Hg$ for all $g\in G$ 
\end{definition}

\begin{definition}[Representation]
  A \textit{representation} of a group $G$ on a vector space $V$ over a filed
  $K$ is a group homomorphism from $G$ to $GL(V)$, the general linear group on
  $V$. That is, a representation is a map
  \begin{equation}
    \rho:G\rightarrow GL(V)
  \end{equation}
such that
\begin{equation}
  \rho(g_1g_2) = \rho(g_1)\rho(g_2), \quad \forall g_1,g_2 \in G.
\end{equation}
\end{definition}

$V$ is often called the \textit{representation space} and the dimension of $V$
is called the \textit{dimension} of the representation. It is common practice
to refer to $V$ itself as the representation when the homomorphism is clear
from the context.

\begin{example}
  Consider the complex number $u = e^{2\pi/3}$ which has the property $u^3=1$.
  The cyclic group $C_3 = \{1,u,u^2\}$ has a representation $\rho$ on
  $\mathbb{C}^2$ given by:
  \begin{equation}
    \rho(1) = \begin{bmatrix} 1 & 0\\ 0 & 1\end{bmatrix},\quad 
    \rho(u) = \begin{bmatrix} 1 & 0\\ 0 & u\end{bmatrix},\quad
    \rho(u^2) = \begin{bmatrix} 1 & 0\\ 0 & u^2\end{bmatrix},\quad.
  \end{equation}
Another representation for $C_3$ on $\mathbb{C}^2$, isomorphic to the previous
one, is
  \begin{equation}
    \rho(1) = \begin{bmatrix} 1 & 0\\ 0 & 1\end{bmatrix},\quad 
    \rho(u) = \begin{bmatrix} u & 0\\ 0 & 1\end{bmatrix},\quad
    \rho(u^2) = \begin{bmatrix} u^2 & 0\\ 0 & 1\end{bmatrix},\quad.
  \end{equation}
\end{example}

\begin{definition}[Subrepresentation]
  A subspace $W$ of $V$ that is invariant under the group action is called
  a subrepresentation.
\end{definition}

\begin{definition}[(Ir)reducible representation]
  If $V$ has exactly two representations, namely the zero-dimensional subspace
  and $V$ itself, then the representation is said to be \textit{irreducible};
  if it has a proper representation of nonzero dimension, the representation is
  said to be \textit{reducible}. The representation of dimension zero is
  considered to be neither reducible nor irreducible.
\end{definition}

\begin{definition}[Quotent Group]
Let $N$ be a normal subgroup of a group $G$. We define the set $G/N$ to be the set of all left cosets of $N$ in $G$, i.e., $G/N = \{aN:a\in G\}$. Define an operation on $G/N$ as follows. For each $aN$ and $bN$ in $G/N$, the product of $aN$ and $bN$ is $(aN)(bN)$. This defines an operation of $G/N$ if we impose $(aN)(bN) = (ab)N$, because $(ab)N$ does not depend on the choice of the representatives $a$ and $b$: if $xN=aN$ and $yN=bN$ for some $x,y \in G$, then:

(ab)N = a(bN) = a(yN) = a(Ny) = (aN)y = (xN)y = x(Ny) = x(yN) = (xy)N

Here it was used in an important way that $N$ is a normal subgroup. It can be shown that this operation on $G/N$ is associative, has identity element $N$ and the inverse of an element $aN \in G/N$ is $a^{-1}N$. Therefore, the set $G/N$ together with the defined operation forms a group; this is known as the \textit{quotient group} or \textit{factor group} of $G$ by $N$
\end{definition}
\section{Lie groups and representations}
Every symmetry group that we work with in field theory is a Lie group.
We will focus on $SU(N)$ as a working example, but everything generalizes
for all other Lie groups. $SU(N)$ consists of (Hermitian and unit determinant)
matrices of size $N\times N$ that act on $\mathbb{C}^n$. $SU(N)$ has
$dim(SU(N))$ basis elements and each of these basis elements is an $N\times N$
matrix. Note that $dim(SU(N))\neq N.$
\par For every Lie group there is an associated Lie algebra (sometimes called
the tangent space in the literature). The Lie algebra of $SU(N)$ is denoted by
$\mathfrak{su}(N)$.$\mathfrak{su}(N)$ has the same dimension as $SU(N)$ has the
same dimension as $SU(N)$. We denote a basis of $\mathfrak{su}(N)$ by
$T^a,\;a=1,\cdots,dim(SU(N))$. Note that $T^a$ are also $N\times N$ matrices.
Every Lie algebra has an additional structure called the \textit{Lie Bracket},
which classifies the algebra, denoted by
\begin{definition}[Lie Bracket]
  \begin{equation}
    \left[T^a, T^b\right]=f^{ab}_cT^c
  \end{equation}
\end{definition}
$f^{ab}_c$ are called the \textit{structure constants} of the Lie algebra. They
encode the structure of the group. We can gen an element of the Lie group by
exponentiating an element of the Lie algebra:
\begin{gather}
  \mathrm{Lie\; algebra}\xrightarrow{exp}\mathrm{Lie\; group}\nonumber\\
  exp(i\omega_aT^a)\in SU(N)
\end{gather}
We have chosen $T^a$ to be Hermitian, resulting in a factor of $i$ in the
exponent. If we choose $T^a$ to be anti-Hermitian, then there would be no
factor of $i$. However, we always work with Hermitian operators, so there is
always a factor of $i$. Since an element of the Lie group can be found by
exponentating some element of the Lie algebra and we can expand any element of
the Lie algebra in terms of the basis of the Lie algebra (for example, $T^a$),
then this basis generates any element of the Lie algebra and by exponentating,
generates any element of the Lie group (that is connected to the
origin)\footnote{A group can be split up into different components. The Lorentz
  group, for example, consists of 4 disconnected components. One of this
  components, called the proper orthochronous Lorentz group $SO(1,3)^+$ is
  connected to the origin. Any relativistic theory will be invariant under
  $SO(1,3)^+$ (or $SO(3,1)$, depending on convention) but not neccesarily under
  the other components. Other groups, such as $SU(N)$ have only one component,
  which is connected to the origin. As such, exponenting the Lie algebra
generates all of $SU(N)$.} For this reason, the basis of the Lie algebra are
called the \textit{generators}.
\par However, $SU(N)$ only acts on $\mathbb{C}^N$. What if we want a group with
the same structure (constants) as $SU(N)$\footnote{It should be noted that $SU(N)$ have different
structure constants for different $N$} but that act on different spaces such as 
$\mathbb{C}^m$, for $m\neq N$? Representations of $SU(N)$ allow us to do this.
They allow us to construct groups with the same structure (constants) as
$SU(N)$ but which act on different spaces.
\subsection{Representations}
A representation of a group $\mathcal{G}$ is a homomorphism from $\mathcal{G}$
to the space of linear maps acting on a representation space,
$\mathcal{V}_{rep}$:
\begin{equation}
  \rho:\mathcal{G}\rightarrow GL(\mathcal{V}_{rep})
\end{equation}
Every linear map can be thought of as a matrix and so an element of
$\rho(\mathcal{G})$, say $R$, can be thought of as a matrix.
$\rho(\mathcal{G})$ is a group of matrices and is called
a \textit{representation} of the group. Each $R$ acts on $\mathcal{V}_{rep}$:
\begin{equation}
  R:\mathcal{V}_{rep}\rightarrow V_{rep}
\end{equation}
Since $R$ is a matrix that acts on $\mathcal{V}$, it has to be a matrix of size
$dim(\mathcal{V}_{rep})\times dim(\mathcal{V}_{rep})$. The degree of the
representation is the dimension of the representation space:
\begin{equation}
  deg(\rho) = dim(\mathcal{V}_{rep})
\end{equation}
So the representation $\rho(\mathcal{G})$ has dimension equal to
$dim(\mathcal{G})$ and each element of $\rho(\mathcal{G})$ is a square matrix
of size $dim(\mathcal{V}_{rep})$. We have to be careful when talking about the 
dimensions of $\mathcal{G}$ and $\mathcal{V}_{rep}$ as it is easy to get
confused when considering both. Examples will help clarify these concept.
Different representations can act on different spaces. Representations are
useful for describing how different states transform under the action of a 
particular symmetry.

\subsection{Some noteworthy points}
\begin{itemize}
  \item The number of basis elements in the representation is the same as the
    number of basis elements of the group, $\mathcal{G}$. Another way of
    saying this is $dim(\rho(\mathcal{G}))=dim(\mathcal{G})$.
  \item A matrix in the representation, say $R$, acts on a representation
    space, $\mathcal{V}_{rep}$, by matrix multiplication. An element of
    $\mathcal{V}_{rep}$ can be considered as a column vector of
    $dim(\mathcal{V}_{rep})$. So $R$ has to be a matrix of size
    $dim(\mathcal{V}_{rep})\times dim(\mathcal{V}_{rep})$. This is why
    we have to seperate out the dimension of $\rho$ from the degree. The
    examples below will clarify this idea.
  \item Recap: The representation of the symmetry group can be described using
    the generators. The generators are a basis for the representation on the
    Lie algebra. The number of generators is the number of basis elements of
    the symmetry group $\;= dim(\mathcal{G})$. Each generator has size
    $dim(\mathcal{V}_{rep})\times dim(\mathcal{V}_{rep})$. Each representation
    has the same structure constants.
\end{itemize}
\subsection{Example}
If we consider the group $\mathcal{G}=SU(2)$ when thinking about spin angular
momentum, this can give an insight into how representations work. For $SU(N)$,
we normally work with irreducible Representation spaces. Consider each
irreducible
representation space labelled by $j$ seperately, i.e.
$\mathcal{V}_{rep}=\mathcal{V}_j$. Call the $3=dim(SU(2))$ generators of
the representation $T^a=J^a$, where we normally call $J^1=J^x, J^2=J^y,
J^3=J^z$ angular momentum. Examine the following table in detail to understand
how the generators can be different sized matrices.
\\
\begin{table}[h!]
  \begin{tabular}{|p{1cm}|p{2cm}|p{4cm}|p{5cm}|}
    \hline
    $j$ & basis for $\mathcal{V}_j$ & $dim(\mathcal{V}_j)=$ degree of rep
        & $T^a = J^a$  \\ 
    \hline
    $j=0$ & $\ket{0,0}$ & 1 & $T^a=0$, so that it doesn't transform under SU(2)   \\ 
    \hline
    $j=\frac{1}{2}$ & $\ket{\frac{1}{2},m}$ & 2 & $T^a$ are 2x2 matrices, equal
    to $\sigma^a/2$, where $\sigma$ are the Pauli matrices  \\ 
    \hline
    $j=1$ & $\ket{1,m}$ & 3 & $T^a$ are 3x3 matrices, equal to the
    generalization of the Pauli matrices  \\ 
    \hline
  \end{tabular}
  \caption{Some representations for SU(2) angular momentum}\label{tab1}
\end{table}
It is important to notice that there are always \textbf{only} $dim(SU(2))=3$
generators $T^a$. However, these generators can be different sized matrices
depending on which representation space $\mathcal{V}_j$ the generators
act on. Each representation has the same structure constants. As such, each
representation has the same structure of the original group but allows
us to act on different spaces.
\subsection{More points}
\begin{itemize}
  \item In general, consider $\chi_1\in\mathcal{V}_{rep}$ and let $T^a$ be a
    generator of a representation of the group, then it acts on the
    representation
    space as:
    \begin{equation}
      T^a\chi_{1}=\chi_2\in\mathcal{V}_{rep}
    \end{equation}
  \item One way to know what representation you are in is to see what
    size matrices the generators $T^a$ are. Some examples are given later.
    If the generators are $m\times m$ matrices, then an element of the
    representation
    of the group, $g$, can be written as $g=exp{(\lambda^a T^a)}$. So $g$ is
    also an $m\times m$ matrix, i.e, every representation of the Lie algebra
    gives a representation of the group by exponentating. Since the generators
    must act on elements of the representation space, and  the generators are
    matrices of size $m\times m$, the elements of the representation space must
    be $m$ component column vectors.

  \item A gauge theoroy has the Lagrangian
    $\mathcal{L}=\bar{\Psi}\left(i\gamma^\mu D_\mu -m\right)\Psi$, where the
    field $\Psi$ is in a representation space of the gauge group and $D_\mu$ is
    the covariant derivative. $D_\mu$ is defined to transform under an element
    of the representation, $U$, of the gauge group as
    \begin{equation}
      D_\mu\Psi\rightarrow U D_\mu \Psi = (U D_\mu U^{-1})(U\Psi)
    \end{equation}
    and so $D_\mu\rightarrow U D_\mu U^{-1}$. This transformation property is
    called the adjoint action of the group on the Lie algebra. Because of
    this transform property, the covariant derivative is said to transform
    in th adjoint action, or just adjointly. $D_\mu=\partial_\mu - igA_\mu(x)$,
    where $A_\mu(x)$ is an element of the representation of the Lie algebra. We
    can expand $A_\mu(x)$ in a basis of the representation so that $A_\mu(x)
    = A^a+\mu(x)T^a$, where $A^a_\mu(x)$ are the gauge field coefficients with
    corresponding (matrix) generators $T^a$. The gauge field $A^a_\mu$ are said
    to transform under the adjoint action in order to make the covariant 
    derivative transform correctly. The transformation is the standard
    transformation:
    \begin{equation}
      A_\mu(x)\rightarrow U(x) A_\mu U(x)^\dagger - \frac{1}{e}(\partial_\mu
      U(x))U(x)^\dagger 
    \end{equation}
    This is precisely the right transformation to make the covariant derivative
    transform as required
    \begin{equation}
      D_{\mu}\chi(x)\rightarrow U(x)D_\mu \Psi(x)
    \end{equation}
    The generators \textbf{do not} have to be  in the adjoint representation of
    the group and in general they are not. They just have to transform
    adjointly.
  \item We want to consider infinitesmal transformations. Let $U=exp(\lambda^a
    T^a) = 1+ \lambda^a T^a +$ higher order terms. Call $\lambda = \lambda_a
    T^a$. We know
    \begin{align}
      A_\mu(x) &\rightarrow UA_\mu U^{-1} - \frac{1}{e}(\partial_\mu
      U)U^{-1}(x)\\
      &= A_\mu - \frac{1}{e}\left\{\partial_\mu\lambda + e\left[A_\mu,
  \lambda\right]\right\}\\
      &= A_\mu - \frac{1}{e}D_\mu\lambda
\end{align}
    Because of  laziness, we don't want to write out $\left[A_\mu,
    \lambda\right]$ all the time. We give it a special name: the adjoint action
    of the Lie algebra on itself. Write it $A_\mu\cdot\lambda = \left[ A_\mu,
    \lambda\right]$ or $(A_\mu\cdot\lambda)^{ad} = \left[A_\mu,
  \lambda\right]$. This comes from the adjoint action of the group on the Lie
  algebra in infinitimesal form. But we can expand in a basis, $A_\mu - A^a_\mu
  T^a, \lambda=\lambda^a T^a$. The $A^a_\mu, \lambda^a$ are field coefficents 
  when expanded in the (matrix) basis $T^a$. They are fields which commute. But
  $T^a$ are matrices which in general don't commute. So
  \begin{align}
    D_\mu\lambda &= \partial_\mu\lambda + (A_\mu\cdot\lambda)^{ad}\nonumber\\
                 &= \partial_\mu\lambda + e\left[A_\mu, \lambda\right]\nonumber\\
                 &= \partial_\mu\lambda^a T^a + e A^a_\mu\lambda^b\left[T^a,
  T^b\right]\nonumber\\
                 &= \partial_\mu\lambda^a T^a + eA^a_\mu\lambda^b f^{ab}_c
T^c\quad\mathrm{using\;}\left[T^a, T^b\right]=f^{ab}_{c} T^c
  \end{align}
  In component form, the covariant derivative acts on an element of the Lie
  algebra as
  \begin{align}
    (D_\mu\lambda)_a &= \partial_\mu\lambda_a
    + (A_\mu\cdot\lambda)_a^{ad}\nonumber\\
                     &= \partial_\mu\lambda_a + e\left[A_\mu,
  \lambda\right]_a\nonumber\\
                     &= \partial_\mu\lambda_a + eA^b_\mu\lambda^c f^{bc}_a
  \end{align}
  \item The main point to  focus on here is the adjoint action. Do not be
    confused between the adjoint $((A_\mu\cdot\lambda)^{ad} = \left[A_\mu,
    \lambda\right])$ and  the adjoint representation. They are two different
    things. The adjoint action is true for all representations, i.e, for all
    $m\times m$ matrices $T^a$. But the  adjoint re[representation is
    a specific representation of the group. In the adjoint representation, the
    matrices have a specific size, where $m=dim(G)$. This  is standard
    (sometimes confusing) terminology.
\end{itemize}
\subsection{Certain Types of Representations}
When thinking about representations of $SU(N)$ as $m\times m$ matrices,
there are three natural numbers to consideer for $m$. One is $m=1$, i.e, a
scalar. This is called the trivial representation. Another is $m=n$, where
the $SU(n)$ matrices themselves are the representation. This is called
the fundamental representation. The last is when $m=dim(SU(n))$. This is
called the adjoint representation.
\par The \textbf{trivial} representation is whre $\rho(g)=0$ so that all
generators of the representation have $T^a=0$. This is the $j=0$ representation 
of $SU(2)$. $\ket{j=0,0}$ is said to be a singlet or scalar as it doesn't
transform under $SU(2)$.
\par The \textbf{fundamental} representation is the representation which
is equal to the group. This is the homomorphism $\rho(g)=g$. In the fundamental
representation, when considering $SU(n)$, the representation
$\rho(SU(n))=SU(n)$, i.e. the $SU(n)$ matrices are the representation matrices.
The representation is classified by $n\times n$ matrices. $SU(n)$ must act on
a column vector of size $n$ and hence the dimension of the representation space
must be $n$. This is the $j=1/2$ representation of $SU(2)$.
\par In the \textbf{adjoint} representation, it can be shown that the
generators have the component form $(T^a_{ad})_{bc} = f^a_{bc}$, where
$f_{bc}^a$ are the anti-symmetric structure constants. If we know the
structure constants, then we can construct the adjoint representation. E.g,
$f_{abc} = \varepsilon_{abv}$ for $SU(2)$.

\subsection{Usefulness of Lie group representations}
Representation spaces are used to clarify spin and isospin using
representations of $SU(2)$. Hadrons are described by representation spaces of
$SU(3)_{\mathrm{flavour}}$, for example the eight fold way. Representation
spaces of the Poincar\'e group are classifeid in terms of a particles spin
and momentum.
\par Historically, weak decays were seen to involve flavour changing decays and 
break parity in experiment. We want to construct a Lagrangian to describe weak
decays and to make theoretical predictions. Parity sends left handed fields to
right handed fields and so we put the left handed components of each generation
of matter into a two component column vector. We call this $L$. As an example,
for the electron and neutrino we have $L=(\nu_e, e_L)^T$. We require that
the terms involving $L$ in the Lagrangian be invariant under a $SU(2)$ gauge
symmetry in order to introduce gauge fields which mediate a force between the
components. We call this $SU(2)_{\mathrm{Left}}$. The kinetic term for $L$
contains a covariant derivative in the form given above. It contains generators 
for the representation of $SU(2)_{\mathrm{L}}$. These generators have to act on
$L$. $L$ is two component and so we see that the generators, $T^a$ must be
$2\times 2$ matrices. This is justthe fundamental representation of
$SU(2)_\mathrm{L}$. So $L$ is said to be in the fundamental representation.
\par Now, in order to maximally break parity, we do not want the right handed
parts of the matter fields to feel force. So we do not put the right handed
parts of matter fields into a doublet. The  right handed part of the matter 
fields do not transform under $SU(2)_\mathrm{L}$. They are in the trivial or
singlet representation of $SU(2)_\mathrm{L}$. This is the $j=0$ representation
of $SU(2)$ discussed above.
\par In QCD, we introduce a new symmetry called $SU(3)_\mathrm{colour}$. We say
that each quark field $\Psi$ now comes with a colour: either red, blue or
green. Put the different coloured quark fields into a column vector $\chi
= \left(\Psi_\mathrm{red},\Psi_\mathrm{blue}\Psi_\mathrm{green}\right)^T$. For 
example, using the up-type quark
$\chi\left(u_\mathrm{red},u_\mathrm{blue},u_\mathrm{green}\right)^T$. Make
a Lagrangian that will be invariant udner a $SU(3)_\mathrm{colour}$ gauge
symmetry:
\begin{equation}
  \mathcal{L} = \bar{\chi}\left(i\gamma^\mu D_\mu -m\right)\chi.
\end{equation}
Since the $\chi$ is three component, the generators for the representation of
the $SU(3)_\mathrm{colour}$ symmetry will have to act on $\chi$ and so these
generators will have to be $3\times 3$ matrices. This is the fundamental
representation
of the group $SO(3)_\mathrm{colour}$. The quark fields are said to be in the
fundamental representation of $SO(3)_\mathrm{colour}$. Leptons do not feel the
strong force eand so transform trivially under $SO(3)_\mathrm{colour}$. This is
the representation where the generators for $SO(3)_\mathrm{colour}$ are
$T^a=0$. It is similar to the right handed fields in the $SU(2)_\mathrm{Left}$
representation.
\par When working in the fundamental representation, it is ok to think of the 
group representation transformations just as the group transformations and the
representation of the Lie algebra as just the Lie algebra.


\section{Automorphic Forms}
In this section we provide a pedestrian introduction to automorphic forms and
theta series, emphasizing on examples important for string theory and M-theory.
Automorphic forms play an important role in physics, especially in the context
of string theory and M-theory dualities. Notably, U-dualities, first discovered
as symmetries of classical toroidal compactifications of 11-dimensional
supergravity by Cremmer and Julia [1] and later on elevated to quantum
postulates by Hull and Townsend [2], motivate the study of automorphic forms
for exceptional arithmetic groups $E_n(\mathbb{Z}) (n=6,7,8\;\mathrm{ or\;
their }\;A_n\;\mathrm{ and }\;D_n$ analogues for $1\leq n\leq 5)$ - see \textit{e.g} [3]
for a review of U-duality. These chapter is aimed to provide a pedestrian
introduction to these seemingly abstract mathematical objects, designed to offer
a concrete footing for physicists. The basic concepts are introduced via the
simple $Sl(2)$ Eisenstein (not to be confused with Einstein!) and theta series.
The general construction of continuous representations and of their
accompanying Eisenstein series is detailed for $Sl(3)$. Thereafter, unipotent
representations and their theta series for arbitrary simply-laced groups are
presented.
\subsection{Eisenstein and Jacobi Theta series}
The general mechanism underlying automorphic forms is best illustrated b taking
a representation-theoretic tour of two familiar $Sl(2,\mathbb{Z})$ examples:
\subsubsection{$Sl(2,\mathbb{Z})$ Eisenstein Series}
Our first example is the non-holomorphic Eisenstein series
\begin{equation}
  \mathcal{E}^{Sl(2)}_s(\tau) = \sum_{(m,n)\in\mathbb{Z}^2\backslash
  (0,0)}\left(\frac{\tau_2}{\abs{m+n\tau}^2}\right)^s
  \label{eq:eisensteinseries}
\end{equation}
which, for $s=3/2$, appears in string theory as the description of the
complete, non-perturbative, four-graviton scattering amplitude at low energies
[6]. It is a  function of the complex modulus $\tau$, taking values on the
Poincar\'e upper half plane, or equivalently points in the symmetric space
$\mathcal{M} = K\backslash G=SO(2)\backslash Sl(2,\mathbb{R})$ with coset representatie
\begin{equation}
  e = \frac{1}{\sqrt{\tau_2}}
  \begin{pmatrix}
    1 & \tau_1\\
    0 & \tau_2
  \end{pmatrix}
  \label{eq:linearrepresentation}
  \in Sl(2,\mathbb{R})
\end{equation}
The Eisenstein series \eqref{eq:eisensteinseries} is invariant under the
modular transformation
\begin{equation}
  \tau \rightarrow (a\tau + b)/(c\tau + d)
\end{equation}
which is the right action of $g\in Sl(2,\mathbb{Z})$ on $\mathcal{M}$.
Invariance follows simply from that of the lattice
$\mathbb{Z}\times\mathbb{Z}$. This setup may be formalized by introducing:

\begin{enumerate}[label=(\roman*)]
  \item The linear representation $\rho$ of $Sl(2,\mathbb{R})$ in the space
    $\mathcal{H}$ of functions of two variables $f(x,y)$,
    \begin{equation}
      \left[\rho(g)\cdot f\right](x,y) = f(ax+by, cx+dy),\quad g=
      \begin{pmatrix}
        a & b\\
        c & d
      \end{pmatrix}
      ,\quad
      ad-bc =1.
    \end{equation}
  \item An $Sl(2,\mathbb{Z})$-invariant distribution
    \begin{equation}
      \delta_{\mathbb{Z}}(x,y) = \sum_{(m,n)\in\mathbb{Z}\backslash (0,0)}
    \delta(x-m)\delta(y-n)
    \label{eq:sumsldistribution}
  \end{equation}
  in the dual space $\mathcal{H}^*$.
\item A vector
  \begin{equation}
    f_K(x,y)=(x^2+y^2)^{-s}
  \end{equation}
  invariant udner the maxiamal compact subgroup $K = SO(2) \in
  G = Sl(2,\mathbb{R})$.
\end{enumerate}
The Eisenstein series may now be recast in a  general notation for automorphic
forms
\begin{equation}
\mathcal{E}_s^{Sl(2)}(e) = \langle \delta_{\mathbb{Z}}, \rho(e)\cdot
f_K\rangle,\quad e\in G.
\label{eq:automorphiceisenstein}
\end{equation}
The modular invariance of $\mathcal{E}_s^{Sl(2)}$ is now manifest: under the
right action $e\rightarrow eg$ of $g\in Sl(2,\mathbb{Z})$, the vector
$\rho(e)\cdot f_k$ transforms by $\rho(g)$, which in turn hits the
$Sl(2,\mathbb{Z})$ invariant distribution $\delta_{\mathbb{Z}}$. Furhtermore,
\eqref{eq:automorphiceisenstein} is ensured to be a function of the
\textit{coset} $K\backslash G$ by invariance of the vector $f_K$ under the
maximal compact $K$. Such a distinguished vector is known as
\textit{spherical}. All the automorphic forms we shall encounter can be written
in terms of a triplet $(\rho,\delta_{\mathbb{Z}},f_K)$.
\par Clearly any other function of the $SO(2)$ invariant norm
$\abs{x,y}_\infty\equiv \sqrt{x^2+y^2}$ would be as good a candidate for
$f_K$. This reflects the reducibility of the representation $\rho$ in
\eqref{eq:linearrepresentation}. However, its restriction to homogeneous, even
functions of degree $2s$,
\begin{equation}
  f(x,y) = \lambda^{2s} f(\lambda x, \lambda y) = y^{-2s}f(\frac{x}{y},1),
\end{equation}
is irreducible. The restriction of the representation $\rho$ acts on the space
of functions of a single variable $z=x/y$ by weight $2s$ conformal
transformations $z\rightarrow(az+b)/(cz+d)$ and admits $f_K(z) = (1+z^2)^{-s}$
as its unique spherical vector. In these variables, the distribution
$\delta_{\mathbb{Z}}$ is rather singular as its support is on all rational
values $z\in\mathbb{Q}$. A related problem is that the behaviour of
$\mathcal{E}_s^{Sl(2)}(\tau)$ at the cusp $\tau\rightarrow i\infty$ is
difficult to assess - yet of considerable interest to physicists being the
limit relevant to non-perturbative instantons.
\par These two problems may be evaded by performing a Poisson resummation on
the integer $m\rightarrow\tilde{m}$ in the sum of the $Sl(2,\mathbb{Z})$
invariant distribution \eqref{eq:sumsldistribution}, after first separating
out terms with $n=0$. The results may be rewritten as a sum over the single
variable $N=\tilde{m}n$, except for two degenerate - or "perturbative"
- contributions:
\begin{multline}
  \mathcal{E}_s^{Sl(2)} = 2\zeta(2s)\tau_2^s
  + \frac{2\sqrt{\pi}\tau_2^{1-s}\Gamma(s-1/2)\zeta(2s-1)}{\Gamma(s)}\\
  + \frac{2\pi^s\sqrt{\tau_2}}{\Gamma(s)}\sum_{N\in\mathbb{Z}\backslash\left\{0\right\}}\mu_s(N)N^{s-1/2}K_{s-1/2}(2\pi\tau_2
    N)e^{2\pi i\tau_1 N}.
\end{multline}
In this expression, the summation measure
\begin{equation}
  \mu_s(N) = \sum_{n\vert N}n^{-2s+1}
  \label{eq:summationmeasure}
\end{equation}
is of prime physical interest, as it is connected to quantum fluctuations in an
instanton background.
\par First focus on the non-degenerate terms in the second line. Analyzing the
transformation properties under the Borel and Cartan $Sl(2)$ generators
$\rho
\begin{pmatrix}
  1&t\\
  0&1
\end{pmatrix}
:\tau_1\rightarrow \tau_1 + t
$
and $\rho
\begin{pmatrix}
  t^{-1} & 0\\
  0 & t
\end{pmatrix}
: \tau_2 \rightarrow t^2\tau_2
$, we readily see that they fit into the framework
\eqref{eq:automorphiceisenstein}, upon identifying
\begin{equation}
  f_K(z) = z^{s-1/2}K_{s-1/2}(z),\quad \delta_{\mathbb{Z}}(z)
  = \sum_{N\in\mathbb{Z}\backslash\left\{0\right\}}\mu_s(N)\delta(z-N),
  \label{eq:shpericalvector}
\end{equation}
and the representation $\rho$ as
\begin{equation}
  E_{+} = iz,\quad E_{-} = i(z\partial_z + 2 -2s)\partial_z,\quad
  H=2z\partial_z + 2 - 2s.
\end{equation}
This is of course equivalent to the representation on homegeneous functions,
upon Fourier transform in the variable $z$. The power-like degenerate terms may
be viewed as regulating the singular value of the distribution $\delta$ at
$z=0$. They may, in principle, be recovered by performing a Weyl reflection on
the regular part. It is also easy to check that the spherical vector condition,
$K\cdot f_K(z) \equiv (E_+ - E_-)\cdot f_K(z) = 0$, is the modified Bessel
equation whose unique decaying solution at $z\rightarrow\infty$ is the
spherical vector in \eqref{eq:shpericalvector}.
\par While the representation $\rho$ and its spherical vector $f_K$ are easily
understood, the distribution $\delta_{\mathbb{Z}}$ requires additional
technology. Remarkably, the summation measure \eqref{eq:summationmeasure} can
be written as an infinite product
\begin{equation}
  \mu_s(z) = \prod_{p\in\;\mathrm{primes}} f_p(z),\quad f_p(z)
  = \frac{1-p^{-2s+1}\abs{z}_p^{2s-1}}{1-p^{-2s+1}}\gamma_p(z).
\end{equation}
(A simple trial computation of $\mu_s(2.3^2)$ will easily convince the reader
of
this equality.) Here $\abs{z}_p$ is the $p$-adic\footnote{A useful physics
  introduction to $p$-adic and adelic fields in [10]. It is worth noting that a 
  special function theory analogous to that over the complex numbers exists for
the $p$-adics.} norm of $z$, i.e $\abs{z}_p = p^{-k}$ with $k$ the largest
integer such that $p^k$ divides $z$. The function $\gamma_p(z)$ is unity
if $z$ is a $p$-adic integer $(\abs{z}_p\leq 1)$ and vanishes otherwise.
Therefore $\mu(z)$ vanishes unless $z$ is an integer $N$. Equation 
\eqref{eq:automorphiceisenstein} can therefore be expressed as
\begin{equation}
  \mathcal{E}_s^{Sl(2)}(e)
  = \sum_{z\in\mathbb{Q}}\prod_{p\in\mathrm{primes},\infty}
    f_p(z)\rho(e)\cdot f_K(z),
    \label{eq:automorphiceisenstein2}
\end{equation}
The key observation now is that $f_p$ is in fact the spherical vector for the 
representation of $Sl(2,\mathbb{Q}_p)$, just as $f_\infty:= f_K$ is the
spherical
vector of $Sl(2,\mathbb{R})$. The reader can be convinced of this important
fact, by evaluating the $p$-adic Fourier transform of $f_p(y)$ on $y$, thereby
reverting to the $Sl(2)$ representation on homogeneous functions: the result
\begin{equation}
  \tilde{f}_p(x) = \int_{\mathbb{Q}_p}\dd z f_p(z)e^{ixz}
  = \abs{1,x}_p^{-2s}\equiv \mathrm{max}(1,\abs{x}_p)^{-2s},
\end{equation}
is precisely the $p$-adic counterpart of the real spherical vector $f_K(x)
= (1+x^2)^{-s}\equiv\abs{1,x}^{-2}_{\infty}$. The analogue of the decay
condition is that $f_p$ should have support over the $p$-adic integers only,
which holds by virtue of the  factor $\gamma_p(y)$. It is easy to check
that the formula \eqref{eq:automorphiceisenstein2} in this representation
reproduces the \eqref{eq:eisensteinseries}.
\par Thus, the $Sl(2,\mathbb{Z})-$invariant distribution $\delta_{\mathbb{Z}}$
can be straightforwardly obtained by computing the  spherical vector over 
all $p$-adic fields $\mathbb{Q}_p$. More conceptually, the  Eisenstein series
may be written \textit{adelically} (or \textit{globally}) as
\begin{equation}
  \mathcal{E}_s^{SL(2)}(e) = \sum_{z\in\mathbb{Q}}\rho(e)\cdot
  f_{\mathbb{A}}(z),\quad f_{\mathbb{A}}
  = \prod_{p=\mathrm{prime},\infty}f_p(z),
\end{equation}
where the sum $z\in\mathbb{Q}$ is over principle adeles\footnote{Adeles are
  infinite sequences $(z_p)_{p=\mathrm{prime},\infty}$ where all but a  finite
  set of $z_p$ are  $p$-adic integers. Principle adeles are constant sequences
  $z_p = z\in \mathbb{Q}$, isomorphic to $\mathbb{Q}$ itself.}, and
  $f_{\mathbb{A}}$ is the spherical vector of $Sl(2,\mathbb{A})$, invariant
  under the maximal compact subgroup $K(\mathbb{A}) =\prod_d
  Sl(2,\mathbb{Z}_p)\times U(1)$ of $Sl(2,\mathbb{A})$. This relation between
  functions on $G(\mathbb{Z})\backslash G(\mathbb{R})/K(\mathbb{R})$ and
  functions on $G(\mathbb{Q})\backslash G(\mathbb{A})/K(\mathbb{A})$ is known
  as the Strong Approximation Theorem, and is a powerful tool in the study of 
  automorphic forms.
  \subsubsection{Jacobi Theta Series}
  Our next example, the Jacobi theta series, demonstates the key role played by
  Fourier invariant Gaussian characters - "\textit{the Fourier transform of the
  Gaussian is the Gaussian"}. Our later generalization will involve cubic type
  characters invariant under Fourier transform.
  \par In contrast to the Eisenstein series, the Jacobi theta series
  \begin{equation}
    \theta(\tau) = \sum_{m\in\mathbb{Z}}e^{i\pi\tau m^2},
  \end{equation}
  is a modular form for a  congruence subgroup $\Gamma_{0}(2)$ of
  $Sl(2,\mathbb{Z})$ with modular weight 1/2 and a non-trivial multiplier
  system. It may, nevertheless, be cast in the framework
  \eqref{eq:automorphiceisenstein}, with a minor caveat. The representation
  $\rho$ now acts on functions of a single variable $x$ as
  \begin{equation}
    E_{+} = i\pi x^2, \quad H = \frac{1}{2}(x\partial_x + \partial_x x), \quad
    E_{-} = \frac{i}{4\pi}\partial^2_x
  \end{equation}
  Here, the action of $E_{+}$ and $H$ may be read off form the usual Borel and
  Cartan actions of $Sl(2)$ ot $\tau$ while the generator $E_{-}$ follows by
  noting that the Weyl reflection $S:\tau\rightarrow -1/\tau$ can be
  compensated by Fourier transform on the integer $m$. The invariance of the
  "comb" distribution $\delta_{\mathbb{Z}} = \sum_{m\in\mathbb{Z}}\delta(x-m)$
  under Fourier transform is just the Poisson resummation formula.
  \par Finally (the caveat), the compact generator $K=E_{+} - E_{-}$ is exactly
  the Hamiltonian of the harmonic oscillator, which notoriously does not admit
  a normalizable zero energy eigenstate, but rather the Fourier-invariant
  ground state $f_{\infty}(x) = e^{-\pi x^2}$ of eigenvalue $i/2$. This
  relaxation of the spherical vector condition is responsible for the
  non-trivial modular weight and multiplier system. Correspondingly, $\rho$
  does not represent the group $Sl(2,\mathbb{R})$, but rather its double cover,
  the metaplectic group.
  \par Just as for the  Eisenstein series, an adelic formula for the summation
  measure exists: note that the $p$-adic spherical vector must be invariant
  under the compact generator $S$ which acts by Fourier transform. Remarkably,
  the function $f_p(x) = \gamma_p(x)$, imposing support on the integers only is
  Fourier-invariant - it is the $p$-adic Gaussian! One therefore recovers the
  "comb" distribution with uniform measure. Note that the $Sl(2)= Sp(1)$ theta
  series generalizes to higher symplectic groups under the  title of Siegel
  theta series, relying in the same way on Gaussian Poisson resummation.
  \subsection{Continuous representations and Eisenstein series}
  The two $Sl(2)$ examples demonstrate that the essential ingredients for
  automorphic forms with respect to an arithmetic group $G(\mathbb{Z})$ are (i)
  an irreducible representation of $\rho$ of $G$ and (ii) corresponding
  spherical vectors over $\mathbb{R}$ and $\mathbb{Q}_p$. We now explain how to
  construct these representations by quantizing coadjoint orbits.aaa
